\documentclass[a4paper]{article}
\usepackage{url}
\usepackage{fullpage}
\usepackage{amsfonts}
\usepackage{amsmath}
\usepackage{amssymb}
\usepackage{makeidx}
\usepackage{theorem}
\usepackage{bbold}
\usepackage{graphicx}
\usepackage{epic}
\usepackage{eepic}

\newtheorem{theorem}{Theorem}[section]
\newtheorem{algorithm}{Algorithm}
\newtheorem{conjecture}[theorem]{Conjecture}
\newtheorem{corollary}[theorem]{Corollary}
\newtheorem{definition}[theorem]{Definition}
\newtheorem{example}[theorem]{Example}
\newtheorem{exercise}{Exercise}[section]
\newtheorem{lemma}[theorem]{Lemma}
\newtheorem{proposition}[theorem]{Proposition}
\newenvironment{proof}[1][Proof]{\textbf{#1.} }{\ \rule{0.5em}{0.5em}}


\date{\today}
\author{Nicky van Foreest}

\title{Simulating Queueing Networks with OMNeT++}

\begin{document}
\maketitle \abstract{\noindent This documents aims at providing a
  quick introduction to simulating queueing   networks with
  OMNeT++. It should get the interested user up and running with a
  simple M/M/1 FIFO queueing demo. After this the user can start
  exploiting more of OMNeT++'s functionality based on the extensive
  user manual included in the standard distribution. The software that
  comes with this document is a rudimentary library of OMNeT++
  functions useful for queueing simulation.}
\tableofcontents

\section{Introduction}
\label{sec:introduction}
OMNeT++ is a discrete event simulation enviroment based on C++. Andr{\'a}s
Varga is the principle author and  currently maintains it.

 When I
started to use it for simulating queueing networks, I was somewhat
awed by the amount of functionality that OMNeT++ provides and the size
of user manual. Besides this, I had no experience with C++
programming, which was an extra hurdle in becoming acquainted with the
power of the tool. Hence I had to invest some time to find out which
parts and functionalities of OMNeT++ are specially useful for queueing
systems analysis, and how to get it working. My hope is that after you
have read this document the burden of learning OMNeT++ is somewhat
lessened and that your appetite has grown towards exploiting it for
your own researches in the realm of queueing networks.

Summarized my goals of this document and software are to:
\begin{itemize}
\item provide a start for queueing systems simulation with OMNeT++
\item give some useful pointers to the user manual of OMNeT++
\item provide a start of an extensive software library that enables
  users to quickly set up queueing simulations.
\end{itemize}

In this tutorial I assume that the reader has experience with
programming, at least with C, and has some basic understanding of
queueing theory, say the first few sections of the chapter on queueing
theory of~\cite{Ross93}.  Furthermore, I will not explain the details
of the code. These are  for the reader to study. This may seem like a
thread, but the code is not that hard to understand.

Perhaps I should have started by mentioning the advantages of using
OMNeT++ here. I refrain from doing this at this point of the document,
but instead postponed it to the end of it. The reason for this is
twofold. In the first place the OMNeT++ user manual and the OMNeT++
homepage mention plenty of examples. IN the second place, I want to
focus on my personal experiences, and this is less important than
having you starting to play with OMNeT++ and this demo.

\subsection{Where to get the software?}
\label{sec:where-get-software}
You can find OMNeT++ at
\url{http://www.hit.bme.hu/phd/vargaa/omnetpp.htm} If you are too lazy
to type this in, a search with \url{goolge.com} on \texttt{omnetpp}
will give an instant hit.

\noindent
If you want to make advanced plots of the simulation results, make
sure to get gnuplot:
\url{http://www.gnuplot.org}

\subsection{Feedback}
\label{sec:feedback}
Please do not hesitate to send me your comments, especially if you
think it contains inaccuracies, unclear passages, etc. When you have
something to contribute to the software library, please inform me so
that I can include it in this package. I will grant full
acknowledgment with respect to authorship.

My email address is: \url{n.d.vanforeest@math.utwente.nl}

OMNeT++ related questions can be posted at the OMNeT++ mailing list:
\url{omnetpp-l@it.swin.edu.au}. Chances are high that Andr{\'a}s himself
answers your questions.


\subsection{Structure of this document}
\label{sec:struct-this-docum}
First of all, in chapter~\ref{sec:simulating-mm1-fifo}, I want to
discuss a working exampe of a queueing simulation in OMNeT++.  The
implemented example is the archetypal M/M/1 FIFO queue. The example
will be slightly baroque in terms of its output--it will be more than
you will typically need---but I want to show as many features as
possibly appropriate with this one example.
Chapter~\ref{sec:next-steps} will hint upon some ways to change the
way the simulation works; the simulation parameters, the simulation
output, and the like.  Chapter~\ref{sec:impl-fifo-simul} will be
devoted to understanding the implementation of the example.  The last
section lists a few of the interesting ways in which the current
example can be extended so as to turn it into a full-flegded library
of queueing simulation tools.


\subsection{Acknowledgements}
\label{sec:acknowledgements}
Andr{\'a}s: Thanks a lot for OMNeT++.


\subsection{Versions}
\noindent First version: Jan 2001

\section{Simulating the M/M/1 FIFO queue}
\label{sec:simulating-mm1-fifo}

\subsection{Getting the simulation to run}
\label{sec:getting-it-running}
The first step is of course to install OMNeT++. Before running the
M/M/1 example it is a good idea to first check that OMNeT++ and all
the software it depens on is installed properly. I usually test this
by running one of the standard samples included in the OMNeT++
distribution, such as the Nim game.

Once you are convinced that OMNeT++ works, run \texttt{opp_makemake -f} in
the directory in which you put the FIFO demo to make a
\texttt{Makefile}. The binary \textbf{opp_makemake} should be provided with OMNeT++.
In case it cannot be found, try running it directly from the
\texttt{src/utils/} directory of the OMNeT++ distribution. Once you
have a \texttt{Makefile}, a simple \texttt{make} should do to get the
binaries for the FIFO simulation.

A general remark is of importance here. In case you change something in one
of the files, you should run \texttt{make} again. When you decide to
copy all files to another directory, or include other files in this
directory, run \texttt{opp_makemake -f } again. As an aside, run
\texttt{make clean} to remove all object files and binaries.

The binary to run has the same name as the directory that contains the
the Makefile that was generated by \texttt{opp_makemake}. In my case this
is \texttt{queues\_}(and then some version number). Run it. If
everything works the way it should, a few windows should pop up.

I consider it a bit too much work to type in \texttt{make} and then
the binary  again and again. Therefore I included a very
simple script \texttt{run} which does this for me.

\subsection{Running the demo}
\label{sec:running-simulation}
Once the simulation has started you should see two windows:
\begin{itemize}
\item A graphics window showing a small queueing network: a job
  generator, a FIFO queue, and a job sink.
\item The other, called \texttt{OMNeT++/TKenv}, containing the main  functionality to
  run the simulation.
\end{itemize}

Double click on the fifo widget on the canvas in the graphics window.
A new box should appear, called \texttt{(Fifo) fifonet.fifo[0]}. Click
on \texttt{Objects/Watches}.  You will see a few lines with text
appear. The line with the string \texttt{(cQueue)} contains the word
\texttt{(empty)}.  This will change to the number of customers in
queue (= one less the number in the system when a customer is served)
once the simulation is running.  Then double click on the line with
the word \texttt{Job Distribution}.  You will see a blank blue (that
is, on my screen) window. Here a histogram will show the probability
density function of the number of jobs in the system as seen by an
arriving job. Finally, click on the line \texttt{Jobs in system}. A
yellow canvas will come up. This will contain a simple plot of the
number of jobs in the system, the one in service included.

Now you can start the simulation by pressing the \texttt{RUN}-button
in the \texttt{OMNeT++/TKenv} window. In the graphics window you should
see jobs hopping from the generator to the queue, and from the queue
to the sink.  Press \texttt{STOP} after several seconds. The first
line of the \texttt{(Fifo) fifonet.fifo[0]} window will now, with probability
$ > 0$, show that the queue contains some customers. When you double
click on this line, another window will tell you which jobs are currently
waiting in the queue.

This may seems not very interesting from a
queueing perspective; the M/M/1 queueing model assumes that
customers do not have real identity---they only differ by their
service requirements and arrival time. However, when studying more
complicated models, for instance when jobs have different type, this
information becomes relevant. It becomes even more important if you
want to study protocol and queue interactions, such as when multiple
TCP sources share buffered resources, i.e., routers.

The histogram window should now contain a few black bars. When you put
your mouse on one, it will change color. (Mine becomes gray.) There is
a line at the bottom of window that contains the number of arrivals
that observed a certain number of jobs in front of it in the system.
In other words, if cell \# 1 contains 10, this means that 10 arrivals
saw one job in the system.

Clicking on the \texttt{Job Distribution} line in the
\texttt{(Fifo) fifonet.fifo[0]} window with the right mouse button, enables
you to choose between a graphical representation of the gathered data,
and a textual one, called \texttt{Object}. I discussed the graphical
one above. The text information will provide you with aggregate statistics
such as the mean number of jobs found upon arrival, etc.

The last step will be to press \texttt{EPRESS} in the
\texttt{OMNeT++/TKenv} window; I leave the other
buttons for you to discover. In fact, my best general advise is to
press on any buttons you may see, and find out what they do. Do not
forget that often both mouse buttons, the left and the right, can be
used for different effects.


\subsection{Analyzing the simulation results}
\label{sec:interpreting-results}
If you have not interrupted the simulation by pressing on one of the
big red \texttt{STOP} buttons, the simulation will finish
after having generated 5000 jobs. On my machine this takes less than a
second, and it is not a particularly fast one.

Notice that the textual and graphical representations of the histogram
are updated after the simulation has finished. The
\texttt{OMNeT++/Tkenv} window contains as well a number of results.
You might need to scroll a bit up and down in the window to view all
the simulation results.

The directory will now contain a few new files as well.
\begin{itemize}
\item \texttt{fifo.sca} This file contains the statistics that were
  gathered in the variable \texttt{jobDist} of class
  \texttt{cDoubleHistogram}, see the user manual section 6.13, such as
  the number of jobs generated, the histogram data, etc. Please have a
  look at it now.  The textual info of the histogram should be
  contained in this file too.
\item \texttt{fifo.vec} This file contains info about the dynamics of
  the number of jobs in the system. You should process this with the
  OMNeT++ tool \texttt{plove}. Chapter 9 contains the instructions.
  Briefly, start up \texttt{plove}. The left most button allows you to
  \texttt{load} the file. Send it to the right window with the arrow
  button in the middle. Press \texttt{PLOT!}. Now \texttt{gnuplot}
  should fire up and show the dynamics of the job distribution. You
  may have to fiddle around with the \texttt{options} to get a nice
  graph.
\end{itemize}

In general you can process these files, with \texttt{awk} for instance,
or another C++ program for that matter, to postprocess it. You might
instead want to do the postprocessing  in the function
\texttt{Fifo::finish}, about which I will talk later. Here comes in
the some of the power    OMNeT++ using  a programming language:
you can incorporate your own postprocessing functions in the
simulation itself, in case you want to.

\paragraph{}
Summarizing, OMNeT++ provides a number of ways to present simulation
data:
\begin{itemize}
\item dynamically, by means of jobs jumping from one node in the
  network to another;
\item graphically, by means of dedicated collection classes such as
  \texttt{cOutVector};
\item verbally, by means of the \texttt{ev <<} statement in the code
  which is output to the  canvas of the \texttt{OMNeT++/Tkenv} window;
\item by means of files.
\end{itemize}

\section{Modifying the basic example}
\label{sec:next-steps}
The above discussed only one type of queue, the M/M/1 queue, with one
specific parameter setting. You
probably want to simulate other types of queues, more generic service
distribution, etc. Another interesting degree of freedom is to speed
up the simulation by leaving out the windows.

\subsection{Changing the  simulation}
\label{sec:changing-basic-mm1}
I will discuss some ways to change and extend the current simulation
environment. The most important way to do this is with the
\texttt{omnetpp.ini} file, see chapter 8 of the user manual for all details.


\paragraph{Changing the interarrival and service rate}
\label{sec:chang-inter-serv}
Change the numbers in \texttt{omnetpp.ini} related to the exponential
distribution. Mind that the interarrival rate and service rate are the
inverse of the parameters you specify, e.g.
\texttt{fifonet.fifo[1].service\_time = exponential(2)} means that the
expected service duration $ = 1/\mu = 2$.

You cannot only change this via the \texttt{omnetpp.ini} file. Another
possiblity is to press \texttt{Params} in the \texttt{(Fifo)
  fifonet.fifo[0]} window. Click for instance on the the line with
\texttt{service\_time}. Now you edit the field in the window that will
appear, for instance try another distribution, see the next paragraph.
Do not forget to press the \texttt{Enter} key on your keypad to make
the edit effective.

\paragraph{Changing the interarrival and service distribution}
\label{sec:chang-inter-serv-1}
OMNeT++ provides a few standard distributions, such as the uniform and
exponential distribution, see sections 4.9.6 and 6.13 of the user
manual. In case you need other distributions, you have to build them
yourself. This is in itself quite interesting, see Ross~\cite{Ross93},
and, of course, Knuth\cite{Knuth97}. Once you know, mathematically
speaking, what to do, OMNeT++ makes the implementation very easy.  You
should define your new distribution in the file
\texttt{distributions.cc}. I have already provided a few simple
examples.  Note that you can use the standard distributions of OMNeT++
in your new functions right away. Once you have implemented your new
distribution, you can simply use its name in \texttt{omnetpp.ini} as
pass appropriate parameters to it. Explicitely, if you want to use, as
an example, the distribution \texttt{perturbedExponential}, which you
can find in the file \texttt{distribution.cc} as
the service distribution, than you should use the following line in
\texttt{omnetpp.ini}:
\texttt{fifonet.fifo[0].service\_time = perturbedExponential(2,1)}
Take any parametersf you like, for this distribution, but check out
the code first in case you do not want to be surprised during the
simulation.

\paragraph{Changing the network}
\label{sec:changing-network}
The \texttt{omnetpp.ini} parameter \texttt{fifonet.num\_buffers}
enables you to make a network of tandem queues. Changing it from the
current value 1 to say $k$, will put $k$ queues in tandem. You should
as well change the line
 \texttt{fifonet.fifo[0].service\_time = exponential(2)}  to
\texttt{fifonet.fifo[*].service\_time = exponential(2)}, that is, the
\texttt{1} has to  change to \texttt{*} to reference all fifos, instead of
just the first one with id zero. I actually chose a slightly different
approach here. First I give \texttt{fifo[0]} its value, and then the
others by means of the \texttt{*}.

You will find included as well a ring network. This is built out of
simple fifo queues that are connected sequentially to each other. If
you want to run it, you have to change a few lines in
\texttt{omnetpp.ini}. This file contains where the changes should be
made. Since the ring is a closed queueing network, there are no
external arrivals of jobs, neither sinks. Therefore these two are not
included in the ring. Furthermore, the buffers should contain some
initial number of jobs, that will start circling around. You can set
these numbers by the parameter \texttt{ring.fifo[*].num\_init\_jobs =
  20} to be found in \texttt{omnetpp.ini}.\footnote{For the
  interested. If you analyze the expected number of jobs in the
  queues, and add them, it will appear as if one job is missing, i.e,
 this is the application of the Arrival Theorem, see e.g.~\cite{Ross93}.}

More general networks are for you to build. Section 4.10 and 6.19 of
the manual will
tell you how. You should as well consult the following OMNeT++
samples. The sample directory \texttt{token} provided with the
standard OMNeT++ distribution shows a circular network. The
\texttt{fddi} sample shows a large network. (Do not forget to click on
one of the rings to see how complicated networks you can actually
simulate with OMNeT++.)


\paragraph{Changing the job scheduling}
\label{sec:chang-job-sched}
Only FIFO scheduling is implemented at the moment. In case you are
interested in building other ones, go ahead.

\paragraph{Changing the random number generator}
\label{sec:chang-rand-numb}
For the more suspicious minds, section 6.9 of the OMNeT++
user manual discusses the random number generator. You can replace
this with your own if you want this.


\subsection{Speeding up the simulation}
\label{sec:speed-up-simul}
If you are convinced that everything works the way it should, and you
are just interested in numerical output, you can run the simulation
straight from the prompt with the \texttt{cmdenv} mode. No more
windows will appear, only the output files will be produced. Due to
this, the simulation will become quicker as well.

To achieve this, change these lines in the  \texttt{Makefile}

\begin{verbatim}
# User interface (uncomment one) (-u option)
#USERIF\_LIBS=\$(CMDENV\_LIBS)
USERIF\_LIBS=\$(TKENV\_LIBS)
\end{verbatim}

to

\begin{verbatim}
# User interface (uncomment one) (-u option)
USERIF\_LIBS=\$(CMDENV\_LIBS)
#USERIF\_LIBS=\$(TKENV\_LIBS)
\end{verbatim}

Do a \texttt{make clean} and \texttt{make} to remove all window
related code from the simulation executable. Now it should work. Be
aware that a new \texttt{opp_makemake -f} reverts the makefile to the old
situation, i.e., the simulation with tk windows.

You will want these options in the following section of
\texttt{omnetpp.ini}:
\begin{verbatim}
[Cmdenv]
runs-to-execute = 1
module-messages = no
verbose-simulation = no
Display-update = 1h
\end{verbatim}
Play with these options to discover that the number of lines of
simulation output will be a bit to much to handle. See section 8.5 of
the manual for more info.

The user manual contains as well some more hints to speed up the
simulation still further, see sections 6.18 and  8.7.

\section{The implementation of the M/M/1 simulation}
\label{sec:impl-fifo-simul}
Before starting the main subject of this chapter, I need to define one
concept: a \emph{functional entity}. A functional entity is a part of
a simulation that carries out a specific action on a job, or a
message. For example, a server or a message generator are functional
units. They are, so to say the essential functional units that take
care of one process step in the lifetime of a job, or message. The
implementation of such entities will be called \emph{modules}. Now
back to the simulator.

The simulation environment is built out of, mainly, two types of
files.  The \texttt{.ned} files roughly describe how the entities
should communicate; the \texttt{.cc} files contain the C++ code that
implement the behavior of the entities. These file types I will
discuss in some more detail below. I expect you to have the
\texttt{.ned} and \texttt{.cc} files belonging to this demo at hand.

After having read this, be sure to check out the discussion of the Nim
game in the manual as well, and the samples \texttt{fifo1} and
\texttt{fifo2}.  They are instructive and show additional
functionality of OMNeT++ relevant for queueing systems analysis.

\subsection{A basic  approach to understanding the code}
There are various ways to try to understand the implementation of a
new simulation, for instance, this demo or one of the sample
simulations. I found the following approach the most useful. First I
run it, of course, to get an understanding of where the various
entities reside and how they exchange jobs. Then I work through the
files belonging to  each
entity  separately. I start with reading the \texttt{.ned} files to
understand what goes in and out of a module, and the parameters it
will need.  Then I give the header files a brief look to become
familiar with the module specific internal variables and functions.
Finally I study the C++ code belonging to the  module. Once I somewhat
understand what it  does, I tackle the next module in the
chain, that is, the module that gets its messages from the one  I
studied. Working this way, I gained a quicker understanding
of what was going on than by first working through all \texttt{.ned}
files, than all the C++ files, etc.

In this document I will not, however, follow the above suggestion,
mainly for brievity. Here I only want to illustrate certain key points
of the simulation, and leave the studying for you.

\subsection{The \texttt{.ned} files}
\label{sec:texttt.ned-files}
Each entity in a simulation needs to communicate via \emph{messages}
with itself and other entities. Messages can be used for various
purposes. One is to represent jobs that need service at queues.
Another is to convey information of the entity's state to itself in
the future, or to other entities. For instance, the fifo example
contains three entities: a source module, a fifo module and a sink
module. The source module generates messages that represent jobs.
These job-messages are sent to the fifo module. This in turn delays
the messages according to the present queue, then services them, and
finally send the messages to the sink module. The sink module
processes these messages to extract some final statistics. The sink is
where the messages leave the queueing network.  The sink module
releases as well the memory allocated to the job messages.

Clearly all these modules need in- and output gates to receive and
send messages. The \texttt{.ned} file specifies these gates. In our
example, the fifo module has an input and an output gate. Furthermore
these modules may need parameters such as service rate, queue size,
etc. These parameters need be \emph{declared}  in the \texttt{.ned} files
too. The \emph{definition} of the parameters, i.e., giving them a
value, takes place when the simulation starts.

The next step is to glue all the modules together to form a network.
This should be done by means of another \texttt{.ned} file that
defines a \emph{compound module}. This new module uses the simple
modules and connects them into networks. Besides this it passes
parameter values on to them during the simulation. Finally it gives
some directives on where on the canvas the simple modules should
appear. To state this in terms of the fifo queue, \texttt{FifoNet},
defined in \texttt{fifonet.ned}, connects the output of the source to
the input of the fifo queue, and the output gate of the queue to the
sink. Consult \texttt{fifonet.ned} for all details. Take especially
notice of the syntax, and be aware of when to use a comma `,', and a
semicolon `;'. I made some time consuming errors by not using them in
the appropriate way.

Note that this hierarchy of
modules---simple modules forming compound modules, forming in turn
other compound modules--provides a very efficient way to build large,
complex queueing networks. Start by building the parts, and then
connect these parts to larger compound modules up to entire networks.

\subsection{The \texttt{.cc} and \texttt{.h} files}
\label{sec:texttt.cc-texttt.h-f}
Now we have defined all entities that take part in the process cycle
of job-messages, and specified the routing between these entities in
the compound modules, we should tell how each entity should process
job-messages. In concrete terms, the generator should generate jobs
with certain interarrival times distributed according to some
specified probability distribution. These jobs are represented by a
message, so that in fact the generator generates messages. Then it
sends these messages out of its output gate. If you will have a look
at \texttt{gen.cc} you will see that it will produce
\texttt{fifonet.gen.num\_messages } messages, each separated some time
\texttt{ia\_time} apart. These two parameters are defined in the file
\texttt{omnetpp.ini}.

\texttt{fifo.cc} contains the heart of the simulation. Here jobs
arrive, and depending on whether the server is idle or not, have to
spend time in the queue. Lets first consider the function
\texttt{Fifo::handleMessage} and see what it does. Suppose the server
is busy. Then the event stack of the simulator should contain an event
that indicates when the job's service is supposed to be finished.  If
you think a bit about how to implement such events, taking into
account that all events are related to messages, you will understand
that the server should send itself an \texttt{endService} message at
the moment a job's service starts, to indicate when this job's service
should stop.  This \texttt{endService} message will be put on the
event stack by the event scheduler (not to be confused with the
queueing scheduler!). At some point in time, the event scheduler will
remove this message from the stack, and will give it to
the fifo module.  Once this module receives the \texttt{endService}
message, it knows that the job's service has ended, and that it should
send it to the sink. When the queue is empty after the departure, the
server should just wait. On the other hand, if there are jobs in
queue, it should take the job in front of the queue, and start another
service period. Now have a look at \texttt{fifo.cc} and think deeply
about it until you understand what is exactly going on. Remember well
that a simulation entity has no other means to communicate with itself
in the future than via self-messages, and that these messages have to
be put on the internal event stack.

\texttt{fifo.cc} contains four more functions. The functionality of
\texttt{Fifo::initialize} should be clear from its implementation. The
function \texttt{Fifo::finish} processes part of the simulation
results when the simulation has finished.  The other two functions
\texttt{Fifo::serviceRequirement} and \texttt{endService} enable you
to specify what to do when a job starts service---here only a random
service duration is generated---and when it leaves. This provides
interesting flexibility with respect to queueing simulation. For
instance, in multi-class networks, jobs may change class from service
station to service station, or a fraction may be sent back for another
processing phase at the server, etc. etc. Do not be tempted too easily
to handle this kind of functionality in the \texttt{handleMessage}
function. It can become quite complex.

The last module  of interest, the sink, should be a breeze once
you have mastered the above. It is a good idea to think about why we
compute the waiting time of a job is this module, and why this is not
possible in \texttt{fifo.cc}. This is of course by no means a generic
property of OMNeT++, but just a consequence of the current
implementation of the queueing entity.

\section{Interesting extentions}
\label{sec:inter-extent}
There are a couple of interesting extensions to make to this fifo
example. Let me number up a few.
\begin{itemize}
\item More interarrival and service distributions, such as an efficient
  implementation of the Coxian distribution, multi-server stations, etc.
\item Other scheduling disciplines, such as Processor Sharing, Last Come First
Serve pre-emptive resume, etc.
\item More methods to statistically analyze simulation results
\item Efficient implementations to simulate rare events.
\end{itemize}
In case you decide to build one of these extensions, please tell me so
that I can take it up in the distribution.

\section{Why use OMNeT++ for queueing system simulation?}
\label{sec:why-use-omnet++}
Up to a few month ago, I worked for a huge telecommunication
company. (I only recently started to work for a university. Hence my
time to write this document, and do some good to the global community)  In
one of the projects, we tried to reduce the convergence
time of a distributed network restoration protocol. The interaction
between the protocol, the involved timers, and the states of the
queues in the equipment became soon completely intractable,
analytically speaking.  Hence we decided to analyze the behavior of
the  network and equipment with a  simulator. As the developers wrote the
protocol  in C++ we wanted to use this  code, as it would be
implemented in real  equipment, and carry it over with minor modifications
 to the simulation environment. Besides this we needed a graphical
environment to see whether the protocol messages travelled in the way
we wanted, and ended up in the right queues. OMNeT++ provided this,
and more.

The general experiences I gained during this and other projects with
respect to simulation complex systems were briefly as follows:
\begin{itemize}
\item Using production code in a simulator saves a major amount of
  time compared to having to redesign the work in some kind of
  `high-level', seemingly user friendly simulation environment.
  Besides this, the simulation itself helps to test the real code,
  that is, the code that will actually implemented in commercial
  equipment.
\item Having insight in the code of the simulation environment is
  helpful to understand details of the simulation. Furthermore, it can
  be necessary to extend parts of the tool's functionality. Being
  dependent on companies to give you support, but keeping you securely
  away from the details of the implementation of their simulator, is
  not always what you want. Worse, you have to wait for their answer,
  which not always relates to your question \ldots :-(.
\end{itemize}
Based on the above experiences as well as on some past work with
commercial simulators, I want to point out some advantages, at least
in my opinion, of running queueing simulations within OMNeT++.
\begin{itemize}
\item The behavior of  systems is programmed in a `normal'
  programming language, in this case C++. This provides the user way
  more flexibility than a `high-end', tool-specific programming
  environment. For instance, if you use Bones, your programming skills
  gained during working with Bones only apply to the Bones
  environment. Besides this, these high-end languages appear, as long
  as you stick to the provided tutorials, to be quite generic. But
  when you try your hands on your own examples,  you will feel how
  restrictive these tool-specific languages sometimes  are.
\item As a consequence of the use of C++, OMNeT++ is flexible and
  extensible. As an example, you can easily implement your own job
  interarrival and service distributions, statistical tests on the
  simulation results, more general networks, scheduling distributions,
  etc.
\item It is under GPL license.
\item OMNeT++ enforces `separation of concerns'. You have to specify
  the separate functional units, such as queues, schedulers, job
  generators, separately from routing functionality, that is the
  definition of the network topology.
\end{itemize}
Let me say one more thing about the first bullet. Of course, these
commercial tools allow you to specify parts of the system in your own
code, and connect this code with the simulation platform by means of
software hooks.  But then you have to figure out how this works plus
all debugging, and you still have to program in a real programming
language.

With respect to the last item, I hesitated to use the
possibly vague words `object-oriented', but that is what it really is.
In my experience this is a bonus, as it makes the simulation
environment  modular. You can easily change only one part, or a
module. As long as the interfaces between the modules remain the same,
everything will remain working. Changing the topology is easy
too. Take for instance this demo. If you want to build Jackson
networks---a bunch of M/M/1 queues connected in a network such that
jobs can arrive and leave the network---you only have to figure out
how to set up such networks.  the functionality of job sources, sinks
and service stations is already there, ready to use. If you want to
change some of the service rates, you can easily do this in the
\texttt{omnetpp.ini} file.

Let me stop here, and let you convince yourself about the uses of
OMNeT++.

\bibliography{biblio_nicky}
\bibliographystyle{alpha}


\end{document}

%%% Local Variables:
%%% mode: latex
%%% mode: outline-minor
%%% TeX-master: t
%%% End:


