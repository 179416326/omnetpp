
\section{Signals}
\label{sec:simple-modules:signals}

This section describes \textit{simulation signals}, or signals for short.
Signals are a versatile concept that first appeared in {\opp} 4.1.

Simulation signal can be used for:

\begin{itemize}
  \item{exposing statistical properties of the model, without specifying
        whether and how to record them}
  \item{getting notified about simulation model changes at runtime, and
        acting upon them}
  \item{implementing a publish-subscribe style communication among modules;
        this is advantageous when the producer and consumer of the information
        do not know about each other, and possibly there is many-to-one or
        many-to-many relationship among them}
  \item{emitting information for other purposes, for example as input for
        custom animation effects}


  FIXME TODO: also: pcap traces can be emitted as signals!!!! in general: if the emitted object can be cast to "ISerializable", serialize it into a file??

\end{itemize}

Signals are emitted by components (modules and channels). Signals propagate
up the module hierarchy up to the root. At any level, one can register
listeners (callback objects); these listeners will get notified (called back)
whenever a signal value is emitted. The result of this is that listeners
registered at a compound module can receive all signals from its subdirectory
tree. A listener registered at the system module can receive all signals from
the whole simulation.

\begin{note}
    A channel's parent is the (compound) module that contains the connection,
    not the start or end module whose gate the channel connects.
\begin{note}

Signals are identified by signal names (strings), but for efficiency reasons
at runtime we use dynamically assigned numeric identifiers (signal IDs,
typedef'd as \ttt{simsignal_t}). The mapping of signal names to signal IDs is
global, so all modules and channels asking to resolve a particular signal name
will get back the same numeric signal ID.

Listeners can subscribe to signal names, and not signal name plus source NED type.
That is, if two different and unrelated module types, say \ttt{Queue} and
\ttt{Buffer}, both emit a signal named \ttt{"length"}, then a listener that
subscribes to the \ttt{"length"} signal at some higher compound module will
get notifications from both \ttt{Queue} and \ttt{Buffer} module instances.
The listener can still look at the source of the signal if it wants to
distinguish the two (it is available as a callback parameter), but the signals
framework itself does not have such a feature.

\begin{note}
  Because the component type that emits the signal is not part of the signal's
  identity, it is advised to choose signal names carefully. Good naming
  facilitates "merging" of signals arriving from different sources but
  meaning the same thing, and reduces the chance of collisions (signals that
  accidentally have the same name but represent different things).
\end{note}

When a signal is emitted, it can carry a value with it. This is realized via
overloaded \fname{emit()} methods in components, and overloaded \fname{receiveSignal()}
methods in listeners. The signal value can be of type \ttt{long}, \ttt{double},
\ttt{simtime_t}, string (\ttt{const char *}), and object (\ttt{cObject *}).
Other types can be sent as \ttt{long} or \ttt{double}, or can be wrapped into
a \ttt{cObject} and emitted as such.

\begin{note}
  The number of supported types was intentially kept low, in order to minimize
  the number of overloaded methods that authors of listeners have to provide.
\end{note}


\subsection{The signals machinery}
\label{sec:simple-modules:signals-api}

Signal-related methods are declared on \cclass{cComponent}, so they are available
for both \cclass{cModule} and \cclass{cChannel}.

\subsubsection{Signal IDs}

Signals are identified by names, but internally numeric signal IDs are used
for efficiency. The \fname{registerSignal()} method takes a signal name as
parameter, and returns the corresponding \ttt{simsignal_t} value.
The method is static, illustrating the fact that signal names are global.
An example:

\begin{cpp}
simsignal_t lengthSignalId = registerSignal("length");
\end{cpp}

The \fname{getSignalName()} method (also static) does the reverse:
it accepts a \ttt{simsignal_t}, and returns the name of the signal as
\ttt{const char *} (or \ttt{NULL} for invalid signal handles):

\begin{cpp}
const char *signalName = getSignalName(lengthSignalId); // --> "length"
\end{cpp}

\begin{note}
  There is some significance to the order of \fname{registerSignal()} calls: the
  first 64 signal names registered have somewhat better notification
  performance characteristics than later signals, so it is advised to
  register frequently emitted signals first.
\end{note}

\subsubsection{Emitting signals}

The \fname{emit()} family of functions emit a signal from the module or
channel. They take two parameters, the signal ID (\ttt{simsignal_t}) and
the value:

\begin{cpp}
emit(lengthSignalId, queue.length());
\end{cpp}

The value can be of type \ttt{long}, \ttt{double}, \ttt{simtime_t},
\ttt{const char *}, or \ttt{cObject *}. Other types can be cast into
one of these types, or wrapped into an object subclassed from \ttt{cObject}.

When there are no listeners, the runtime cost of \ttt{emit()} is usually minimal.
However, if producing a value has a significant runtime cost, the
\fname{mayHaveListeners()} or \fname{hasListeners()} method can be used
to check beforehand whether the given signal has any listeners at all --
if not, emitting the signal can be skipped. For some signals (the first 64
in \{opp} 4.1), the information whether it has listeners is cached,
and can be produced in constant time.

The \fname{mayHaveListeners()} method is very efficient (a constant-time
operation) because it only uses this cached information; if the state is
not cached for the signal, it just returns \ttt{true}. In contrast,
\fname{hasListeners()} computes the answer if it is not cached, so it can
be slower. We recommend that you take into account the cost of producing
notification information when deciding between \fname{mayHaveListeners()}
and \fname{hasListeners()}.

Example usage:

\begin{cpp}
if (mayHaveListeners(distanceToTargetSignal))
{
    double d = sqrt((x-targetX)*(x-targetX) + (y-targetY)*(y-targetY));
    emit(distanceToTargetSignal, d);
}
\end{cpp}

\subsubsection{Signal data objects}

When emitting a signal with a \ttt{cObject*} pointer, you can pass as data
an object that you already have in the model, provided you have a suitable
object at hand. However, it is often necessary to declare a custom class
to hold all the details, and fill in an instance just for the purpose of
emitting the signal.

The custom notification class must be derived from \cclass{cObject}.
We recommend that you also add \cclass{noncopyable} as base class, because
then you don't need to write a copy constructor, assignment operator, and
a \fname{dup()} function, sparing some work. When emitting the signal, you
can create a temporary object, and pass its pointer to the \ffunc{emit()}
function.

An example of that is how model change notifications are fired (see
\ref{sec:ch-simple-modules:model-change}). The data class that accompanies
a signal that announces that a gate or gate vector is about to be created
looks like this:

\begin{cpp}
class cPreGateAddNotification : public cObject, noncopyable
{
  public:
    cModule *module;
    const char *gateName;
    cGate::Type gateType;
    bool isVector;
};
\end{cpp}

And the code that emits the signal:

\begin{cpp}
if (hasListeners(PRE_MODEL_CHANGE))
{
    cPreGateAddNotification tmp;
    tmp.module = this;
    tmp.gateName = gatename;
    tmp.gateType = type;
    tmp.isVector = isVector;
    emit(PRE_MODEL_CHANGE, &tmp);
}
\end{cpp}


\subsubsection{Subscribing to signals}

The \fname{subscribe()} method registers a listener for a signal.
Listeners are objects that extend the \cclass{cIListener} class.
The same listener object can be subscribed to multiple signals.
\fname{subscribe()} has two arguments: the signal and a pointer to
the listener object:

\begin{cpp}
cIListener *listener = ...;
simsignal_t lengthSignalId = registerSignal("length");
subscribe(lengthSignalId, listener);
\end{cpp}

For convenience, the \fname{subscribe()} method has a variant
that takes the signal name directly, so the \fname{registerSignal()}
call can be spared:

\begin{cpp}
cIListener *listener = ...;
subscribe("length", listener);
\end{cpp}

One can also subscribe at other modules, not only the local one.
For example, in order to get signals from all parts of the model,
one can subscribe at the system module:

\begin{cpp}
cIListener *listener = ...;
simulation.getSystemModule()->subscribe("length", listener);
\end{cpp}

The \fname{unsubscribe()} method has the same parameter list
as \fname{subscribe()}, and unregisters the given listener
from the signal:

\begin{cpp}
unsubscribe(lengthSignalId, listener);
\end{cpp}

or

\begin{cpp}
unsubscribe("length", listener);
\end{cpp}

\fname{subscribe()} has no effect if the same listener is already
subscribed, and similarly, \fname{unsubscribe()} does nothing
if the listener is not subscribed.

XXX if subscribed, MUST be unsubscribed!!! especially if subscribed
at modules higher in the hierarchy. Module dtors release listeners,
but this is not enough!

It is possible to test whether a listener is subscribed to a signal,
using the \fname{isSubscribed()} method which also takes the same
parameter list.

\begin{cpp}
if (isSubscribed(lengthSignalId, listener))
{
    ...
}
\end{cpp}

For completeness, there are methods for getting the list of signals
that the component has subscribed to (\fname{getLocalListenedSignals()}),
and the list of listeners for a given signal (\fname{getLocalSignalListeners()}).
The former returns \ttt{std::vector<simsignal_t>}, the latter takes
a signal ID (\ttt{simsignal_t}) and returns \ttt{std::vector<cIListener*>}.

The following example prints the number of listeners for each signal:

\begin{cpp}
EV << "Signal listeners:\n";
std::vector<simsignal_t> signals = getLocalListenedSignals();
for (unsigned int i = 0; i < signals.size(); i++) {
    simsignal_t signalID = signals[i];
    std::vector<cIListener*> listeners = getLocalSignalListeners(signalID);
    EV << getSignalName(signalID) << ": " << listeners.size() << " signals\n";
}
\end{cpp}

\subsubsection{Listeners}

Listeners are objects that subclass from the \cclass{cIListener} class:

\begin{cpp}
class cIListener
{
  public:
    virtual ~cIListener() {}
    virtual void receiveSignal(cComponent *source, simsignal_t signalID, long l) = 0;
    virtual void receiveSignal(cComponent *source, simsignal_t signalID, double d) = 0;
    virtual void receiveSignal(cComponent *source, simsignal_t signalID, simtime_t t) = 0;
    virtual void receiveSignal(cComponent *source, simsignal_t signalID, const char *obj) = 0;
    virtual void receiveSignal(cComponent *source, simsignal_t signalID, cObject *obj) = 0;
    virtual void finish(cComponent *component) {}
    virtual void listenerAdded(cComponent *component, simsignal_t signalID) {}
    virtual void listenerRemoved(cComponent *component, simsignal_t signalID) {}
};
\end{cpp}

This class has a number of virtual methods:

\begin{itemize}
  \item Several overloaded \fname{receiveSignal()} methods, one for each
    data type. Whenever a signal is emitted (via \fname{emit()}),
    the matching \fname{receiveSignal()} methods of subscribed listeners
    are invoked.
  \item \fname{finish()}: Method gets called when XXX FIXME TODO WHICH COMPONENT's
    \fname{finish} method gets called. This method can be used to record
    statistics. It does not get called if the simulation has terminated
    with an error. Also, the method may get invoked several times (i.e.
    if the listener is subscribed to multiple signals).
  \item \fname{listenerAdded()}, \fname{listenerRemoved()}. They are called
    when this listener object is subscribed to a signal, and when it is
    unsubscribed.

    FIXME TODO what is this good for.. e.g. reference counting.

    The \fname{listenerRemoved()} method is also called from \cclass{cComponent}'s
    destructor for all listeners, so at this point it is not safe to cast the
    component pointer to any other type. Also, the method may be called several times
    (i.e. if the listener was subscribed to multiple signals).

    It is OK for the listener to delete itself in this method (\ttt{delete this}).
    However, since this method may be called more than once if the listener
    is subscribed multiple times (see above), one must be careful to prevent
    double deletion, e.g. by using reference counting.

\end{itemize}

Since \cclass{cIListener} has too many pure virtual methods, it is more
convenient to subclass from \cclass{cListener}, a do-nothing implementation
instead. It defines \fname{finish()}, \fname{listenerAdded()} and
\fname{listenerRemoved()} with an empty body, and the \fname{receiveSignal()}
methods with a bodies that throw a \ttt{"Data type not supported"} error.
You can redefine the \fname{receiveSignal()} method(s) whose data type
you want to support, and signals emitted with other (unexpected) data
types will result in an error instead of going unnoticed.

XXX cListener does reference counting!

TODO What happens at an emit() call.

The order in which listeners will be notified is undefined (it is not necessarily
the same order in which listeners were subscribed.)


\subsection{Design considerations and rationale}

The design was based on the following assumptions:

1. subscribe/unsubscribe operations are quite rare,
and memory usage and efficiency of notification matter a lot more;

there are relatively few (<10) subscribers per module

2. there should be no memory and minimal CPU overhead for signals emitted but not listened on

emitting signals is virtually free if they are unused (i.e. there are
no listeners). The only memory overhead is the signal handle you need
to keep, and runtime cost is very small as well, because emit()
returns immediately if there are no listeners. (This is true for the
first 64 signals; signals above 64 are slightly less efficient if there
are no listeners, because it has to check the listeners lists at
the modules up to the root).

signals 0..63 are more efficient (emit() has near zero runtime cost if there
are no listeners)

when there are no listeners on the ancestor chain, length of ancestor chain
does not affect notification performance (for signals 0..63)


\subsection{Listening for model changes}
\label{sec:ch-simple-modules:model-change}

In simulation models it is often useful to hold references to other
modules, a connecting channel or other objects, or to cache information
derived from the model topology. However, such pointer s or data may
become invalid when the model changes at runtime, and need to be updated
or recalculated. The problem is how to get notified that something has
changed in the model.

\begin{note}
  Whenever you see a \ttt{cModule*}, \ttt{cChannel*}, \ttt{cGate*} or
  similar pointer kept as state in a simple module, you should think about
  how it will be kept up-to-date if the model changes at runtime.
\end{note}

The solution is, of course, signals. {\opp} has two built-in signals,
\fmac{PRE_MODEL_CHANGE} and \fmac{POST_MODEL_CHANGE} (these macros
are \ttt{simsignal_t} values, not names) that get emitted before and
after each model change.

Pre/post model change notifications are emitted with data objects that
carry the details of the change. The data classes are:

\begin{itemize}
  \item \cclass{cPreModuleAddNotification} / \cclass{cPostModuleAddNotification}
  \item \cclass{cPreModuleDeleteNotification} / \cclass{cPostModuleDeleteNotification}
  \item \cclass{cPreModuleReparentNotification} / \cclass{cPostModuleReparentNotification}
  \item \cclass{cPreGateAddNotification} / \cclass{cPostGateAddNotification}
  \item \cclass{cPreGateDeleteNotification} / \cclass{cPostGateDeleteNotification}
  \item \cclass{cPreGateVectorResizeNotification} / \cclass{cPostGateVectorResizeNotification}
  \item \cclass{cPreGateConnectNotification} / \cclass{cPostGateConnectNotification}
  \item \cclass{cPreGateDisconnectNotification} / \cclass{cPostGateDisconnectNotification}
  \item \cclass{cPrePathCreateNotification} / \cclass{cPostPathCreateNotification}
  \item \cclass{cPrePathCutNotification} / \cclass{cPostPathCutNotification}
  \item \cclass{cPreParameterChangeNotification} / \cclass{cPostParameterChangeNotification}
  \item \cclass{cPreDisplayStringChangeNotification} / \cclass{cPostDisplayStringChangeNotification}
\end{itemize}

They all subclass from \cclass{cModelChangeNotification}, which is of course a
\cclass{cObject}. Inside the listener, you can use \ttt{dynamic_cast<>} to figure
out what notification arrived.

\begin{note}
  Please look up these classes in the API documentation to see their data fields,
  when exactly they get fired, and what one needs to be careful about when using them.
\end{note}

An example listener that prints a message when a module gets deleted:

\begin{cpp}
class MyListener : public cListener
{
   ...
};

void MyListener::receiveSignal(cComponent *source, simsignal_t signalID, cObject *obj)
{
    if (dynamic_cast<cPreModuleDeleteNotification *>(obj))
    {
        cPreModuleDeleteNotification *data = (cPreModuleDeleteNotification *)obj;
        EV << "Module " << data->module->getFullPath() << " is about to be deleted\n";
    }
}
\end{cpp}

If you'd like to get notification about the deletion of any module, you need
to install the listener on the system module:

\begin{cpp}
simulation.getSystemModule()->subscribe(PRE_MODEL_CHANGE, listener);
\end{cpp}

\begin{note}
  \fmac{PRE_MODEL_CHANGE} and \fmac{POST_MODEL_CHANGE} are fired on the
  module (or channel) affected by the change, and \textit{not} on the module
  which executes the code that causes the change. For example,
  \textit{pre-module-deleted} is fired on the module to be removed, and
  \textit{post-module-deleted} is fired on its parent (because the original
  module no longer exists), and not on the module that contains the
  \ffunc{deleteModule()} call.
\end{note}



\subsection{Exposing statistics as signals}

motivation: we don't know in advance what, and with what level of detail we want to collect.

- level of detail: record all values as vector; record the mean/stddev/min/max/avg;
  or record a histogram as well;
- we you may want to process the results before recording them, i.e. record the sum,
  the time average, the count, a smoothed value etc).
- you may want aggregate statistics, e.g. packet drop count for the whole network
- or derived statistics, e.g. drop percentage (drop count/total pks)

These are all possible with signals.

@namespace(bubu);
simple Queue {
    @class(SimpleQueue);
    @signal[queueLength](type=double;interpolation-mode=...);
}

**.queue.queueLength.on-signal = vector(name="queue length", interval=0..500s), scalar(name=..,operation=count), listener(class="XXXListener",....)
**.queue.queueLength*.on-signal = listener(class="SignalJoiner";signals="length,limit,drop";output="")

Queuenet.e.queueLength.on-signal = listener(class="SignalJoiner";signals="length,limit,drop";output="")

**.queue.queueLength*.on-signal = listener:INET::Aggregate(...),
**.queue.queueLength*.on-signal = listener=INET::Aggregate(...),




%%% Local Variables:
%%% mode: latex
%%% TeX-master: "usman"
%%% End:
