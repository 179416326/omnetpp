\chapter{Embedding the Simulation Kernel}
\label{cha:embedding}

\section{Architecture}
\label{sec:embedding:architecture}

{\opp} has a modular architecture. The following diagram illustrates the
high-level architecture of {\opp} simulations:

\begin{figure}[htbp]
  \begin{center}
    \includesvg[scale=0.9]{figures/embed-architecture}
    \caption{The architecture of {\opp} simulations}
  \end{center}
\end{figure}

The blocks represent the following components:

\begin{itemize}
  \item \textbf{Sim} is the simulation kernel and class
    library\index{simulation!kernel}. Sim is a library linked to
    simulation programs.
  \item \textbf{Envir} is another library that contains all code
    that is common to all the user interfaces. \ffunc{main()} also resides
    in the Envir library. Envir presents itself towards Sim and the executing model
    as an instance of the \cclass{cEnvir} facade class. Some aspects of the
    Envir library like result recording can be customized\index{customization}
    using plugin interfaces. Embedding {\opp} into applications\index{embedding}
    usually involves writing a custom \cclass{cEnvir} subclass (see sections
    \ref{sec:plugin-exts:user-interface} and \ref{sec:embedding:embedding}.)
  \item \textbf{Cmdenv, Qtenv} are Envir-based libraries that contain
    specific user interface implementations. A simulation program
    is linked with one or more of them; in the latter case, one of the UI libraries
    is chosen and instantiated either explicitly or automatically when the program starts.
  \item The \textbf{Model Component Library} includes simple module definitions and
    their C++ implementations, compound module types, channels, networks,
    message types, and everything belonging to models that
    have been linked to the simulation program. A simulation program can
    run any model that contains all of the required linked components.
  \item The \textbf{Executing Model} is the model that is set up
    for simulation. This model contains objects (modules, channels, and so on) that
    are all instances of the components in the model component library.
\end{itemize}

The arrows in the figure describe how components interact with
each other:

\begin{itemize}
  \item \textbf{Executing Model $\Leftrightarrow$ Sim}. The simulation kernel
    manages the future events and activates modules in the executing model
    as events occur. The modules of the executing model are stored in an
    instance of the class \cclass{cSimulation}. In turn, the executing model
    calls functions in the simulation kernel and uses classes in the Sim library.
  \item \textbf{Sim $\Leftrightarrow$ Model Component Library}. The simulation kernel
    instantiates simple modules and other components when the simulation model
    is set up at the beginning of the simulation run. In addition, it refers
    to the component library when dynamic module creation is used.
    The mechanisms for registering and looking up components in the model
    component library are implemented as part of Sim.
  \item \textbf{Executing Model $\Leftrightarrow$ Envir}. The Envir presents itself
    as a facade object towards the executing model. Model code directly accesses Envir
    e.g. for logging (\ttt{EV<<}).
  \item \textbf{Sim $\Leftrightarrow$ Envir}. Envir is in full command of what
    happens in the simulation program. Envir contains the \ttt{main()} function
    where execution begins. Envir determines which models should be set up
    for simulation, and instructs Sim to do so. Envir contains the main
    simulation loop (\textit{determine-next-event}, \textit{execute-event}
    sequence) and invokes the simulation kernel for the necessary
    functionality (event scheduling and event execution are implemented in Sim).
    Envir catches and handles errors and exceptions that occur
    in the simulation kernel or in the library classes during execution.
    Envir presents a single facade object toward Sim -- no Envir
    internals are visible to Sim or the executing model.
    During simulation model setup, Envir supplies module parameter values for
    Sim when Sim asks for them. Sim writes output vectors via Envir,
    so one can redefine the output vector storing mechanism by changing Envir.
    Sim and its classes use Envir to print debug information.
  \item \textbf{Envir $\Leftrightarrow$ Cmdenv/Qtenv}. Cmdenv, and Qtenv
    are concrete user interface implementations. When a simulation program
    is started, the \ttt{main()} function (which is part of Envir) determines
    the appropriate user interface class, creates an instance and runs it.
    Sim's or the model's calls on Envir are delegated to the user interface.
\end{itemize}


\section{Embedding the {\opp} Simulation Kernel}
\label{sec:embedding:embedding}

This section discusses the issues of embedding the simulation kernel
or a simulation model into a larger application. We assume that you
do not just want to change one or two aspects of the simulator
(such as , event scheduling or result recording) or create a new user interface
similar to Cmdenv or Qtenv -- if so, see chapter \ref{cha:plugin-exts}.

For the following section, we assume that you will write the embedding
program from scratch, that is, starting from a \ffunc{main()} function.

\subsection{The main() Function}
\label{sec:embedding:main-function}

The minimalistic program described below initializes the simulation library
and runs two simulations. In later sections we will review the details
of the code and discuss how to improve it.

\begin{cpp}
#include <omnetpp.h>
using namespace omnetpp;

int main(int argc, char *argv[])
{
    // the following line MUST be at the top of main()
    cStaticFlag dummy;

    // initializations
    CodeFragments::executeAll(CodeFragments::STARTUP);
    SimTime::setScaleExp(-12);

    // load NED files
    cSimulation::loadNedSourceFolder("./foodir");
    cSimulation::loadNedSourceFolder("./bardir");
    cSimulation::doneLoadingNedFiles();

    // run two simulations
    simulate("FooNetwork", 1000);
    simulate("BarNetwork", 2000);

    // deallocate registration lists, loaded NED files, etc.
    CodeFragment::executeAll(CodeFragment::SHUTDOWN);
    return 0;
}
\end{cpp}

The first few lines of the code initialize the simulation library. The
purpose of \cclass{cStaticFlag} is to set a global variable to \ttt{true}
for the duration of the \ttt{main()} function, to help the simulation
library handle exceptions correctly in extreme cases.
\ttt{CodeFragment::executeAll(CodeFragment::STARTUP)} performs various startup
tasks, such as building registration tables out of the \fmac{Define\_Module()},
\fmac{Register\_Class()} and similar entries throughout the code.
\ttt{SimTime::setScaleExp(-12)} sets the simulation time resolution to
picoseconds; other values can be used as well, but it is mandatory to
choose one.

\begin{note}
The simulation time exponent cannot be changed at a later stage, since it is
a global variable, and the values of the existing \ttt{simtime\_t} instances
would change.
\end{note}

The code then loads the NED files from the \ttt{foodir} and
\ttt{bardir} subdirectories of the working directory (as if the NED path
was \ttt{./foodir;./bardir}), and runs two simulations.


\subsection{The simulate() Function}
\label{sec:embedding:simulate-function}

A minimalistic version of the \ttt{simulate()} function is shown below.
In order to shorten the code, the exception handling code has been ommited (\ttt{try}/\ttt{catch} blocks)
apart from the event loop. However, every line is marked with ``\ttt{E!}'' where various
problems with the simulation model can occur and can be thrown as exceptions.

\begin{cpp}
void simulate(const char *networkName, simtime_t limit)
{
    // look up network type
    cModuleType *networkType = cModuleType::find(networkName);
    if (networkType == nullptr) {
        printf("No such network: %s\n", networkName);
        return;
    }

    // create a simulation manager and an environment for the simulation
    cEnvir *env = new CustomSimulationEnv(argc, argv, new EmptyConfig());
    cSimulation *sim = new cSimulation("simulation", env);
    cSimulation::setActiveSimulation(sim);

    // set up network and prepare for running it
    sim->setupNetwork(networkType); //E!
    sim->setSimulationTimeLimit(limit);

    // prepare for running it
    sim->callInitialize();

    // run the simulation
    bool ok = true;
    try {
        while (true) {
            cEvent *event = sim->takeNextEvent();
            if (!event)
                break;
            sim->executeEvent(event);
        }
    }
    catch (cTerminationException& e) {
        printf("Finished: %s\n", e.what());
    }
    catch (std::exception& e) {
        ok = false;
        printf("ERROR: %s\n", e.what());
    }

    if (ok)
        sim->callFinish();  //E!

    sim->deleteNetwork();  //E!

    cSimulation::setActiveSimulation(nullptr);
    delete sim; // deletes env as well
}
\end{cpp}

The function accepts a network type name (which must be fully qualified
with a package name) and a simulation time limit.

In the first few lines, the code looks up the network among the available
module types, and prints an error message if it is not found.

Then it proceeds to create and activate a simulation manager object
(\cclass{cSimulation}). The simulation manager requires another object,
called the environment object. The environment object is used by the
simulation manager to read the configuration. In addition, the simulation
results are also written via the environment object.

The environment object (\ttt{CustomSimulationEnv} in the above code) must
be provided by the programmer; this is described in detail in a later section.

\begin{note}
In versions 4.x and earlier, the simulation manager and the
environment object could be accessed as \fmac{simulation} and \fmac{ev}
(which were global variables in 3.x and macros in 4.x). In 5.x they can be
accessed with the \ffunc{getSimulation()} and \ffunc{getEnvir()} functions,
which are basically aliases to \ttt{cSimulation::getActiveSimulation()} and
\ttt{cSimulation::getActiveSimulation()->getEnvir()}.
\end{note}

The network is then set up in the simulation manager. The
\ttt{sim->}\ffunc{setupNetwork()} method creates the system module and
recursively all modules and their interconnections; module parameters are
also read from the configuration (where required) and assigned. If there is
an error (for example, module type not found), an exception will be thrown. The
exception object is some kind of \cclass{std::exception}, usually a
\cclass{cRuntimeError}.

If the network setup is successful, \ttt{sim->}\ffunc{callInitialize()} is
invoked next, to run the initialization code of modules and channels in the
network. An exception is thrown if something goes wrong in any of the
\ffunc{initialize()} methods.

The next lines run the simulation by calling
\ttt{sim->}\ffunc{takeNextEvent()} and \ttt{sim->}\ffunc{executeEvent()}
in a loop. The loop is exited when an exception occurs. The exception
may indicate a runtime error, or a normal termination condition such as
when there are no more events, or the simulation time limit has been
reached. (The latter are represented by \cclass{cTerminationException}.)

If the simulation has completed successfully (\ttt{ok==true}), the code
goes on to call the \ffunc{finish()} methods of modules and channels. Then,
regardless whether there was an error, cleanup takes place by calling
\ttt{sim->}\ffunc{deleteNetwork()}.

Finally, the simulation manager object is deallocated, but the active
simulation manager is not allowed to be deleted; therefore it is deactivated
using \ffunc{setActiveSimulation(nullptr)}.


\subsection{Providing an Environment Object}
\label{sec:embedding:providing-an-environment-object}

The environment object needs to be subclassed from the \cclass{cEnvir} class,
but since it has many pure virtual methods, it is easier
to begin by subclassing \cclass{cNullEnvir}. \cclass{cNullEnvir} defines all
pure virtual methods with either an empty body or with a body that throws
an \ttt{"unsupported method called"} exception. You can redefine methods
to be more sophisticated later on, as you progress with the development.

You must redefine the \ffunc{readParameter()} method. This enables
module parameters to obtain their values. For debugging purposes, you can also
redefine \ffunc{sputn()} where module log messages are written to.
\cclass{cNullEnvir} only provides one random number generator, so if your
simulation model uses more than one, you also need to redefine the
\ffunc{getNumRNGs()} and \ffunc{getRNG(k)} methods. To print or store
simulation records, redefine \ffunc{recordScalar()}, \ffunc{recordStatistic()}
and/or the output vector related methods. Other \cclass{cEnvir} methods
are invoked from the simulation kernel to inform the environment about
messages being sent, events scheduled and cancelled, modules created, and so on.

The following example shows a minimalistic environment class that is enough
to get started:

\begin{cpp}
class CustomSimulationEnv : public cNullEnvir
{
  public:
    // constructor
    CustomSimulationEnv(int ac, char **av, cConfiguration *c) :
        cNullEnvir(ac, av, c) {}

    // model parameters: accept defaults
    virtual void readParameter(cPar *par) {
        if (par->containsValue())
            par->acceptDefault();
        else
            throw cRuntimeError("no value for %s", par->getFullPath().c_str());
    }

    // send module log messages to stdout
    virtual void sputn(const char *s, int n) {
        (void) ::fwrite(s,1,n,stdout);
    }
};
\end{cpp}


\subsection{Providing a Configuration Object}
\label{sec:embedding:providing-a-configuration-object}

The configuration object needs to subclass from \cclass{cConfiguration}.
\cclass{cConfiguration} also has several methods, but the typed ones
(\ffunc{getAsBool()}, \ffunc{getAsInt()}, etc.) have default implementations
that delegate to the much fewer string-based methods (\ffunc{getConfigValue()}, etc.).

It is fairly straightforward to implement a configuration class that
emulates an empty ini file:

\begin{cpp}
class EmptyConfig : public cConfiguration
{
  protected:
    class NullKeyValue : public KeyValue {
      public:
        virtual const char *getKey() const {return nullptr;}
        virtual const char *getValue() const {return nullptr;}
        virtual const char *getBaseDirectory() const {return nullptr;}
    };
    NullKeyValue nullKeyValue;

  protected:
    virtual const char *substituteVariables(const char *value) {return value;}

  public:
    virtual const char *getConfigValue(const char *key) const
        {return nullptr;}
    virtual const KeyValue& getConfigEntry(const char *key) const
        {return nullKeyValue;}
    virtual const char *getPerObjectConfigValue(const char *objectFullPath,
        const char *keySuffix) const {return nullptr;}
    virtual const KeyValue& getPerObjectConfigEntry(const char *objectFullPath,
        const char *keySuffix) const {return nullKeyValue;}
};
\end{cpp}


\subsection{Loading NED Files}
\label{sec:embedding:loading-ned-files}

NED files can be loaded with any of the following static methods of
\cclass{cSimulation}: \ffunc{loadNedSourceFolder()}, \ffunc{loadNedFile()},
and \ffunc{loadNedText()}. The first method loads an entire subdirectory tree,
the second method loads a single NED file, and the third method takes a literal
string containing NED code and parses it.

\begin{note}
One use of \ffunc{loadNedText()} is to parse NED sources previously converted
to C++ string constants and linked into the executable. This enables
creating executables that are self-contained, and do not require NED files
to be distributed with them.
\end{note}

The above functions can also be mixed, but after the last call,
\ffunc{doneLoadingNedFiles()} must be invoked (it checks for unresolved
NED types).

Loading NED files has a global effect; therefore they cannot be unloaded.


\subsection{How to Eliminate NED Files}
\label{sec:embedding:eliminating-ned-files}

It is possible to get rid of NED files altogether. This would also
remove the dependency on the \ttt{oppnedxml} library and the code in
\ttt{sim/netbuilder}, although at the cost of additional coding.

\begin{note}
When the only purpose is to get rid of NED files as external dependency
of the program, it is simpler to use \ffunc{loadNedText()} on NED files
converted to C++ string constants instead.
\end{note}

The trick is to write \cclass{cModuleType} and \cclass{cChannelType} objects
for simple module, compound module and channel types, and register them
manually. For example, \cclass{cModuleType} has pure virtual methods called
\ffunc{createModuleObject()}, \ffunc{addParametersAndGatesTo(module)},
\ffunc{setupGateVectors(module)}, \ffunc{buildInside(module)}, which you
need to implement. The body of the \ffunc{buildInside()} method would
be similar to C++ files generated by \fprog{nedtool} of {\opp} 3.x.


\subsection{Assigning Module Parameters}
\label{sec:embedding:assigning-module-parameters}

As already mentioned, modules obtain values for their input parameters
by calling the \ffunc{readParameter()} method of the environment object
(\cclass{cEnvir}).

\begin{note}
\ffunc{readParameter()} is only called for parameters that have not
been set to a fixed (i.e. non-\ttt{default}) value in the NED files.
\end{note}

The \ffunc{readParameter()} method should be written in a manner that enables it to assign
the parameter. When doing so, it can recognize the parameter from its name
(\ttt{par->getName()}), from its full path (\ttt{par->getFullPath()}),
from the owner module's class (\ttt{par->getOwner()->getClassName()})
or NED type name (\ttt{((cComponent *)par->getOwner())->getNedTypeName()}).
Then it can set the parameter using one of the typed setter methods
(\ffunc{setBoolValue()}, \ffunc{setLongValue()}, etc.), or set it
to an expression provided in string form (\ffunc{parse()} method).
It can also accept the default value if it exists (\ffunc{acceptDefault()}).

The following code is a straightforward example that answers parameter
value requests from a pre-filled table.

\begin{cpp}
class CustomSimulationEnv : public cNullEnvir
{
  protected:
    // parameter (fullpath,value) pairs, needs to be pre-filled
    std::map<std::string,std::string> paramValues;
  public:
    ...
    virtual void readParameter(cPar *par) {
        if (paramValues.find(par->getFullPath())!=paramValues.end())
            par->parse(paramValues[par->getFullPath()]);
        else if (par->containsValue())
            par->acceptDefault();
        else
            throw cRuntimeError("no value for %s", par->getFullPath().c_str());
    }
};
\end{cpp}


\subsection{Extracting Statistics from the Model}
\label{sec:embedding:extracting-statistics}

There are several ways you can extract statistics from the
simulation.

\subsubsection{C++ Calls into the Model}
\label{sec:embedding:statistics-via-cpp-calls}

Modules in the simulation are C++ objects. If you add the appropriate
public getter methods to the module classes, you can call them from the
main program to obtain statistics. Modules may be looked up with the
\ffunc{getModuleByPath()} method of \cclass{cSimulation}, then cast to the
specific module type via \ffunc{check\_and\_cast<>()} so that the getter
methods can be invoked.

\begin{cpp}
cModule *mod = getSimulation()->getModuleByPath("Network.client[2].app");
WebApp *appMod = check_and_cast<WebApp *>(mod);
int numRequestsSent = appMod->getNumRequestsSent();
double avgReplyTime = appMod->getAvgReplyTime();
...
\end{cpp}

The drawback of this approach is that getters need to be added manually
to all affected module classes, which might not be practical, especially
if modules come from external projects.

\subsubsection{\cclass{cEnvir} Callbacks}
\label{sec:embedding:statistics-via-cenvir-callbacks}

A more general way is to catch \ffunc{recordScalar()} method calls in the
simulation model. The \cclass{cModule}'s \ffunc{recordScalar()} method
delegates to the similar function in \cclass{cEnvir}. You may define the
latter function so that it stores all recorded scalars (for example in an
\ttt{std::map}), where the main program can find them later.
Values from output vectors can be captured in a similar manner.

An example implementation:

\begin{cpp}
class CustomSimulationEnv : public cNullEnvir
{
  private:
    std::map<std::string, double> results;
  public:
    virtual void recordScalar(cComponent *component, const char *name,
                              double value, opp_string_map *attributes=nullptr)
    {
       results[component->getFullPath()+"."+name] = value;
    }

    const std::map<std::string, double>& getResults() {return results;}
};

...

const std::map<std::string, double>& results = env->getResults();
int numRequestsSent = results["Network.client[2].app.numRequestsSent"];
double avgReplyTime = results["Network.client[2].app.avgReplyTime"];
\end{cpp}

A drawback of this approach is that compile-time checking of statistics names is lost, but
the advantages are that any simulation model can now be used
without changes, and that capturing additional statistics does not require
code modification in the main program.


\subsection{The Simulation Loop}
\label{sec:embedding:simulation-loop}

To run the simulation, the \ffunc{takeNextEvent()} and \ffunc{executeEvent()}
methods of \cclass{cSimulation} must be called in a loop:

\begin{cpp}
cSimulation *sim = getSimulation();
while (sim->getSimTime() < limit) {
    cEvent *event = sim->takeNextEvent();
    sim->executeEvent(event);
}
\end{cpp}

Depending on the concrete scheduler class, the \ffunc{takeNextEvent()}
may return \ttt{nullptr} in certain cases. The default
\cclass{cSequentialScheduler} never returns \ttt{nullptr}.

The execution may terminate in various ways. Runtime errors cause a
\cclass{cRuntimeError} (or other kind of \cclass{std::exception}) to be
thrown. \cclass{cTerminationException} is thrown on normal termination
conditions, such as when the simulation runs out of events to process.

You may customize the loop to exit on other termination conditions as well,
such as on a simulation time limit (see above), on a CPU time limit, or when
results reach a required accuracy. It is relatively straightforward to
build in progress reporting and interactivity (start/stop).

Animation can be hooked up to the appropriate callback methods of
\cclass{cEnvir}: \ffunc{beginSend()}, \ffunc{sendHop()}, \ffunc{endSend()},
and others.


\subsection{Multiple, Coexisting Simulations}
\label{sec:embedding:multiple-coexisting-simulations}

It is possible for several instances of \cclass{cSimulation} to coexist,
and also to set up and simulate a network in each instance. However, this
requires frequent use of \ffunc{cSimulation::set\-Active\-Simulation()}.
Before invoking any \cclass{cSimulation} method or module method,
the corresponding \cclass{cSimulation} instance needs to be designated
as the active simulation manager.

Every \cclass{cSimulation} instance should have its own associated
environment object (\cclass{cEnvir}). Environment objects may not be
shared among several \cclass{cSimulation} instances. The
\cclass{cSimulation}'s destructor also removes the associated
\cclass{cEnvir} instance.

\cclass{cSimulation} instances may be reused from one simulation to another,
but it is also possible to create a new instance for each simulation run.

\begin{note}
It is not possible to run different simulations concurrently from
different theads, due to the use of global variables which are not easy
to eliminate, such as the active simulation manager pointer and the active
environment object pointer. Static buffers and objects (like string pools)
are also used for efficiency reasons in some places inside the simulation
kernel.
\end{note}


\subsection{Installing a Custom Scheduler}
\label{sec:embedding:installing-a-custom-scheduler}

The default event scheduler is \cclass{cSequentialScheduler}. To replace
it with a different scheduler (e.g. \cclass{cRealTimeScheduler} or your
own scheduler class), add a \ffunc{setScheduler()} call into \ttt{main()}:

\begin{cpp}
cScheduler *scheduler = new CustomScheduler();
getSimulation()->setScheduler(scheduler);
\end{cpp}

It is usually not a good idea to change schedulers in the middle of
a simulation, therefore \ffunc{setScheduler()} may only be called when
no network is set up.


\subsection{Multi-Threaded Programs}
\label{sec:embedding:multi-threaded-programs}

The {\opp} simulation kernel is not reentrant; therefore it must be protected
against concurrent access.


%%% Local Variables:
%%% mode: latex
%%% TeX-master: "usman"
%%% End:

