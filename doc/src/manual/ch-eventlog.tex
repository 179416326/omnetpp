\chapter{Eventlog}
\label{cha:eventlog}

\section{Introduction}
The eventlog feature and the related tools are completely new in {\opp} 4.0. They aim to
help in understanding complex simulation models and to help correctly implement the
desired component behaviors. Using these tools you will be able to easily examine every
minute detail of the simulation back and forth in terms of simulation time, or step-by-step,
focusing on the behavior instead of the statistical results of your model.

The eventlog file is created automatically during a simulation run upon explicit request
configurable in the ini file. The resulting file can be viewed in the {\opp} IDE using
the Sequence Chart and the Eventlog Table or can be processed by the command line Eventlog
Tool. These tools support filtering the collected data to allow you to focus on events
relevant to what you are looking for. They allow examining causality relationships and
provide filtering based on simulation times, event numbers, modules and messages.

The simulation kernel records into the eventlog among others: user level messages,
creation and deletion of modules, gates and connections, scheduling of self messages,
sending of messages to other modules either through gates or directly, and processing of
messages (that is events). Optionally, detailed message data can also be automatically
recorded based on a message filter. The result is an eventlog file which contains detailed
information of the simulation run and later can be used for various purposes.

\begin{note}
    The eventlog file may become quite large for long running simulations
    (often hundreds of megabytes, but occasionally several gigabytes), because it
    contains a lot of information about the run, especially when message detail
    recording is turned on.
\end{note}

\section{Configuration}

To record an eventlog file during the simulation, insert the following line into
the ini file:

\begin{inifile}
record-eventlog = true
\end{inifile}

\begin{note}
    Eventlog recording is turned off by default, because creating the eventlog file
    might significantly decrease the overall simulation performance.
\end{note}

\subsection{File Name}

The simulation kernel will write the eventlog file during the simulation into the file
specified by the following ini file configuration entry (showing the default file name
pattern here):

\begin{inifile}
eventlog-file = ${resultdir}/${configname}-${runnumber}.elog
\end{inifile}

\subsection{Recording Intervals}

The size of an eventlog file is approximately proportional to the number of
events it contains. To reduce the file size and speed up the simulation, it
might be useful to record only certain events. The
\ttt{eventlog-recording-intervals} configuration option instructs the
kernel to record events only in the specified intervals. The syntax is
similar to that of \ttt{vector-recording-intervals}.

An example:

\begin{inifile}
eventlog-recording-intervals = ..10.2, 22.2..100, 233.3..
\end{inifile}

\subsection{Recording Modules}

Another factor that affects the size of an eventlog file is the number of
modules for which the simulation kernel records events during the
simulation. The \ttt{module-eventlog-recording} per-module configuration
option instructs the kernel to record only the events that occurred in the
matching modules. The default is to record events from all modules. This
configuration option only applies to simple modules.

The following example records events from any of the routers whose index is
between 10 and 20, and turns off recording for all other modules.

\begin{inifile}
**.router[10..20].**.module-eventlog-recording = true
**.module-eventlog-recording = false
\end{inifile}

\subsection{Recording Message Data}

Since recording message data dramatically increases the size of the
eventlog file and also slows down the simulation, it is turned off by
default, even if writing the eventlog is enabled. To turn on message data
recording, supply a value for the \ttt{eventlog-message-detail-pattern}
option in the ini file.

%TODO explain the syntax a little bit... it is not trivial

%TODO: how to record all details?  =* ? test!

An example configuration for an IEEE 80211 model that records the \ttt{encapsulationMsg} field
and all other fields whose name ends in \ttt{Address}, from messages whose class name ends in
\ttt{Frame} looks like this:

\begin{inifile}
eventlog-message-detail-pattern = *Frame:encapsulatedMsg,*Address
\end{inifile}

An example configuration for a TCP/IP model that records the port and address
fields in all network packets looks like the following:

\begin{inifile}
eventlog-message-detail-pattern =
 PPPFrame:encapsulatedPacket|IPDatagram:encapsulatedPacket,*Address|TCPSegment:*Port
\end{inifile}

% TODO : Assuming you have a message type named \ttt{WirelessFrame}, the following
% option records -- what?
%
% \begin{inifile}
% eventlog-message-detail-pattern = WirelessFrame:declaredOn(WirelessFrame) or bitLength
%\end{inifile}


\section{Eventlog Tool}

The Eventlog Tool is a command line tool to process eventlog files. Invoking it without
parameters will display usage information. The following are the most useful commands for users.

\subsection{Filter}

The eventlog tool provides off line filtering that is usually applied to the eventlog file
after the simulation has been finished and before actually opening it in the {\opp} IDE
or processing it by any other means. Use the filter command and its various options to
specify what should be present in the result file.

\subsection{Echo}

Since the eventlog file format is text based and users are encouraged to implement their
own filters, a way is needed to check whether an eventlog file is
correct. The echo command provides a way to check this and help users creating custom
filters. Anything not echoed back by the eventlog tool will not be taken into
consideration by the other tools found in the {\opp} IDE.

\begin{note}
    Custom filter tools should filter out whole events only, otherwise the
    consequences are undefined.
\end{note}

%%% Local Variables:
%%% mode: latex
%%% TeX-master: "usman"
%%% End:
