\chapter{Analyzing Simulation Results}
\label{cha:analyzing-simulation-results}


\section{Output vectors}
\label{sec:ch-ana-sim:output-vectors}

Output vectors are time series data: values with timestamps.
You can use output vectors to record end-to-end delays or
round trip times of packets, queue lengths, queueing times,
link utilization, the number of dropped packets, etc. --
anything that is useful to get a full picture of what happened
in the model during the simulation run.

Output vectors are recorded from simple modules, by \cclass{cOutVector} objects
(see section \ref{sec:ch-sim-lib:coutvector}). Since output vectors usually
record a large amount of data, in \ttt{omnetpp.ini} you can disable vectors
or specify a simulation time interval for recording
(see section \ref{sec:ch-config-sim:outvectors}).

All \cclass{cOutVector} objects write to the same, common file.
The following sections describe the format of the file, and
how to process it.


\subsection{Plotting output vectors with Plove}

\subsubsection{Plove features}

Typically, you'll get output vector files as a result of a simulation.
Data written to \cclass{cOutVector} objects from
simple modules are written to output vector
files\index{output!vector file}. You can use Plove to look
into the output vector files and plot vectors from them.

Plove\index{Plove} is a handy tool for plotting
{\opp} output vectors. Line type (lines, dots etc) for each vector can be set
as well as the most frequent drawing options like axis bounds, scaling, titles
and labels. You can save the graphs to files (as Encapsulated Postscript or
raster formats such as GIF) with a click. On Windows, you can also
copy the graph to the clipboard in a vector format (Windows metafile)
and paste it into other applications.
    \footnote{Note: prior to {\opp} 3.0, Plove has been a front-end to
    gnuplot. This older version of Plove is no longer supported, but it is still
    available in the {\opp} source distribution.}

Filtering the results\index{filtering results} before plotting is
possible. Filters can do averaging, truncation of extreme values,
smoothing, they can do density estimation\index{density estimation} by
calculating histograms\index{histogram} etc. Some filters are built
in, and you can create new filters by parameterizing and aggregating existing ones.
You can apply several filters to a vector.

On startup, Plove automatically reads the \texttt{.ploverc} file in
your home directory. The file contains general application settings,
including the custom filters you created.




\subsubsection{Usage}

First, you load an output vector file (\texttt{.vec}) into the left
pane. You can copy vectors from the left
pane to the right pane by clicking the button with the right arrow icon
in the middle.
The PLOT button will initiate plotting the \textit{selected}
vectors in the right pane. Selection works as in Windows: dragging
and shift+left click selects a range, and ctrl+left click selects/deselects
individual items. To adjust drawing style, change vector title or add
a filter, click the Options... button. This works for several selected
vectors too. Plove accepts nc/mc-like keystrokes: F3, F4, F5, F6, F8,
grey '+' and grey '*'.


The left pane works as a general storage for vectors you're working
with. You can load several vector files, delete vectors you don't
want to deal with, rename them etc. These changes will not affect
the vector files on disk.
(Plove never modifies the output vector files themselves.)
In the right pane, you can duplicate
vectors if you want to filter the vector and also keep the original.
If you set the right options for a vector but temporarily do
not want it to hang around in the right pane, you can put it
back into the left pane for storage.



\subsection{Format of output vector files}

An output vector file\index{output!vector file} contains several
series of data produced during simulation. The file is textual, and it
looks like this:

\begin{Verbatim}[commandchars=\\\{\}]
\textbf{mysim.vec:}
vector 1   "subnet[4].term[12]"  "response time"  1
1  12.895  2355.66666666
1  14.126  4577.66664666
vector 2   "subnet[4].srvr"  "queue length"  1
2  16.960  2.00000000000.63663666
1  23.086  2355.66666666
2  24.026  8.00000000000.44766536
\end{Verbatim}


There two types of lines: vector declaration lines (beginning with
the word \ttt{vector}), and data lines.
A \textit{vector declaration line} introduces a new output vector, and
its columns are: vector Id, module of creation, name of \cclass{cOutVector}
object, and multiplicity (usually 1). Actual data recorded in this
vector are on \textit{data lines} which begin with the vector Id.
Further columns on data lines are the simulation time and the recorded value.




\subsection{Working without Plove}

In case you have a large number of repeated experiments, you'll probably
want to automate processing of the output vector files. {\opp} lets you
use any tool you see fit for this purpose, because the output
vector files are text files and their format is simple enough to be
processed by common tools such as \textit{perl}, \textit{awk},
\textit{octave}, etc.

\subsubsection{Extracting vectors from the file}

You can use the Unix \fprog{grep} tool to extract a particular vector
from the file. As the first step, you must find out the Id of the
vector. You can find the appropriate vector line with a text editor or
you can use \fprog{grep} for this purpose:

\begin{verbatim}
$ grep "queue length" vector.vec
\end{verbatim}

Or, you can get the list of all vectors in the file by typing:

\begin{verbatim}
$ grep ^vector vector.vec
\end{verbatim}

This will output the appropriate vector line:

\begin{verbatim}
vector 6  "subnet[4].srvr"  "queue length"  1
\end{verbatim}

Pick the vector Id, which is 6 in this case, and grep the file
for the vector's data lines:

\begin{verbatim}
grep ^6 vector.vec > vector6.vec
\end{verbatim}


Now, \texttt{vector6.vec} contains the appropriate vector. The only
potential problem is that the vector Id is there at the beginning of
each line and this may be hard to digest for some programs that you
use for post-processing and/or visualization. This problem is
eliminated by the {\opp} \fprog{splitvec} utility (written in
\fprog{awk}), to be discussed in the next section.




\subsubsection{Using splitvec}

The \fprog{splitvec} script (part of {\opp}) automates the process described
in the previous section: it breaks the vector file into several files which
contain one vector each. The command

\begin{verbatim}
$ splitvec mysim.vec
\end{verbatim}

would create the files \ttt{mysim1.vec}, \ttt{mysim2.vec} etc. with
contents similar to the following:

\begin{Verbatim}[commandchars=\\\{\}]
\textbf{mysim1.vec:}
# vector 1  "subnet[4].term[12]"  "response time"  1
12.895  2355.66666666
14.126  4577.66664666
23.086  2355.66666666

\textbf{mysim2.vec:}
# vector 2  "subnet[4].srvr"  "queue length"  1
16.960  2.00000000000.63663666
24.026  8.00000000000.44766536
\end{Verbatim}


As you can see, the vector Id column has been stripped from the files.
The resulting files can be directly loaded e.g. into spreadsheets or other
programs (\ref{sec:ana-sim:tools}).


\section{Scalar statistics}
\label{sec:ch-ana-sim:output-scalars}

Output vectors capture the transient behaviour of the simulation run.
However, to compare model behaviour under various parameter settings,
output scalars are more useful.

\subsection{Format of output scalar files}

Scalar results are recorded with \fname{record()} calls,
usually from the \ttt{finish()} methods of modules, with code like this:

\begin{verbatim}
void EtherMAC::finish()
{
    double t = simTime();
    if (t==0) return;

    recordScalar("simulated time", t);
    recordScalar("rx channel idle (%)", 100*totalChannelIdleTime/t);
    recordScalar("rx channel utilization (%)", 100*totalSuccessfulRxTxTime/t);
    recordScalar("rx channel collision (%)", 100*totalCollisionTime);

    recordScalar("frames sent",    numFramesSent);
    recordScalar("frames rcvd",    numFramesReceivedOK);
    recordScalar("bytes sent",     numBytesSent);
    recordScalar("bytes rcvd",     numBytesReceivedOK);
    recordScalar("collisions",     numCollisions);

    recordScalar("frames/sec sent", numFramesSent/t);
    recordScalar("frames/sec rcvd", numFramesReceivedOK/t);
    recordScalar("bits/sec sent",   8*numBytesSent/t);
    recordScalar("bits/sec rcvd",   8*numBytesReceivedOK/t);
}
\end{verbatim}

The corresponding output scalar file (by default, \fname{omnetpp.sca})
will look like this:

\begin{verbatim}
run 1 "lan"
scalar "lan.hostA.mac" "simulated time"             120.249243
scalar "lan.hostA.mac" "rx channel idle (%)"        97.5916992
scalar "lan.hostA.mac" "rx channel utilization (%)" 2.40820676
scalar "lan.hostA.mac" "rx channel collision (%)"   0.011312
scalar "lan.hostA.mac" "frames sent"                99
scalar "lan.hostA.mac" "frames rcvd"                3088
scalar "lan.hostA.mac" "bytes sent"                 64869
scalar "lan.hostA.mac" "bytes rcvd"                 3529448
scalar "lan.hostA.mac" "frames/sec sent"            0.823290006
scalar "lan.hostA.mac" "frames/sec rcvd"            25.6799953
scalar "lan.hostA.mac" "bits/sec sent"              4315.63632
scalar "lan.hostA.mac" "bits/sec rcvd"              234808.83
scalar "lan.hostB.mac" "simulated time"             120.249243
scalar "lan.hostB.mac" "rx channel idle (%)"        97.5916992
scalar "lan.hostB.mac" "rx channel utilization (%)" 2.40820676
scalar "lan.hostB.mac" "rx channel collision (%)"   0.011312
[...]
scalar "lan.hostC.mac" "simulated time"             120.249243
scalar "lan.hostC.mac" "rx channel idle (%)"        97.5916992
scalar "lan.hostC.mac" "rx channel utilization (%)" 2.40820676
scalar "lan.hostC.mac" "rx channel collision (%)"   0.011312
[...]

run 2 "lan"
scalar "lan.hostA.mac" "simulated time"             235.678665
[...]
\end{verbatim}

Every \fname{record()} call generates one "scalar" line in the file.
(If you record statistics objects (\cclass{cStatictic} subclasses
such as \cclass{cStdDev}) via their \fname{record()} methods,
they'll generate several lines: mean, standard deviation, etc.)
In addition, several simulation runs can record their results into a single file --
this facilitates comparing them, creating x-y plots
(\textit{offered load vs throughput}-type diagrams), etc.



\subsection{The Scalars tool}

The \fprog{Scalars} program can be used to visualize the contents of the
\fname{omnetpp.sca} file. It can draw bar charts, x-y plots
(e.g. throughput vs offered load), or export data via the clipboard
for more detailed analysis into spreadsheets or other programs.

You can open a scalar file either from the Scalars program's menu
or by specifying it as a command-line argument to Scalars.

The program displays the data in a table with columns showing the
file name, run number, module name where it was recorded, and the value.
There're usually too many rows to get an overview, so you can filter
by choosing from (or editing) the three combo boxes at the top.
(The filters also accept \ttt{*}, \ttt{**} wildcards.)

You could actually load further scalar files into the window,
and thus analyse them together.

You can copy the selected rows to the clipboard by Edit|Copy or
the corresponding toolbar button, and paste them e.g. into
OpenOffice Calc, MS Excel or Gnumeric.

The bar chart toolbar button creates -- well -- a bar chart
in a new window. You can customize the chart by right-clicking
on it and choosing from the context menu. It can also be exported
to EPS, GIF, or as metafile via the Windows clipbard (the latter
is not available on Unix of course).

\section{Scave Tool}

The \fprog{scavetool} is a command line program to process the result
files.

\subsection{Filter command}

[...]

\subsection{Index command}

Rebuilds the index file from the vector file.

If the vector file was generated by an \cclass{cOutputVectorManager},
then specify the -r option to rearrange the rows of the vector file.



\section{Analysis and visualization tools}
\label{sec:ana-sim:tools}

Output vector files (or files produced by \fprog{splitvec}) and
output scalar files can be analysed and/or plotted by
a number of applications in addition to the \fprog{Analysis Editor}.
These programs can produce output in various forms (on the screen,
as PostScript, in various image formats, etc.)

One straightforward solution is to import or paste them into spreadsheet
programs such as OpenOffice Calc, Microsoft Excel or GNOME Gnumeric.
These programs have good charting and statistical features, but the number
of rows is usually limited to about 32,000..64,000.
One useful functionality spreadsheets offer for analysing scalar files is
known as \textit{PivotTable} in Excel, and as \textit{DataPilot} in in OpenOffice.
The easiest way to import scalar files into them is via copy/paste
from Scalars.

Alternatively, one can use numerical packages such as \textit{Octave},
\textit{Matlab} or the statistics package \textit{R}.
In addition to their support for statistical computations, they can also
create various plots.

There are also open-source programs directly for plotting, \textit{Gnuplot}
still being the most commonly used one. Other, potentially more powerful ones
include \textit{MatPlotLib}, \textit{Grace}, \textit{ROOT} and \textit{PlotMTV}.


\subsection{GNU R}

R is a free software environment for statistical computing and graphics.
R has powerful plotting capabilities.

\subsection{MatPlotLib}

MatPlotlib is a plotting library for the Python programming language and
its NumPy numerical mathematics extension. It provides a "pylab" API designed
to closely resemble that of MATLAB, thereby making it easy to learn
for experienced MATLAB users. Matplotlib is written and maintained primarily
by John Hunter, and is distributed under a BSD-style license.

\subsection{Grace}

\textit{Grace} (also known as \textit{xmgrace}, a successor of \textit{ACE/gr} or
\textit{Xmgr}) is a GPL-ed powerful data visualization program
with a WYSIWIG point-and-click graphical user interface. It was developed for
Unix, but there is a Windows version, too.

You load the appropriate file by selecting it in a dialog box.
The icon bar and menu commands can be used to customize the graph.

As of June 2003, Grace 1.5.12 can export graphics to (E)PS, PDF, MIF, SVG,
PNM, JPEG and PNG formats. It has many useful features like built-in statistics
and analysis functions (e.g. correlation, histogram), fitting, splines, etc.,
and it also sports its own built-in programming language.


\subsection{ROOT}

\textit{ROOT} is a powerful object-oriented data analysis framework,
with strong support for plotting and graphics in general.
ROOT was developed at CERN, and is distributed under a BSD-like license.

ROOT is based on \textit{CINT}, a ``C/C++ interpreter''
aimed at processing C/C++ scripts. It is probably harder to get started
using ROOT than with either Gnuplot or Grace, but if you are serious
about analysing simulation results, you will find that ROOT provides
power and flexibility that would be unattainable the other two programs.

Curt Brune's page at Stanford (http://www.slac.stanford.edu/~curt/omnet++/)
shows examples what you can achieve using ROOT with {\opp}.


\subsection{Gnuplot}
\index{Gnuplot}

Gnuplot has an interactive command interface. To plot the data in
\texttt{mysim1.vec} and \texttt{mysim4.vec} (produced by \fprog{splitvec})
plotted in the same graph, you can type:

\begin{verbatim}
plot "mysim1.vec" with lines, "mysim4.vec" with lines
\end{verbatim}

To adjust the $y$ range, you would type:

\begin{verbatim}
set yrange [0:1.2]
replot
\end{verbatim}

Several commands are available to adjust ranges, plotting style, labels,
scaling etc. Gnuplot can also plot 3D graphs. Gnuplot
is available for Windows and other platforms.
On Windows, you can copy the resulting graph to the clipboard from
the Gnuplot window's system menu, then insert it into the application you
are working with.



%%% Local Variables:
%%% mode: latex
%%% TeX-master: "usman"
%%% End:
