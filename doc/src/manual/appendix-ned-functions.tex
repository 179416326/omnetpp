\appendixchapter{NED Functions}
\label{cha:ned-functions}

The functions that can be used in NED expressions and ini files are the
following. The question mark (as in ``\ttt{rng?}'') marks optional arguments.

%
% generated with "opp_run -h nedfunctions" option and tools/processnedfuncs.pl
%

\subsubsection{Category "conversion":}

\begin{description}
\item[double]: \ttt{double double(any x)} \\
    Converts x to double, and returns the result. A boolean argument becomes 0 or 1; a string is interpreted as number; an XML argument causes an error.
\item[int]: \ttt{int int(any x)} \\
    Converts x to an integer (C++ long), and returns the result. A boolean argument becomes 0 or 1; a double is converted using floor(); a string is interpreted as number; an XML argument causes an error.
\item[string]: \ttt{string string(any x)} \\
    Converts x to string, and returns the result.

\end{description}

\subsubsection{Category "math":}

\begin{description}
\item[acos]: \ttt{double acos(double)} \\
    Trigonometric function; see standard C function of the same name
\item[asin]: \ttt{double asin(double)} \\
    Trigonometric function; see standard C function of the same name
\item[atan]: \ttt{double atan(double)} \\
    Trigonometric function; see standard C function of the same name
\item[atan2]: \ttt{double atan2(double, double)} \\
    Trigonometric function; see standard C function of the same name
\item[ceil]: \ttt{double ceil(double)} \\
    Rounds down; see standard C function of the same name
\item[cos]: \ttt{double cos(double)} \\
    Trigonometric function; see standard C function of the same name
\item[exp]: \ttt{double exp(double)} \\
    Exponential; see standard C function of the same name
\item[fabs]: \ttt{quantity fabs(quantity x)} \\
    Returns the absolute value of the quantity.
\item[floor]: \ttt{double floor(double)} \\
    Rounds up; see standard C function of the same name
\item[fmod]: \ttt{quantity fmod(quantity x, quantity y)} \\
    Returns the floating-point remainder of x/y; unit conversion takes place if needed.
\item[hypot]: \ttt{double hypot(double, double)} \\
    Length of the hypotenuse; see standard C function of the same name
\item[log]: \ttt{double log(double)} \\
    Natural logarithm; see standard C function of the same name
\item[log10]: \ttt{double log10(double)} \\
    Base-10 logarithm; see standard C function of the same name
\item[max]: \ttt{quantity max(quantity a, quantity b)} \\
    Returns the greater one of the two quantities; unit conversion takes place if needed.
\item[min]: \ttt{quantity min(quantity a, quantity b)} \\
    Returns the smaller one of the two quantities; unit conversion takes place if needed.
\item[pow]: \ttt{double pow(double, double)} \\
    Power; see standard C function of the same name
\item[sin]: \ttt{double sin(double)} \\
    Trigonometric function; see standard C function of the same name
\item[sqrt]: \ttt{double sqrt(double)} \\
    Square root; see standard C function of the same name
\item[tan]: \ttt{double tan(double)} \\
    Trigonometric function; see standard C function of the same name

\end{description}

\subsubsection{Category "misc":}

\begin{description}
\item[firstAvailable]: \ttt{string firstAvailable(...)} \\
    Accepts any number of strings, interprets them as NED type names (qualified or unqualified), and returns the first one that exists and its C++ implementation class is also available. Throws an error if none of the types are available.
\item[select]: \ttt{any select(int index, ...)} \\
    Returns the <index>th item from the rest of the argument list; numbering starts from 0.
\item[simTime]: \ttt{quantity simTime()} \\
    Returns the current simulation time.
\end{description}

\subsubsection{Category "ned":}

\begin{description}
\item[ancestorIndex]: \ttt{int ancestorIndex(int numLevels)} \\
    Returns the index of the ancestor module numLevels levels above the module or channel in context.
\item[fullName]: \ttt{string fullName()} \\
    Returns the full name of the module or channel in context.
\item[fullPath]: \ttt{string fullPath()} \\
    Returns the full path of the module or channel in context.
\item[parentIndex]: \ttt{int parentIndex()} \\
    Returns the index of the parent module, which has to be part of module vector.

\end{description}

\subsubsection{Category "random/continuous":}

\begin{description}
\item[beta]: \ttt{double beta(double alpha1, double alpha2, int rng?)} \\
    Returns a random number from the Beta distribution
\item[cauchy]: \ttt{quantity cauchy(quantity a, quantity b, int rng?)} \\
    Returns a random number from the Cauchy distribution
\item[chi\_square]: \ttt{double chi\_square(int k, int rng?)} \\
    Returns a random number from the Chi-square distribution
\item[erlang\_k]: \ttt{quantity erlang\_k(int k, quantity mean, int rng?)} \\
    Returns a random number from the Erlang distribution
\item[exponential]: \ttt{quantity exponential(quantity mean, int rng?)} \\
    Returns a random number from the Exponential distribution
\item[gamma\_d]: \ttt{quantity gamma\_d(double alpha, quantity theta, int rng?)} \\
    Returns a random number from the Gamma distribution
\item[lognormal]: \ttt{double lognormal(double m, double w, int rng?)} \\
    Returns a random number from the Lognormal distribution
\item[normal]: \ttt{quantity normal(quantity mean, quantity stddev, int rng?)} \\
    Returns a random number from the Normal distribution
\item[pareto\_shifted]: \ttt{quantity pareto\_shifted(double a, quantity b, quantity c, int rng?)} \\
    Returns a random number from the Pareto-shifted distribution
\item[student\_t]: \ttt{double student\_t(int i, int rng?)} \\
    Returns a random number from the Student-t distribution
\item[triang]: \ttt{quantity triang(quantity a, quantity b, quantity c, int rng?)} \\
    Returns a random number from the Triangular distribution
\item[truncnormal]: \ttt{quantity truncnormal(quantity mean, quantity stddev, int rng?)} \\
    Returns a random number from the truncated Normal distribution
\item[uniform]: \ttt{quantity uniform(quantity a, quantity b, int rng?)} \\
    Returns a random number from the Uniform distribution
\item[weibull]: \ttt{quantity weibull(quantity a, quantity b, int rng?)} \\
    Returns a random number from the Weibull distribution

\end{description}

\subsubsection{Category "random/discrete":}

\begin{description}
\item[bernoulli]: \ttt{int bernoulli(double p, int rng?)} \\
    Returns a random number from the Bernoulli distribution
\item[binomial]: \ttt{int binomial(int n, double p, int rng?)} \\
    Returns a random number from the Binomial distribution
\item[geometric]: \ttt{int geometric(double p, int rng?)} \\
    Returns a random number from the Geometric distribution
\item[intuniform]: \ttt{int intuniform(int a, int b, int rng?)} \\
    Returns a random number from the Intuniform distribution
\item[negbinomial]: \ttt{int negbinomial(int n, double p, int rng?)} \\
    Returns a random number from the Negbinomial distribution
\item[poisson]: \ttt{int poisson(double lambda, int rng?)} \\
    Returns a random number from the Poisson distribution

\end{description}

\subsubsection{Category "strings":}

\begin{description}
\item[choose]: \ttt{string choose(int index, string list)} \\
    Interprets list as a space-separated list, and returns the item at the given index. Negative and out-of-bounds indices cause an error.
\item[contains]: \ttt{bool contains(string s, string substr)} \\
    Returns true if string s contains substr as substring
\item[endsWith]: \ttt{bool endsWith(string s, string substr)} \\
    Returns true if s ends with the substring substr.
\item[expand]: \ttt{string expand(string s)} \\
    Expands \${} variables (\${configname}, \${runnumber}, etc.) in the given string, and returns the result.
\item[indexOf]: \ttt{int indexOf(string s, string substr)} \\
    Returns the position of the first occurrence of substring substr in s, or -1 if s does not contain substr.
\item[length]: \ttt{int length(string s)} \\
    Returns the length of the string
\item[replace]: \ttt{string replace(string s, string substr, string repl, int startPos?)} \\
    Replaces all occurrences of substr in s with the string repl. If startPos is given, search begins from position startPos in s.
\item[replaceFirst]: \ttt{string replaceFirst(string s, string substr, string repl, int startPos?)} \\
    Replaces the first occurrence of substr in s with the string repl. If startPos is given, search begins from position startPos in s.
\item[startsWith]: \ttt{bool startsWith(string s, string substr)} \\
    Returns true if s begins with the substring substr.
\item[substring]: \ttt{string substring(string s, int pos, int len?)} \\
    Return the substring of s starting at the given position, either to the end of the string or maximum len characters
\item[substringAfter]: \ttt{string substringAfter(string s, string substr)} \\
    Returns the substring of s after the first occurrence of substr, or the empty string if s does not contain substr.
\item[substringAfterLast]: \ttt{string substringAfterLast(string s, string substr)} \\
    Returns the substring of s after the last occurrence of substr, or the empty string if s does not contain substr.
\item[substringBefore]: \ttt{string substringBefore(string s, string substr)} \\
    Returns the substring of s before the first occurrence of substr, or the empty string if s does not contain substr.
\item[substringBeforeLast]: \ttt{string substringBeforeLast(string s, string substr)} \\
    Returns the substring of s before the last occurrence of substr, or the empty string if s does not contain substr.
\item[tail]: \ttt{string tail(string s, int len)} \\
    Returns the last len character of s, or the full s if it is shorter than len characters.
\item[toLower]: \ttt{string toLower(string s)} \\
    Converts s to all lowercase, and returns the result.
\item[toUpper]: \ttt{string toUpper(string s)} \\
    Converts s to all uppercase, and returns the result.
\item[trim]: \ttt{string trim(string s)} \\
    Discards whitespace from the start and end of s, and returns the result.

\end{description}

\subsubsection{Category "units":}

\begin{description}
\item[convertUnit]: \ttt{quantity convertUnit(quantity x, string unit)} \\
    Converts x to the given unit.
\item[dropUnit]: \ttt{double dropUnit(quantity x)} \\
    Removes the unit of measurement from quantity x.
\item[replaceUnit]: \ttt{quantity replaceUnit(quantity x, string unit)} \\
    Replaces the unit of x with the given unit.
\item[unitOf]: \ttt{string unitOf(quantity x)} \\
    Returns the unit of the given quantity.

\end{description}

\subsubsection{Category "xml":}

\begin{description}
\item[xml]: \ttt{xml xml(string xmlstring, string xpath?)} \\
    Parses the given XML string into a cXMLElement tree, and returns the root element. When called with two arguments, it returns the first element from the tree that matches the expression given in simplified XPath syntax.
\item[xmldoc]: \ttt{xml xmldoc(string filename, string xpath?)} \\
    Parses the given XML file into a cXMLElement tree, and returns the root element. When called with two arguments, it returns the first element from the tree that matches the expression given in simplified XPath syntax.
\end{description}

%%% Local Variables:
%%% mode: latex
%%% TeX-master: "usman"
%%% End:
