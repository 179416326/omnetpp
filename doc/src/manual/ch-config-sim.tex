\chapter{Configuring Simulations}
\label{cha:config-sim}

\section{The Configuration File}
\label{sec:config-sim:config-file}

Configuration and input data for the simulation are in
a configuration file usually called \ffilename{omnetpp.ini}.

\subsection{An Example}
\label{sec:config-sim:example-inifile}

For a start, let us see a simple \ffilename{omnetpp.ini} file which
can be used to run the Fifo example simulation.

\begin{inifile}
[General]
network = FifoNet
sim-time-limit = 100h
cpu-time-limit = 300s
#debug-on-errors = true
#record-eventlog = true

[Config Fifo1]
description = "low job arrival rate"
**.gen.sendIaTime = exponential(0.2s)
**.gen.msgLength = 100b
**.fifo.bitsPerSec = 1000bps

[Config Fifo2]
description = "high job arrival rate"
**.gen.sendIaTime = exponential(0.01s)
**.gen.msgLength = 10b
**.fifo.bitsPerSec = 1000bps
\end{inifile}

The file is grouped into \textit{sections} named \ttt{[General]}, \ttt{[Config Fifo1]}
and \ttt{[Config Fifo2]}, each one containing several \textit{entries}.


\subsection{File Syntax}
\label{sec:config-sim:inifile-syntax}

An {\opp} configuration file is an ASCII text file, but non-ASCII
characters are permitted in comments and string literals. This allows for
using encodings that are a superset of ASCII, for example ISO 8859-1 and
UTF-8. There is no limit on the file size or on the line length.

Comments may be placed at the end of any line after a hash mark, ``\#''.
Comments extend to the end of the line, and are ignored during processing.
Blank lines are also allowed and ignored.

The file is line oriented, and consists of \textit{section heading lines},
\textit{key-value lines}, and \textit{directive lines}:

\begin{enumerate}
\item \textit{Section heading lines} contain a section name enclosed in square brackets.
\item \textit{Key-value lines} have the \textit{<key>=<value>} syntax; spaces
      are allowed (but not required) on both sides of the equal sign.
      If a line contains more than one equal sign, the leftmost one is taken
      as the key-value separator.
\item Currently there is only one kind of directive line, \textit{include}. An
      include line starts with the \ttt{include} word, followed by the name of
      the file to be included.
\end{enumerate}

Key-value lines may not occur above the first section heading line (except
in included files, see later).

Keys may be further classified based on syntax alone:

\begin{enumerate}
\item Keys that do not contain dots represent global or per-run \textit{configuration options}.
\item If a key contains a dot, its last component (substring after the last dot)
      is considered. If the last component contains a hyphen or is equal to
      \ttt{typename}, the key represents a \textit{per-object configuration option}.
\item Otherwise, the key represents a \textit{parameter assignment}. Thus, parameter
      assignment keys contain a dot, and no hyphen after the last dot.
\end{enumerate}

Long lines can be broken up using the backslash notation: if the last
character of a line is ``\textbackslash'', it will be merged with the next
line.

An example:

\begin{inifile}
# This is a comment line
[General]                       # section heading
network = Foo                   # configuration option
debug-on-errors = false         # another configuration option

**.vector-recording = false     # per-object configuration option
**.app*.typename = "HttpClient" # per-object configuration option

**.app*.interval = 3s           # parameter value
**.app*.requestURL = "http://www.example.com/this-is-a-very-very-very-very\
-very-long-url?q=123456789"     # a two-line parameter value
\end{inifile}

\subsection{File Inclusion}
\label{sec:config-sim:inifile-inclusion}

{\opp} supports including an ini file in another\index{ini file!file inclusion},
via the \ttt{include} keyword. This feature allows one to partition a large ini
file into logical units, fixed and varying part, etc.

An example:

\begin{inifile}
# omnetpp.ini
...
include params1.ini
include params2.ini
include ../common/config.ini
...
\end{inifile}

One can also include files from other directories. If the included ini file
further includes others, their path names will be understood as relative to
the location of the file which contains the reference, rather than relative
to the current working directory of the simulation.

This rule also applies to other file names occurring in ini files (such as
the \fconfig{load-libs}, \fconfig{output-vector-file},
\fconfig{output-scalar-file}, etc. options, and \ttt{xmldoc()} module
parameter values.)

In included files, it is allowed to have key-value lines without first having a
section heading line. File inclusion is conceptually handled as text
substitution, except that a section heading in an included file will not
change the current section the main file. The following example illustrates
the rules:

\begin{inifile}
# incl.ini
foo1 = 1          # no preceding section heading: these lines will go into
foo2 = 2          # whichever section the file is included into
[Config Bar]
bar = 3           # this will always go to into [Config Bar]
\end{inifile}

\begin{inifile}
# omnetpp.ini
[General]
include incl.ini  # adds foo1/foo2 to [General], and defines [Config Bar] w/ bar
baz1 = 4          # include files don't change the current section, so these
baz2 = 4          # lines still belong to [General]
\end{inifile}

\begin{note}
The concept of file inclusion implies that include files may not make sense
on their own. Thus, when an included ini file is opened in the ini editor
in the IDE, file contents may be flagged with errors and warnings. These
errors/warnings disappear when the file is viewed as part of its main file.
\end{note}


\section{Sections}
\label{sec:config-sim:sections}

An ini file may contain a \ttt{[General]} section and several \ttt{[Config <configname>]}
sections. The order of the sections doesn't matter.

\subsection{The [General] Section}
\label{sec:config-sim:general-section}

The most commonly used options of the \texttt{[General]} section are the
following.

\begin{itemize}
  \item{The \fconfig{network} option selects the model to be set up and run.}
  \item{The length of the simulation can be set with the
    \fconfig{sim-time-limit} and the \fconfig{cpu-time-limit} options (the
    usual time units such as ms, s, m, h, etc. can be used).}
\end{itemize}

Note that the NED files loaded by the simulation may contain several
networks, and any of them may be specified in the \fconfig{network}
option.


\subsection{Named Configurations}
\label{sec:config-sim:named-configurations}

Named configurations are sections of the form \ttt{[Config <configname>]}, where
\textit{<configname>} is by convention a camel-case string that starts with a capital letter:
\ttt{Config1}, \ttt{WirelessPing}, \ttt{OverloadedFifo}, etc. For example,
\ffilename{omnetpp.ini} for an Aloha simulation might have the following skeleton:

\begin{inifile}
[General]
...
[Config PureAloha]
...
[Config SlottedAloha1]
...
[Config SlottedAloha2]
...
\end{inifile}

Some configuration options (such as user interface selection) are only
accepted in the \ttt{[General]} section, but most of them can go into \ttt{Config}
sections as well.

When a simulation is run, one needs to select one of the configurations
to be activated. In Cmdenv, this is done with the \ttt{-c} command-line option:

\begin{commandline}
$ aloha -c PureAloha
\end{commandline}

The simulation will then use the contents of the \ttt{[Config PureAloha]}
section to set up the simulation. (Tkenv, of course, lets the user choose
the configuration from a dialog.)


\subsection{Section Inheritance}
\label{sec:config-sim:section-inheritance}

When the PureAloha configuration is activated, the contents of
the \ttt{[General]} section will also be taken into account: if some
configuration option or parameter value is not found in \ttt{[Config PureAloha]},
then the search will continue in the \ttt{[General]} section. In
other words, lookups in \ttt{[Config PureAloha]} will fall back to \ttt{[General]}.
The \ttt{[General]} section itself is optional; when it is absent, it is
treated like an empty \ttt{[General]} section.

All named configurations fall back to \ttt{[General]} by default. However, for
each configuration it is possible to specify the fall-back section or a list of
fallback sections explicitly, using the \fconfig{extends} key. Consider the
following ini file skeleton:

\begin{inifile}
[General]
...
[Config SlottedAlohaBase]
...
[Config LowTrafficSettings]
...
[Config HighTrafficSettings]
...

[Config SlottedAloha1]
extends = SlottedAlohaBase, LowTrafficSettings
...
[Config SlottedAloha2]
extends = SlottedAlohaBase, HighTrafficSettings
...
[Config SlottedAloha2a]
extends = SlottedAloha2
...
[Config SlottedAloha2b]
extends = SlottedAloha2
...
\end{inifile}


When \ttt{SlottedAloha2b} is activated, lookups will consider sections in the
following order (this is also called the \textit{section fallback chain}):
\ttt{SlottedAloha2b}, \ttt{SlottedAloha2}, \ttt{SlottedAlohaBase},
\ttt{HighTrafficSettings}, \ttt{General}.

The effect is the same as if the contents of the sections
SlottedAloha2b, SlottedAloha2, SlottedAlohaBase, HighTrafficSettings and
General were copied together into one section, one after another,
\ttt{[Config SlottedAloha2b]} being at the top, and \ttt{[General]}
at the bottom. Lookups always start at the top, and stop at the first
matching entry.

The order of the sections in the \textit{fallback chain} is computed
using the \textit{C3 linearization algorithm} (\cite{Barrett1996}):

The \textit{fallback chain} of a configuration \ttt{A} is
\begin{itemize}
  \item{if \ttt{A} does not have an \fconfig{extends} key then
        \ttt{A}, \ttt{General}}
  \item{otherwise the merge of the configurations enumerated in
        the \fconfig{extends} key, and all of their
        \textit{fallback section chains}.
        The merge is \textit{monotonic}: if some configuration \ttt{X}
        precedes configuration \ttt{Y} in one of the input chains,
        it will precede it in the output chain too. }
\end{itemize}

The \textit{section fallback chain} can be printed by the \ttt{-X} option
of the command line of the simulation program:

\begin{commandline}
$ aloha -X SlottedAloha2b
`\opp` Discrete Event Simulation
...
Config SlottedAloha2b
Config SlottedAloha2
Config SlottedAlohaBase
Config HighTrafficSettings
General
\end{commandline}

The \textit{section fallback} concept is similar to multiple inheritance in
object-oriented languages, and benefits are similar too; one can factor out
the common parts of several configurations into a ``base'' configuration,
and additionally, one can reuse existing configurations without copying, by
using them as a base. In practice one will often have ``abstract''
configurations too (in the C++/Java sense), which assign only a subset of
parameters and leave the others open, to be assigned in derived
configurations.

When experimenting with a lot of different parameter settings for a
simulation model, file inclusion and section inheritance can make it much
easier to manage ini files.


\section{Assigning Module Parameters}
\label{sec:config-sim:parameter-settings}

Simulations get input via module parameters, which can be assigned a
value in NED files or in \ffilename{omnetpp.ini} -- in this order. Since parameters
assigned in NED files cannot be overridden in \ffilename{omnetpp.ini}, one can
think about them as being ``hardcoded''. In contrast, it is easier
and more flexible to maintain module parameter settings in \ffilename{omnetpp.ini}.

In \ffilename{omnetpp.ini}, module parameters are referred to by their full paths
(hierarchical names). This name consists of the dot-separated list of
the module names (from the top-level module down to the module containing
the parameter), plus the parameter name (see section \ref{sec:sim-lib:fullpath}).

An example \ffilename{omnetpp.ini} which sets the \ttt{numHosts} parameter of
the toplevel module and the \ttt{transactionsPerSecond} parameter of the
\ttt{server} module:

\begin{inifile}
[General]
Network.numHosts = 15
Network.server.transactionsPerSecond = 100
\end{inifile}

Typename pattern assignments are also accepted:

\begin{inifile}
[General]
Network.host[*].app.typename = "PingApp"
\end{inifile}


\subsection{Using Wildcard Patterns}
\label{sec:config-sim:wildcards}

Models can have a large number of parameters to be configured, and it would
be tedious to set them one-by-one in \ffilename{omnetpp.ini}. {\opp} supports
\textit{wildcard patterns} which allow for setting several model parameters
at once. The same pattern syntax is used for per-object configuration options;
for example \ttt{<object-path-pattern>.record-scalar}, or \ttt{<module-path-pattern>.rng-<N>}.

The pattern syntax is a variation on Unix \textit{glob}-style
patterns. The most apparent differences to globbing rules are the
distinction between \ttt{*} and \ttt{**}, and that character ranges should
be written with curly braces instead of square brackets; that is,
\textit{any-letter} is expressed as \ttt{\{a-zA-Z\}} and not as
\ttt{[a-zA-Z]}, because square brackets are reserved for the notation of
module vector indices.

Pattern syntax:

\begin{itemize}
  \item \ttt{?} : matches any character except dot (.)
  \item \ttt{*} : matches zero or more characters except dot (.)
  \item \ttt{**} : matches zero or more characters (any character)
  \item \ttt{\{a-f\}} : \textit{set}: matches a character in the range a-f
  \item \ttt{\{{\textasciicircum}a-f\}}: \textit{negated set}: matches a character
    NOT in the range a-f
  \item \ttt{\{38..150\}} : \textit{numeric range}: any number (i.e. sequence of digits)
    in the range 38..150, inclusive; both limits are optional
  \item \ttt{[38..150]} : \textit{index range}: any number in square brackets in the
    range 38..150, inclusive; both limits are optional
  \item backslash ({\textbackslash}) : takes away the special meaning of the
    subsequent character
\end{itemize}

\subsubsection{Precedence}
\label{sec:config-sim:precedence-of-entries}

The order of entries is very important with wildcards. When a key matches
several wildcard patterns, the \textit{first} matching occurrence
is used. This means that one needs to list specific settings first, and
more general ones later. Catch-all settings should come last.

An example ini file:

\begin{inifile}
[General]
*.host[0].waitTime = 5ms   # specifics come first
*.host[3].waitTime = 6ms
*.host[*].waitTime = 10ms  # catch-all comes last
\end{inifile}


\subsubsection{Asterisk vs Double Asterisk}
\label{sec:config-sim:asterisk-vs-double-asterisk}

The \ttt{*} wildcard is for matching a single module or parameter name in the
path name, while \ttt{**} can be used to match several components in the path.
For example, \ttt{**.queue*.bufSize} matches the \ttt{bufSize} parameter of any module
whose name begins with \ttt{queue} in the model, while \ttt{*.queue*.bufSize}
or \ttt{net.queue*.bufSize} selects only queues immediately on network level.
Also note that \ttt{**.queue**.bufSize} would match \ttt{net.queue1.foo.bar.bufSize}
as well!

\subsubsection{Sets, Negated Sets}
\label{sec:config-sim:sets-negated-sets}

Sets and negated sets can contain several character ranges and also
enumeration of characters. For example, \ttt{\{\_a-zA-Z0-9\}} matches any letter
or digit, plus the underscore; \ttt{\{xyzc-f\}} matches any of the characters
x, y, z, c, d, e, f.
To include '-' in the set, put it at a position where it cannot be
interpreted as character range, for example: \ttt{\{a-z-\}} or \ttt{\{-a-z\}}.
To include '\}' in the set, it must be the first character: \ttt{\{\}a-z\}},
or as a negated set: \ttt{\{{\textasciicircum}\}a-z\}}.
A backslash is always taken as a literal backslash (and not as an escape
character) within set definitions.


\subsubsection{Numeric Ranges and Index Ranges}
\label{sec:config-sim:numeric-ranges-and-index-ranges}

Only nonnegative integers can be matched.  The start or the end of the range
(or both) can be omitted: \ttt{\{10..\}}, \ttt{\{..99\}} or \ttt{\{..\}}
are valid numeric ranges (the last one matches any number).
The specification must use exactly two dots.
Caveat: \ttt{*\{17..19\}} will match \ttt{a17}, \ttt{117} and \ttt{963217} as well,
because the \ttt{*} can also match digits!

An example for numeric ranges:

\begin{inifile}
[General]
*.*.queue[3..5].bufSize = 10
*.*.queue[12..].bufSize = 18
*.*.queue[*].bufSize = 6  # this will only affect queues 0,1,2 and 6..11
\end{inifile}


\subsection{Using the Default Values}
\label{sec:config-sim:using-default-values}

It is also possible to utilize the default values specified in the NED files.
The \textit{<parameter-fullpath>}\ttt{=default} setting assigns
the default value to a parameter if it has one.

The \textit{<parameter-fullpath>}\ttt{=ask} setting will try to get the parameter
value interactively from the user.

If a parameter was not set but has a default value, that value will be
assigned. This is like having a \ttt{**=default} line at the
bottom of the \ttt{[General]} section.

If a parameter was not set and has no default value, that will either
cause an error or will be interactively prompted for, depending
on the particular user interface.

\begin{note}
In Cmdenv, one must explicitly enable interactive mode with the
\ttt{-{}-cmdenv-interactive=true} option, otherwise the simulation
program will stop with an error in the setup phase.
\end{note}

More precisely, parameter resolution takes place as follows:

\begin{enumerate}
\item If the parameter is assigned in NED, it cannot be overridden in the
    configuration. The value is applied and the process finishes.
\item If the first match is a value line (matches
    \textit{<parameter-fullpath>}\ttt{=}\textit{<value>}), the value is
    applied and the process finishes.
\item If the first match is a \textit{<parameter-fullpath>}\ttt{=default} line,
    the default value is applied and the process finishes.
\item If the first match is a \textit{<parameter-fullpath>}\ttt{=ask} line,
    the parameter will be asked from the user interactively (UI dependent).
\item If there was no match and the parameter has a default value, it is applied
    and the process finishes.
\item Otherwise the parameter is declared unassigned, and handled accordingly
    by the user interface. It may be reported as an error, or may be
    asked from the user interactively.
\end{enumerate}


\section{Parameter Studies}
\label{sec:config-sim:parameter-studies}

It is quite common in simulation studies that the simulation model is
run several times with different parameter settings, and the results
are analyzed in relation to the input parameters. {\opp} 3.x had no
direct support for batch runs, and users had to resort to writing shell
(or Python, Ruby, etc.) scripts that iterated over the required
parameter space, to generate a (partial) ini file and run the
simulation program in each iteration.

{\opp} 4.x largely automates this process, and eliminates the need for
writing batch execution scripts. It is the ini file where the user can
specify iterations over various parameter settings.
Here is an example:

\begin{inifile}
[Config AlohaStudy]
*.numHosts = ${1, 2, 5, 10..50 step 10}
**.host[*].generationInterval = exponential(${0.2, 0.4, 0.6}s)
\end{inifile}

This parameter study expands to 8*3 = 24 simulation runs, where the number of
hosts iterates over the numbers 1, 2, 5, 10, 20, 30, 40, 50, and for
each host count three simulation runs will be done, with the generation
interval being exponential(0.2), exponential(0.4), and
exponential(0.6).

How does it work? First of all, Cmdenv with the \ttt{-x} option will print
how many simulation runs a given section expands to. (One should of course
use Cmdenv for batch runs, not a GUI one like Tkenv or Qtenv.)

\begin{commandline}
$ aloha -u Cmdenv -x AlohaStudy

`\opp` Discrete Event Simulation
...
Config: AlohaStudy
Number of runs: 24
\end{commandline}

When the \ttt{-g} option is also added, the program also prints the values
of the iteration variables for each run. (Use \ttt{-G} for even more info.)
Note that the parameter study actually maps to nested loops, with the last
\ttt{\$\{...\}} becoming the innermost loop. The iteration variables are
just named \ttt{\$0} and \ttt{\$1} -- we'll see that it is possible to give
meaningful names to them. Please ignore the \ttt{\$repetition=0} part in
the printout for now.

\begin{commandline}
$ aloha -u Cmdenv -x AlohaStudy -g
`\opp` Discrete Event Simulation
...
Config: AlohaStudy
Number of runs: 24
Run 0: $0=1, $1=0.2, $repetition=0
Run 1: $0=1, $1=0.4, $repetition=0
Run 2: $0=1, $1=0.6, $repetition=0
Run 3: $0=2, $1=0.2, $repetition=0
Run 4: $0=2, $1=0.4, $repetition=0
Run 5: $0=2, $1=0.6, $repetition=0
Run 6: $0=5, $1=0.2, $repetition=0
Run 7: $0=5, $1=0.4, $repetition=0
...
Run 19: $0=40, $1=0.4, $repetition=0
Run 20: $0=40, $1=0.6, $repetition=0
Run 21: $0=50, $1=0.2, $repetition=0
Run 22: $0=50, $1=0.4, $repetition=0
Run 23: $0=50, $1=0.6, $repetition=0
\end{commandline}

Any of these runs can be executed by passing the \ttt{-r <runnumber>}
option to Cmdenv. So, the task is now to run the simulation program 24
times, with \ttt{-r} running from 0 through 23:

\begin{commandline}
$ aloha -u Cmdenv -c AlohaStudy -r 0
$ aloha -u Cmdenv -c AlohaStudy -r 1
$ aloha -u Cmdenv -c AlohaStudy -r 2
...
$ aloha -u Cmdenv -c AlohaStudy -r 23
\end{commandline}

This batch can be executed either from the {\opp} IDE (where you are
prompted to pick an executable and an ini file, choose the configuration
from a list, and just click Run), or using a little command-line
batch execution tool (\fprog{opp\_runall}) supplied with {\opp}.

Actually, it is also possible to get Cmdenv execute all runs in one go,
by simply omitting the \ttt{-r} option.

\begin{commandline}
$ aloha -u Cmdenv -c AlohaStudy

`\opp` Discrete Event Simulation
Preparing for running configuration AlohaStudy, run #0...
...
Preparing for running configuration AlohaStudy, run #1...
...
...
Preparing for running configuration AlohaStudy, run #23...
\end{commandline}


However, this approach is not recommended, because it is more
susceptible to C++ programming errors in the model. (For example, if
any of the runs crashes, the whole batch stops -- which may
not be what the user wants.)


\subsection{Iterations}
\label{sec:config-sim:iterations}

Let us return to the example ini file in the previous section:

\begin{inifile}
[Config AlohaStudy]
*.numHosts = ${1, 2, 5, 10..50 step 10}
**.host[*].generationInterval = exponential( ${0.2, 0.4, 0.6}s )
\end{inifile}

The \ttt{\$\{...\}} syntax specifies an iteration. It is sort of a macro: at
each run, the whole \ttt{\$\{...\}} string is textually replaced with the
current iteration value. The values to iterate over do not need to be
numbers (although the \textit{"a..b"} and \textit{"a..b step c"} forms
only work on numbers), and the substitution takes place even inside
string constants. So, the following examples are all valid (note that
textual substitution is used):

\begin{Verbatim}[commandchars=\\\{\}]
*.param = 1 + \$\{1e-6, 1/3, sin(0.5)\}
    ==> *.param = 1 + 1e-6
        *.param = 1 + 1/3
        *.param = 1 + sin(0.5)
*.greeting = "We will simulate \$\{1,2,5\} host(s)."
    ==> *.greeting = "We will simulate 1 host(s)."
        *.greeting = "We will simulate 2 host(s)."
        *.greeting = "We will simulate 5 host(s)."
\end{Verbatim}

To write a literal \ttt{\$\{..\}} inside a string constant, quote
the left brace with a backslash: \ttt{\${\textbackslash}\{..\}}.

\begin{note}
Inside \ttt{\$\{..\}}, the values are separated with commas. However,
not every comma is taken as a value separator because the parser
tries to be smart about what is meant. Commas inside (nested) parentheses,
brackets or curly braces are ignored so that \ttt{\$\{uniform(0,3)\}} is
parsed as one value and not as \ttt{uniform(0} plus \ttt{3)}. Commas, curly
braces and other charachers inside double-quoted string literals are
also ignored, so \ttt{\$\{"Hello, world"\}} yields a single \ttt{"Hello, world"}
string and not \ttt{"Hello} plus \ttt{world"}. It is assumed that string literals
use backslash as an escape characher, like in C/C++ and NED.

To include a literal comma or close-brace inside a value, one needs
to escape it with a backslash: \ttt{\$\{foo{\textbackslash},bar{\textbackslash}\}baz\}}
will parse as a single value, \ttt{foo,bar\}baz}. Backslashes themselves must be doubled.
As the above examples illustrate, the parser removes one level of backslashes,
except inside string literals where they are left intact.
\end{note}


\subsection{Named Iteration Variables}
\label{sec:config-sim:named-iteration-variables}

One can assign names to iteration variables, which has the advantage
that meaningful names will be displayed in the Cmdenv output instead
of \ttt{\$0} and \ttt{\$1}, and also lets one reference iteration
variables at other places in the ini file. The syntax is
\ttt{\$\{<varname>=<iteration>\}}, and variables can be referred to simply as
\ttt{\$\{<varname>\}}:

\begin{inifile}
[Config Aloha]
*.numHosts = ${N=1, 2, 5, 10..50 step 10}
**.host[*].generationInterval = exponential( ${mean=0.2, 0.4, 0.6}s )
**.greeting = "There are ${N} hosts"
\end{inifile}

The scope of the variable name is the section that defines it, plus
sections based on that section (via \fconfig{extends}).

\subsubsection{Referencing Other Iteration Variables}
\label{sec:config-sim:referencing-other-iteration-variables}

Iterations may refer to other iteration variables, using the dollar syntax
(\ttt{\$var}) or the dollar-brace syntax (\ttt{\$\{var\}}).

This feature makes it possible to have loops where the inner iteration range
depends on the outer one. An example:

\begin{inifile}
**.foo = ${i=1..10}  # outer loop
**.bar = ${j=1..$i}  # inner loop depends on $i
\end{inifile}

When needed, the default top-down nesting order of iteration loops is
modified (loops are reordered) to ensure that expressions only refer to
more outer loop variables, but not to inner ones. When this is not
possible, an error is generated with the ``circular dependency'' message.

For instance, in the following example the loops will be nested in
\textit{k - i - j} order, \textit{k} being the outermost and \textit{j}
the innermost loop:

\begin{inifile}
**.foo = ${i=0..$k}   # must be inner to $k
**.bar = ${j=$i..$k}  # must be inner to both $i and $k
**.baz = ${k=1..10}   # may be the outermost loop
\end{inifile}

And the next example will stop with an error because there is no ``good''
ordering:

\begin{inifile}
**.foo = ${i=0..$j}
**.bar = ${j=0..$k}
**.baz = ${k=0..$i} # --> error: circular references
\end{inifile}

Variables are substituted \textit{textually}, and the result is normally
\textit{not} evaluated as an arithmetic expression.  The result of the
substitution is only evaluated where needed, namely in the three arguments
of iteration ranges (\textit{from}, \textit{to}, \textit{step}), and in the
value of the \fconfig{constraint} configuration option.

To illustrate textual substitution, consider the following contorted example:

\begin{inifile}
**.foo = ${i=1..3, 1s+, -}001s
\end{inifile}

Here, the \ttt{foo} NED parameter will receiving the following values in subsequent runs:
\ttt{1001s}, \ttt{2001s}, \ttt{3001s}, \ttt{1s+001s}, \ttt{-001s}.

\begin{caution}
Due to textual substitution, variables in arithmetic expressions should be
protected with parentheses -- just like in C/C++ function-style macros.
Consider the following example:

\begin{inifile}
**.foo = ${i=10}
**.bar = ${j=$i+5}
**.baz = ${k=2*$j}   # bogus! $j should be written as ($j)
constraint = $i+50 < 2*$j  # ditto: should use ($i) and ($j)
\end{inifile}

Here, the \ttt{baz} parameter will receive the string \ttt{2*10+5} after
the substitutions and hence evaluate to $25$ instead of the correct
$2*(10+5)=30$; the constraint expression is similarly wrong. Mind the parens!
\end{caution}

Substitution also works inside string constants within iterations (\ttt{\$\{..\}}).

\begin{inifile}
**.foo = "${i=Jo}"  # -> Jo
**.bar = ${"Hi $i", "Hi ${i}hn"}  # -> Hi Jo /John
\end{inifile}

However, outside iterations the plain dollar syntax is not understood, only
the dollar-brace syntax is:

\begin{inifile}
**.foo = "${i=Day}"
**.baz = "Good $i"     # -> remains "Good $i"
**.baz = "Good ${i}"   # -> becomes "Good Day"
\end{inifile}

\begin{rationale}
The text substitution model was chosen for greater flexibility as well as
the ability to produce more consistent semantics. The advantages outweigh
the inconvenience of having to parenthesize variable references in
arithmetic expressions.
\end{rationale}


\subsection{Parallel Iteration}
\label{sec:config-sim:parallel-iteration}

The body of an iteration may end in an exclamation mark followed by the
name of another iteration variable. This syntax denotes a \textit{parallel
iteration}. A parallel iteration does not define a loop of its own, but
rather, the sequence is advanced in lockstep with the variable after the
``!''. In other words, the ``!'' syntax chooses the \textit{kth} value from
the iteration, where \textit{k} is the position (iteration count) of the
iteration variable after the ``!''.

An example:

\begin{inifile}
**.plan =     ${plan= "A", "B", "C", "D"}
**.numHosts = ${hosts= 10,  20,  50, 100 ! plan}
**.load =     ${load= 0.2, 0.3, 0.3, 0.4 ! plan}
\end{inifile}

In the above example, the only loop is defined by the first line, the \ttt{plan}
variable. The other two iterations, \ttt{hosts} and \ttt{load} just follow it;
for the first value of \ttt{plan} the first values of \ttt{host} and \ttt{load}
are selected, and so on.



\subsection{Predefined Variables, Run ID}
\label{sec:config-sim:predefined-variables}

There are a number of predefined variables: \ttt{\$\{configname\}} and
\ttt{\$\{runnumber\}} with the obvious meanings; \ttt{\$\{network\}} is the
name of the network that is simulated; \ttt{\$\{processid\}} and
\ttt{\$\{datetime\}} expand to the OS process id of the simulation and the
time it was started; and there are some more: \ttt{\$\{runid\},}
\ttt{\$\{iterationvars\}} and \ttt{\$\{repetition\}.}

\ttt{\$\{runid\}} holds the \textit{run ID}. When a simulation is run, a
run ID is assigned that uniquely identifies that instance of running the
simulation: every subsequent run of the same simulation will produce a different
run ID. The run ID is generated as the concatenation of several variables like
\ttt{\$\{configname\},} \ttt{\$\{runnumber\},} \ttt{\$\{datetime\}} and
\ttt{\$\{processid\}.} This yields an identifier that is unique ``enough''
for all practical purposes, yet it is meaningful for humans. The run ID is
recorded into result files written during the simulation, and can be used
to match vectors and scalars written by the same simulation run.


\subsection{Constraint Expression}
\label{sec:config-sim:constraint-expression}

In cases when not all combinations of the iteration variables make sense
or need to be simulated, it is possible to specify an additional
constraint expression. This expression is interpreted as a conditional
(an "if" statement) within the innermost
loop, and it must evaluate to \ttt{true} for
the variable combination to generate a run. The expression should be
given with the \fconfig{constraint} configuration option. An example:

\begin{inifile}
*.numNodes = ${n=10..100 step 10}
**.numNeighbors = ${m=2..10 step 2}
constraint = ($m) <= sqrt($n)  # note: parens needed due to textual substitution
\end{inifile}

The expression syntax supports most C language operators including boolean,
conditional and binary shift operations, and most \ttt{<math.h>} functions;
data types are boolean, double and string. The expression must evaluate to
a boolean.

\begin{note}
Remember that variables are substituted textually into the expresssion, so
they must be protected with parentheses to preserve evaluation order.
\end{note}


\subsection{Repeating Runs with Different Seeds}
\label{sec:config-sim:repeating-runs-with-different-seeds}

It is directly supported to perform several runs with the same
parameters but different random number seeds. There are two
configuration options related to this: \fconfig{repeat} and \fconfig{seed-set}.
The first one simply specifies how many times a run needs to be repeated. For
example,

\begin{inifile}
repeat = 10
\end{inifile}

causes every combination of iteration variables to be repeated 10 times,
and the \ttt{\$\{repetition\}} predefined variable holds the loop counter.
Indeed, \ttt{repeat=10} is equivalent to adding \ttt{\$\{repetition=0..9\}}
to the ini file. The \ttt{\$\{repetition\}} loop always becomes the innermost loop.

The \fconfig{seed-set} configuration key affects seed selection. Every
simulation uses one or more random number generators (as configured by
the \fconfig{num-rngs} key), for which the simulation kernel can automatically
generate seeds. The first simulation run may use one set of seeds (seed
set 0), the second run may use a second set (seed set 1), and so on.
Each set contains as many seeds as there are RNGs configured. All
automatic seeds generate random number sequences that are far apart in
the RNG's cycle, so they will never overlap during
simulations.

\begin{note}
Mersenne Twister, the default RNG of {\opp} has a cycle length of
$2^{19937}$, which is more than enough for any conceivable purpose.
\end{note}

The \fconfig{seed-set} key tells the simulation kernel which seed set to use.
It can be set to a concrete number (such as \ttt{seed-set=0}), but it
usually does not make sense as it would cause every simulation to run
with exactly the same seeds. It is more practical to set it to either
\ttt{\$\{runnumber\}} or to \ttt{\$\{repetition\}}. The default setting is
\ttt{\$\{runnumber\}}:

\begin{inifile}
seed-set = ${runnumber}   # this is the default
\end{inifile}

This causes every simulation run to execute with a unique seed set. The
second option is:

\begin{inifile}
seed-set = ${repetition}
\end{inifile}

where all \ttt{\$repetition=0} runs will use the same seeds (seed set 0), all
\ttt{\$repetition=1} runs use another seed set, \ttt{\$repetition=2} a third seed
set, etc.

To perform runs with manually selected seed sets, one needs to define an
iteration for the \fconfig{seed-set} key:

\begin{inifile}
seed-set = ${5,6,8..11}
\end{inifile}

In this case, the \fconfig{repeat} key should be left out, as \fconfig{seed-set}
already defines an iteration and there is no need for an extra loop.

It is of course also possible to manually specify individual seeds for
simulations. The parallel iteration feature is very convenient here:

\begin{inifile}
repeat = 4
seed-1-mt = ${53542, 45732, 47853, 33434 ! repetition}
seed-2-mt = ${75335, 35463, 24674, 56673 ! repetition}
seed-3-mt = ${34542, 67563, 96433, 23567 ! repetition}
\end{inifile}

The meaning of the above is this: in the first repetition, the first
column of seeds is chosen, for the second repetition, the second
column, etc. The "!" syntax chooses the
\textit{kth} value from the iteration, where \textit{k} is the position
(iteration count) of the iteration variable after the
"!". Thus, the above example is equivalent to the following:

\begin{inifile}
# no repeat= line!
seed-1-mt = ${seed1 = 53542, 45732, 47853, 33434}
seed-2-mt = ${        75335, 35463, 24674, 56673 ! seed1}
seed-3-mt = ${        34542, 67563, 96433, 23567 ! seed1}
\end{inifile}

That is, the iterators of \ttt{seed-2-mt} and \ttt{seed-3-mt} are advanced
in lockstep with the \ttt{seed1} iteration.



\subsection{Experiment-Measurement-Replication}
\label{sec:config-sim:experiment-measurement-replication}

We have introduced three concepts that are useful for organizing
simulation results generated by batch executions or several batches of
executions.

During a simulation study, a user prepares several
\textit{experiments}. The purpose of an experiment is to find out the
answer to questions like \textit{"how does the number of
nodes affect response times in the network?"} For an
experiment, several \textit{measurements} are performed on the
simulation model, and each measurement runs the simulation model with a
different set of parameters. To eliminate the bias introduced by the
particular random number stream used for the simulation, several
\textit{replications} of every measurement are run with different
random number seeds, and the results are averaged.

{\opp} result analysis tools can take advantage of the \textit{experiment},
\textit{measurement} and \textit{replication} labels recorded into result
files, and display simulation runs and recorded results accordingly on
the user interface.

These labels can be explicitly specified in the ini file using the
\fconfig{experiment-label}, \fconfig{measurement-label} and \fconfig{replication-label}
config options. If they are missing, the default is the following:

\begin{inifile}
experiment-label = "${configname}"
measurement-label = "${iterationvars}"
replication-label = "#${repetition},seed-set=<seedset>"
\end{inifile}

That is, the default experiment label is the configuration name; the
measurement label is concatenated from the iteration variables; and the
replication label contains the repeat loop variable and seed-set.
Thus, for our first example the \textit{experiment-measurement-replication}
tree would look like this:

\begin{Verbatim}[commandchars=\\\{\}]
"PureAloha"\textrm{\textit{--experiment}}
  \$N=1,\$mean=0.2\textrm{\textit{ -- measurement}}
    \#0, seed-set=0\textrm{\textit{ -- replication}}
    \#1, seed-set=1
    \#2, seed-set=2
    \#3, seed-set=3
    \#4, seed-set=4
  \$N=1,\$mean=0.4
    \#0, seed-set=5
    \#1, seed-set=6
    ...
    \#4, seed-set=9
  \$N=1,\$mean=0.6
    \#0, seed-set=10
    \#1, seed-set=11
    ...
    \#4, seed-set=14
  \$N=2,\$mean=0.2
    ...
  \$N=2,\$mean=0.4
    ...
    ...
\end{Verbatim}

The \textit{experiment-measurement-replication} labels should be enough to
reproduce the same simulation results, given of course that the ini
files and the model (NED files and C++ code) haven't changed.

Every instance of running the simulation gets a unique run ID. We can
illustrate this by listing the corresponding run IDs under each
repetition in the tree. For example:

\begin{Verbatim}[commandchars=\\\{\}]
"PureAloha"
  \$N=1,\$mean=0.2
    \#0, seed-set=0
      \textit{PureAloha-0-20070704-11:38:21-3241}
      \textit{PureAloha-0-20070704-11:53:47-3884}
      \textit{PureAloha-0-20070704-16:50:44-4612}
    \#1, seed-set=1
      \textit{PureAloha-1-20070704-16:50:55-4613}
    \#2, seed-set=2
      \textit{PureAloha-2-20070704-11:55:23-3892}
      \textit{PureAloha-2-20070704-16:51:17-4615}
      ...
\end{Verbatim}

The tree shows that ("PureAloha", "\$N=1,\$mean=0.2", "\#0, seed-set=0")
was run three times. The results produced
by these three executions should be identical, unless, for example,
some parameter was modified in the ini file, or a bug got fixed in the
C++ code.

The default way of generating the \textit{experiment/measurement/replication}
labels is useful and sufficient for the majority of simulation studies.
However, it can be customized if needed. For example, here is a way to join two
configurations into one experiment:

\begin{inifile}
[Config PureAloha_Part1]
experiment-label = "PureAloha"
...
[Config PureAloha_Part2]
experiment-label = "PureAloha"
...
\end{inifile}

Measurement and replication labels can be customized in a similar way,
making use of named iteration variables, \ttt{\$\{repetition\}},
\ttt{\$\{runnumber\}} and other predefined variables. One possible benefit is
to customize the generated measurement and replication labels. For
example:

\begin{inifile}
[Config PureAloha_Part1]
measurement = "${N} hosts, exponential(${mean}) packet generation interval"
\end{inifile}

One should be careful with the above technique though, because if some
iteration variables are left out of the measurement labels, runs with
all values of those variables will be grouped together to the same
replications.


\section{Configuring the Random Number Generators}
\label{sec:config-sim:rng-config}

The random number architecture of {\opp} was already outlined
in section \ref{sec:sim-lib:random-number-generators}. Here
we'll cover the configuration of RNGs in \ffilename{omnetpp.ini}.

\subsection{Number of RNGs}
\label{sec:config-sim:number-of-rngs}

The \fconfig{num-rngs} configuration option sets the number of
random number generator instances (i.e. random number streams)
available for the simulation model (see \ref{sec:sim-lib:random-number-generators}).
Referencing an RNG number greater or equal to this number
(from a simple module or NED file) will cause a runtime error.


\subsection{RNG Choice}
\label{sec:config-sim:rng-choice}

The \fconfig{rng-class} configuration option sets the random number
generator class to be used. It defaults to \ttt{"cMersenneTwister"},
the Mersenne Twister RNG. Other available classes are \ttt{"cLCG32"}
(the "legacy" RNG of {\opp} 2.3 and earlier versions, with a cycle length
of $2^{31}-2$), and \ttt{"cAkaroaRNG"} (Akaroa's random number generator,
see section \ref{sec:run-sim:akaroa}).

\subsection{RNG Mapping}
\label{sec:config-sim:rng-mapping}

The RNG numbers used in simple modules may be arbitrarily mapped to the
actual random number streams (actual RNG instances) from \ffilename{omnetpp.ini}.
The mapping allows for great flexibility in RNG usage and random number
streams configuration -- even for simulation models which were not
written with RNG awareness.

RNG mapping may be specified in \ffilename{omnetpp.ini}. The syntax of
configuration entries is the following.

\begin{inifile}
[General]
<modulepath>.rng-N = M  # where N,M are numeric, M < num-rngs
\end{inifile}

This maps module-local RNG N to physical RNG M. The following
example maps all \ttt{gen} module's default (N=0) RNG to physical RNG 1,
and all \ttt{noisychannel} module's default (N=0) RNG to physical RNG 2.

\begin{inifile}
[General]
num-rngs = 3
**.gen[*].rng-0 = 1
**.noisychannel[*].rng-0 = 2
\end{inifile}

The value also allows expressions, including those containing \fkeyword{index},
\fkeyword{parentIndex}, and \fkeyword{ancestorIndex(level)}. This allows things
like assigning a separate RNG to each element of a module vector.

This mapping allows variance reduction techniques to be applied to
{\opp} models, without any model change or recompilation.


\subsection{Automatic Seed Selection}
\label{sec:config-sim:automatic-seed-selection}

Automatic seed selection is used for an RNG if one does not explicitly
specify seeds in \ffilename{omnetpp.ini}. Automatic and manual seed selection can
co-exist; for a particular simulation, some RNGs can be configured
manually, and some automatically.

The automatic seed selection mechanism uses two inputs: the \textit{run number}
 and the \textit{RNG number}. For the same run number and RNG number,
{\opp} always selects the same seed value for any simulation model.
If the run number or the RNG number is different, {\opp} does its best
to choose different seeds which are also sufficiently separated in the RNG's sequence
so that the generated sequences don't overlap.

The run number can be specified either in in \ffilename{omnetpp.ini} (e.g. via the
\fconfig{cmdenv-runs-to-execute} option) or on the command line:

\begin{commandline}
./mysim -r 1
./mysim -r 2
./mysim -r 3
\end{commandline}

For the \ttt{cMersenneTwister} random number generator, selecting seeds
so that the generated sequences don't overlap is easy,
due to the extremely long sequence of the RNG.
The RNG is initialized from the 32-bit seed value $seed = runNumber*numRngs + rngNumber$.
(This implies that simulation runs participating in the study should have
the same number of RNGs set).
    \footnote{While (to our knowledge) no one has proven that the seeds 0,1,2,...
    are well apart in the sequence, this is probably true, due to the extremely
    long sequence of MT. The author would however be interested in papers
    published about seed selection for MT.}

For the \ttt{cLCG32} random number generator, the situation is more difficult,
because the range of this RNG is rather short ($2^{31}-1$, about 2 billion).
For this RNG, {\opp} uses a table of 256 pre-generated seeds, equally spaced
in the RNG's sequence. Index into the table is calculated with the
$runNumber*numRngs + rngNumber$ formula. Care should be taken that
one doesn't exceed 256 with the index, or it will wrap and the
same seeds will be used again. It is best not to use the \ttt{cLCG32}
at all -- \ttt{cMersenneTwister} is superior in every respect.


\subsection{Manual Seed Configuration}
\label{sec:config-sim:manual-seed-configuration}

In some cases, one may want to manually configure seed values. The
motivation for doing so may be the use of variance reduction techniques, or
the intention to reuse the same seeds for several simulation runs.

To manually set seeds for the Mersenne Twister RNG, use the \ttt{seed-}\textit{k}\ttt{-mt}
option, where \textit{k} is the RNG index. An example:

\begin{inifile}
[General]
num-rngs = 3
seed-0-mt = 12
seed-1-mt = 9
seed-2-mt = 7
\end{inifile}

\label{sec:config-sim:seedtool}

For the now obsolete cLCG32 RNG, the name of the corresponding option is
\ttt{seed-}\textit{k}\ttt{-lcg32}.

\section{Logging}
\label{sec:config-sim:logging}

The {\opp} logging infrastructure provides a few configuration options that
affect what is written to the log output. It supports configuring multiple
filters: global compile-time, global runtime, and per-component runtime log
level filters. For a log statement to actually produce output, it must pass
each filter simulatenously. In addition, one can also specify a log prefix
format string which determines the context information that is written before
each log line. In the following sections, we look how to configure logging.

\subsection{Compile-Time Filtering}
\label{sec:config-sim:compile-time-log-filtering}

The \fmac{COMPILETIME\_LOGLEVEL} macro determines which log statements
are compiled into the executable. Any log statment which uses a log level below
the specified compile-time log level is omitted. In other words, no matter how
the runtime log levels are configured, such log statements are not even executed.
This is mainly useful to avoid the performance penalty paid for log statements
which are not needed.

\begin{cpp}
#define COMPILETIME_LOGLEVEL LOGLEVEL_INFO
EV_INFO << "Packet received successfully" << endl;
EV_DEBUG << "CRC check successful" << endl;
\end{cpp}

In the above example, the output of the second log statement is omitted:

\begin{filelisting}
[INFO] Packet received successfully
\end{filelisting}

If simulation performance is critical, and if there are lots of log statements
in the code, it might be useful to omit all log statements from the executable.
This can be very simply achieved by putting the following macro into effect for
the compilation of all source files.

\begin{cpp}
#define COMPILETIME_LOGLEVEL LOGLEVEL_OFF
\end{cpp}

On the other hand, if there's some hard to track down issue, it might be useful
to just do the opposite. Compiling with the lowest log level ensures that the
log output contains as much information as possible.

\begin{cpp}
#define COMPILETIME_LOGLEVEL LOGLEVEL_TRACE
\end{cpp}

By default, the \fmac{COMPILETIME\_LOGLEVEL} macro is set to
\fmac{LOGLEVEL\_TRACE} if the code is compiled in debug mode (\fmac{NDEBUG} is
not set). However, it is set to \fmac{LOGLEVEL\_DETAIL} if the code is compiled
in release mode (\fmac{NDEBUG} is set).

In fact, the \fmac{COMPILETIME\_LOG\_PREDICATE} macro is the most generic compile
time predicate that determines which log statements are compiled into the executable.
Mostly, there's no need to redefine this macro, but it can be useful sometimes.
For example, one can do compile-time filtering for log categories by redefining
this macro. By default, the \fmac{COMPILETIME\_LOG\_PREDICATE} macro is defined
as follows:

\begin{cpp}
#define COMPILETIME_LOG_PREDICATE(object, loglevel, category) \
        (loglevel >= COMPILETIME_LOGLEVEL)
\end{cpp}

\subsection{Runtime Filtering}
\label{sec:config-sim:runtime-log-filtering}

The \fvar{cLog::logLevel} variable restricts during runtime which log statements
produce output. By default, the global runtime log level doesn't filter logging,
it is set to \fmac{LOGLEVEL\_TRACE}. Although due to its global nature it's not
really modular, nevertheless it's still allowed to change the value of this
variable. It is mainly used in interactive user interfaces to implement efficient
global filtering, but it may also be useful for various debugging purposes.

In addition to the global variable, there's also a per-component runtime log
level which only restricts the output of a particular component of the
simulation. By default, the runtime log level of all components are set to
\fmac{LOGLEVEL\_TRACE}. Programmatically, these log levels can be retrieved
with \ffunc{cComponent::getLogLevel()} and changed with
\ffunc{cComponent::setLogLevel()}.

In general, any log statment which uses a log level below the specified global
runtime log level, or below the specified per-component runtime log level, is
omitted. If the log statement appears in a module source, then the module's
per-component runtime log level is checked. In any other C++ code, the context
module's per-component runtime log level is checked.

In fact, the \fvar{cLog::noncomponentLogPredicate} and the \fvar{cLog::componentLogPredicate}
are the most generic runtime predicates that determines which log statements are
executed. Mostly, there's no need to redefine these predicates, but it can be
useful sometimes. For example, one can do runtime filtering for log categories
by redefining them. To cite a real example, the \fvar{cLog::componentLogPredicate} function
contains the following runtime checks:

\begin{cpp}
return statementLogLevel >= cLog::loglevel &&
       statementLogLevel >= sourceComponent->getLogLevel() &&
       getEnvir()->isLoggingEnabled(); // for express mode
\end{cpp}

\subsection{Log Prefix Format}
\label{sec:config-sim:log-prefix-format}

The log prefix format is a string which determines the log prefix that is written
before each log line. The format string contains constant parts interleaved with
special format directives. The latter always start with the \ttt{\%} character
followed by another character that identifies the format directive. Constant
parts are simply written to the output, while format directives are substituted
at runtime with the corresponding data that is captured by the log statement.

The following is the list of predefined log prefix format directives. They are
organized into groups based on what kind of information they provide.

Log statement related format directives:
\begin{itemize}
  \item \ttt{\%l} log level name
  \item \ttt{\%c} log category
\end{itemize}

Current simulation state related format directives:
\begin{itemize}
  \item \ttt{\%e} current event number
  \item \ttt{\%t} current simulation time
  \item \ttt{\%g} current fingerprint if fingerprint verification is enabled in the configuration, otherwise empty
  \item \ttt{\%v} current message or event name
  \item \ttt{\%a} current message or event class name
  \item \ttt{\%n} module name of current event
  \item \ttt{\%m} module path of current event
  \item \ttt{\%o} module class name of current event
  \item \ttt{\%s} simple NED type name of module of current event
  \item \ttt{\%q} fully qualified NED type name of module of current event
  \item \ttt{\%N} context component name
  \item \ttt{\%M} context component path
  \item \ttt{\%O} context component class name
  \item \ttt{\%S} context component NED type simple name
  \item \ttt{\%Q} context component NED type fully qualified name
\end{itemize}

Simulation run related format directives:
\begin{itemize}
  \item \ttt{\%G} config name
  \item \ttt{\%R} run number
  \item \ttt{\%X} network module class name
  \item \ttt{\%Y} network module NED type simple name
  \item \ttt{\%Z} network module NED type fully qualified name
\end{itemize}

C++ source related (where the log statement is) format directives:
\begin{itemize}
  \item \ttt{\%p} source object pointer
  \item \ttt{\%b} source object name
  \item \ttt{\%d} source object path
  \item \ttt{\%z} source class name
  \item \ttt{\%u} source function name
  \item \ttt{\%x} source component NED type simple name
  \item \ttt{\%y} source component NED type fully qualified
  \item \ttt{\%f} source file name
  \item \ttt{\%i} source line number
\end{itemize}

Operating system related format directives:
\begin{itemize}
  \item \ttt{\%w} user time in seconds
  \item \ttt{\%W} human readable wall time
  \item \ttt{\%H} host name
  \item \ttt{\%I} process id
\end{itemize}

Compound field format directives:
\begin{itemize}
  \item \ttt{\%E} event object (class name, name)
  \item \ttt{\%U} module of current event (NED type, full path)
  \item \ttt{\%C} context component (NED type, full path)
  \item \ttt{\%K} context component, if different from current module (NED type, full path)
  \item \ttt{\%J} source component or object (NED type or class, full path or pointer)
  \item \ttt{\%L} source component or object, if different from context component (NED type or class, full path or pointer)
\end{itemize}

Padding format directives:
\begin{itemize}
  \item \ttt{\%[0-9]+} add spaces until specified column
  \item \ttt{\%|} adaptive tabstop: add padding until longest prefix seen so far
  \item \ttt{\%>} function call depth times 2-space indentation (see Enter\_Method, Enter\_Method\_Silent)
\end{itemize}

Conditional format directives:
\begin{itemize}
  \item \ttt{\%?} ignore the following constant part if the preceding directive didn't print anything (useful for separators)
\end{itemize}

Escaping the \% character:
\begin{itemize}
  \item \ttt{\%\%} one \% character
\end{itemize}

\subsection{Configuring Cmdenv}
\label{sec:config-sim:configuring-cmdenv}

In Cmdenv, logging can be configured using \ffilename{omnetpp.ini} configuration
options. The configured settings remain in effect during the whole simulation run
unless overridden programatically.

\begin{itemize}
  \item \fconfig{cmdenv-output-file} redirects standard output to a file
  \item \fconfig{cmdenv-log-prefix} determines the log prefix of each line
  \item \fconfig{<object-full-path>.cmdenv-log-level} restricts output on a per-component basis
\end{itemize}

By default, the log is written to the standard output but it can be redirected to
a file. The output can be completely disabled from \ffilename{omnetpp.ini}, so
that it doesn't slow down simulation when it is not needed. The per-component
runtime log level option must match the full path of the targeted component. The
supported values for this configuration option are the following:

\begin{itemize}
  \item \ttt{off} completely disables log output
  \item \ttt{fatal} omits log output below \ttt{LOGLEVEL\_FATAL}
  \item \ttt{error} omits log output below \ttt{LOGLEVEL\_ERROR}
  \item \ttt{warn} omits log output below \ttt{LOGLEVEL\_WARN}
  \item \ttt{info} omits log output below \ttt{LOGLEVEL\_INFO}
  \item \ttt{detail} omits log output below \ttt{LOGLEVEL\_DETAIL}
  \item \ttt{debug} omits log output below \ttt{LOGLEVEL\_DEBUG}
  \item \ttt{trace} completely enables log output
\end{itemize}

By default, the log prefix format is set to \ttt{"[\%l]\textbackslash{}t"}. The
default setting is intentionally quite simple to avoid cluttered standard output,
it produces similar log output:

\begin{filelisting}
[INFO]  Packet received successfully
[DEBUG] CRC check successful
\end{filelisting}

The log messages are aligned vertically because there's a \ttt{TAB} character
in the format string. Setting the log prefix format to an empty string disables
writing a log prefix altogether. Finally, here is a more detailed format string:
\ttt{"[\%l]\textbackslash{}t\%C for \%E: \%|"}, it produces similar output:

\begin{filelisting}
[INFO]  (IPv4)host.ip for (ICMPMessage)ping0:     Pending (IPv4Datagram)ping0
[INFO]  (ARP)host.arp for (ICMPMessage)ping0:     Starting ARP resolution
[DEBUG] (ARP)host.arp for (ICMPMessage)ping0:     Sending (ARPPacket)arpREQ
[INFO]  (Mac)host.wlan.mac for (ARPPacket)arpREQ: Enqueing (ARPPacket)arpREQ
\end{filelisting}

In express mode, for performance reasons, log output is disabled during the whole
simulation. However, during the simulation finish stage, logging is automatically
re-enabled to allow writing statistical and other results to the log. One can
completely disable all logging by adding following configuration option at the
beginning of \ffilename{omnetpp.ini}:

\begin{inifile}
[General]
**.cmdenv-log-level = off
\end{inifile}

Finally, the following is a more complex example that sets the per-component
runtime log levels for all PHY components to \fmac{LOGLEVEL\_WARN}, except for
all MAC modules where it is set to \fmac{LOGLEVEL\_DEBUG}, and for all other
modules it is set \fmac{LOGLEVEL\_OFF}.

\begin{inifile}
[General]
**.phy.cmdenv-log-level = warn
**.mac.cmdenv-log-level = debug
**.cmdenv-log-level = off
\end{inifile}

\subsection{Configuring Tkenv and Qtenv}
\label{sec:config-sim:configuring-tkenv-and-qtenv}

The graphical user interfaces, Tkenv and Qtenv, provide their own configuration
dialogs where the user can configure logging. These dialogs offer setting the
global runtime log level and the log prefix format string. The per-component
runtime log levels can be set from the context menu of components. As in Cmdenv,
it's also possible to set the log levels to \ttt{off}, effectively disabling
logging globally or for specific components only.

In contrast to Cmdenv, setting the runtime log levels is possible even if the
simulation is already running. This feature allows continuous control over the
level of detail of what is written to the log output. For obvious reasons,
changing the log levels has no effect back in time, so already written log
content in the log windows will not change.

By default, the log prefix format is set to \ttt{"\%l \%C: "}, it produces
similar log output:

\begin{filelisting}
INFO  Network.server.wlan[0].mac: Packet received successfully
DEBUG Network.server.wlan[0].mac: CRC check successful
\end{filelisting}


%%% Local Variables:
%%% mode: latex
%%% TeX-master: "usman"
%%% End:

