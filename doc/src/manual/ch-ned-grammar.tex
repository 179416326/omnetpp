\chapter{NED Language Grammar}
\label{cha:ned-language-grammar}

The NED language\index{ned!language}, the network topology description language of
{\opp} will be given using the extended BNF notation.

Space, horizontal tab and new line characters counts as delimiters,
so one or more of them is required between two elements of the
description which would otherwise be unseparable. '//' (two slashes)
may be used to write comments that last to the end of the line.
The language only distinguishes between lower and upper case
letters in names, but not in keywords.


In this description, the \{xxx...\} notation stands for one or
more xxx's separated with spaces, tabs or new line characters,
and \{xxx,,,\} stands for one or more xxx's, separated with a
comma and (optionally) spaces, tabs or new line characters.


For ease of reading, in some cases we use textual definitions.
The \textit{networkdescription} symbol is the sentence symbol of the
grammar.


\begin{Verbatim}[commandchars=\\\{\}]
        \textbf{notation    meaning}
        [a]         0 or 1 time a
        \{a\}         a
        \{a,,,\}      1 or more times a, separated by commas
        \{a...\}      1 or more times a, separated by spaces
        a|b         a or b
        `a'         the character a
        \textbf{bold}        keyword
        \textit{italic}      identifier


networkdescription ::=
    \{ definition... \}

definition    ::=
      include
    | channeldefinition
    | simpledefinition
    | moduledefinition
    | networkdefinition

include ::=
    \textbf{include} \{ fileName ,,, \} ;

channeldefinition ::=
    \textbf{channel} \textit{channeltype}
     [ \textbf{delay} numericvalue ]
     [ \textbf{error} numericvalue ]
     [ \textbf{datarate} numericvalue ]
    \textbf{endchannel}

simpledefinition ::=
    \textbf{simple} \textit{simplemoduletype}\index{module!simple}
     [ paramblock ]
     [ gateblock ]
    \textbf{endsimple} [ \textit{simplemoduletype} ]

moduledefinition ::=
    \textbf{module} \textit{compoundmoduletype}\index{module!compound}
     [ paramblock ]
     [ gateblock ]
     [ submodblock ]
     [ connblock ]
    \textbf{endmodule} [ \textit{compoundmoduletype} ]

moduletype ::=
    \textit{simplemoduletype} | \textit{compoundmoduletype}

paramblock ::=
    \textbf{parameters:} \{ parameter ,,, \} ;

parameter ::=
    \textit{parametername}
    | \textit{parametername} : \textbf{const} [ \textbf{numeric} ]
    | \textit{parametername} \textbf{: string}
    | \textit{parametername} \textbf{: bool}
    | \textit{parametername} \textbf{: char}
    | \textit{parametername} \textbf{: anytype}

gateblock ::=
    \textbf{gates:}
     [ \textbf{in:} \{ gate ,,, \} ; ]
     [ \textbf{out:} \{ gate ,,, \} ; ]
gate ::=
    \textit{gatename} [ '[]' ]

submodblock ::=
    \textbf{submodules:} \{ submodule... \}

submodule ::=
    \{ \textit{submodulename} : \textit{moduletype} [ vector ]
     [ substparamblock... ]
     [ gatesizeblock... ] \}
  | \{ \textit{submodulename} : \textit{parametername} [ vector ] \textbf{like} \textit{moduletype}
     [ substparamblock... ]
     [ gatesizeblock... ] \}

substparamblock    ::=
    \textbf{parameters} [ \textbf{if} expression ]\textbf{:}
      \{ \textit{substparamname} = substparamvalue,,, \} ;

substparamvalue ::=
    ( [ \textbf{ancestor} ] [ \textbf{ref} ] \textit{name} )
    | parexpression

gatesizeblock ::=
    \textbf{gatesizes} [ \textbf{if} expression ]\textbf{:}
      \{ \textit{gatename} vector ,,, \} ;

connblock ::=
    \textbf{connections} [ \textbf{nocheck} ]\textbf{:} \{ connection ,,, \} ;

connection ::=
     normalconnection | loopconnection

loopconnection ::=
    \textbf{for} \{ index... \} \textbf{do}
      \{ normalconnection ,,, \} ;
    \textbf{endfor}

index ::=
    \textit{indexvariable} '=' expression ``...'' expression

normalconnection ::=
     \{ gate \{ --> | <-- \} gate [ \textbf{if} expression ]\}
   | \{gate --> channel --> gate [ \textbf{if} expression ]\}
   | \{gate <-- channel <-- gate [ \textbf{if} expression ]\}

channel ::=
     \textit{channeltype}
    | [ \textbf{delay} expression ] [ \textbf{error} expression ] [ \textbf{datarate} expression ]


gate ::=
    [ \textit{modulename} [vector]. ] \textit{gatename} [vector]

networkdefinition ::=
    \textbf{network} \textit{networkname} : \textit{moduletype}
     [ substparamblock ]
    \textbf{endnetwork}

vector ::=    '[' expression ']'

parexpression ::=
    expression | otherconstvalue

expression    ::=
      expression + expression
    | expression - expression
    | expression * expression
    | expression / expression
    | expression \% expression
    | expression {\textasciicircum} expression
    | expression == expression
    | expression != expression
    | expression \texttt{<} expression
    | expression \texttt{<}= expression
    | expression \texttt{>} expression
    | expression \texttt{>}= expression
    | expression ? expression : expression
    | expression \textbf{and} expression
    | expression \textbf{or} expression
    | \textbf{not} expression
    | '(' expression ')'
    | \textit{functionname} '(' [ expression ,,, ] ')'
    | - expression
    | numconstvalue
    | inputvalue
    | [ \textbf{ancestor} ] [ \textbf{ref} ] \textit{parametername}
    | \textbf{sizeof} '(' \textit{gatename} ')'
    | \textbf{index}

numconstvalue ::=
    \textit{integerconstant} | \textit{realconstant} | \textit{timeconstant}

otherconstvalue ::=
      '\textit{characterconstant'}
    | "\textit{stringconstant}"
    | \textbf{true}
    | \textbf{false}

inputvalue ::=
    \textbf{input} '(' default , "\textit{prompt-string}" ')'

default ::=
    expression | otherconstvalue
\end{Verbatim}


%%% Local Variables:
%%% mode: latex
%%% TeX-master: "usman"
%%% End:
