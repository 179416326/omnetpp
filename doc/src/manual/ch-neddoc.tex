\chapter{Documenting NED and Messages}
\label{cha:neddoc}

\section{Overview}

{\opp} provides a tool which can generate HTML documentation from NED files
and message definitions. Like Javadoc and Doxygen, \fprog{opp\_neddoc}
makes use of source code comments. \fprog{opp\_neddoc}-generated
documentation lists simple and compound modules, and presents their details
including description, gates, parameters, unassigned submodule parameters
and syntax-highlighted source code.
The documentation also includes clickable network diagrams (exported via the GNED
graphical editor) and module usage diagrams as well as inheritance diagrams
for messages.

\fprog{opp\_neddoc} works well with Doxygen, which means that it can hyperlink
simple modules and message classes to their C++ implementation classes in
the Doxygen documentation. If you also generate the C++ documentation with some
Doxygen features turned on (such as \textit{inline-sources} and
\textit{referenced-by-relation}, combined with
\textit{extract-all}, \textit{extract-private} and \textit{extract-static}),
the result is an easily browsable and very informative presentation of the
source code. Of course, one still has to write documentation comments
in the code.



\section{Authoring the documentation}


\subsection{Documentation comments}

Documentation is embedded in normal comments. All \texttt{//} comments
that are in the ``right place'' (from the documentation tool's
point of view) will be included in the generated documentation.
  \footnote{In contrast, Javadoc and Doxygen use special comments (those
     beginning with \texttt{/**}, \texttt{///}, \texttt{//<} or a similar
     marker) to distinguish documentation from ``normal'' comments in the
     source code. In {\opp} there's no need for that: NED and the message
     syntax is so compact that practically all comments one would want to write
     in them can serve documentation purposes. Still, there is a way to write
     comments that \textit{don't} make it into the documentation -- by starting
     them with \texttt{//\#}.}

Example:

\begin{verbatim}
//
// An ad-hoc traffic generator to test the Ethernet models.
//
simple Gen
    parameters:
        destAddress: string,  // destination MAC address
        protocolId: numeric,  // value for SSAP/DSAP in Ethernet frame
        waitMean: numeric;    // mean for exponential interarrival times
    gates:
        out: out;             // to Ethernet LLC
endsimple
\end{verbatim}

You can also place comments above parameters and gates. This is useful
if they need long explanations. Example:

\begin{verbatim}
//
// Deletes packets and optionally keeps statistics.
//
simple Sink
    parameters:
        // You can turn statistics generation on and off. This is
        // a very long comment because it has to be described what
        // statistics are collected (or not).
        statistics: bool;
    gates:
        in: in;
endsimple
\end{verbatim}

If you want a comment line \textit{not} to appear in the documentation,
begin it with \texttt{//\#}. Those lines will be ignored by the
documentation generation, and can be used to comment out
unused NED code or to make ``private'' comments like \texttt{FIXME} or
\texttt{TBD}.

\begin{verbatim}
//
// An ad-hoc traffic generator to test the Ethernet models.
//# FIXME above description needs to be refined
//
simple Gen
    parameters:
        destAddress: string,  // destination MAC address
        protocolId: numeric,  // value for SSAP/DSAP in Ethernet frame
        //# burstiness: numeric;  -- not yet supported
        waitMean: numeric;    // mean for exponential interarrival times
    gates:
        out: out;             // to Ethernet LLC
endsimple
\end{verbatim}


\subsection{Text layout and formatting}

If you write longer descriptions, you'll need text formatting capabilities.
Text formatting works like in Javadoc or Doxygen -- you can break up the
text into paragraphs and create bulleted/numbered lists without
special commands, and use HTML for more fancy formatting.

Paragraphs are separated by empty lines, like in LaTeX or Doxygen.
Lines beginning with `\ttt{-}' will be turned into bulleted lists,
and lines beginning with `\ttt{-\#}' into numbered lists.

Example:

\begin{verbatim}
//
// Ethernet MAC layer. MAC performs transmission and reception of frames.
//
// Processing of frames received from higher layers:
// - sends out frame to the network
// - no encapsulation of frames -- this is done by higher layers.
// - can send PAUSE message if requested by higher layers (PAUSE protocol,
//   used in switches). PAUSE is not implemented yet.
//
// Supported frame types:
// -# IEEE 802.3
// -# Ethernet-II
//
\end{verbatim}


\subsection{Special tags}

\fname{\opp\_neddoc} understands the following tags and will render them accordingly:
\ttt{@author}, \ttt{@date}, \ttt{@todo}, \ttt{@bug}, \ttt{@see}, \ttt{@since},
\ttt{@warning}, \ttt{@version}. An example usage:

\begin{verbatim}
//
// @author Jack Foo
// @date 2005-02-11
//
\end{verbatim}


\subsection{Additional text formatting using HTML}

Common HTML tags are understood as formatting commands.
The most useful of these tags are: \ttt{<i>..</i>} (italic),
\ttt{<b>..</b>} (bold), \ttt{<tt>..</tt>} (typewriter font),
\ttt{<sub>..</sub>} (subscript), \ttt{<sup>..</sup>} (superscript),
\ttt{<br>} (line break), \ttt{<h3>} (heading),
\ttt{<pre>..</pre>} (preformatted text) and \ttt{<a href=..>..</a>} (link),
as well as a few other tags used for table creation (see below).
For example, \texttt{<i>Hello</i>} will be rendered as ``\textit{Hello}''
(using an italic font).

The complete list of HTML tags interpreted by \ttt{opp\_neddoc} are:
\texttt{<a>}, \texttt{<b>}, \texttt{<body>}, \texttt{<br>}, \texttt{<center>},
\texttt{<caption>}, \texttt{<code>}, \texttt{<dd>}, \texttt{<dfn>}, \texttt{<dl>},
\texttt{<dt>}, \texttt{<em>}, \texttt{<form>}, \texttt{<font>}, \texttt{<hr>},
\texttt{<h1>}, \texttt{<h2>}, \texttt{<h3>}, \texttt{<i>}, \texttt{<input>}, \texttt{<img>},
\texttt{<li>}, \texttt{<meta>}, \texttt{<multicol>}, \texttt{<ol>}, \texttt{<p>}, \texttt{<small>},
\texttt{<span>}, \texttt{<strong>},
\texttt{<sub>}, \texttt{<sup>}, \texttt{<table>}, \texttt{<td>}, \texttt{<th>}, \texttt{<tr>},
\texttt{<tt>}, \texttt{<kbd>}, \texttt{<ul>}, \texttt{<var>}.

Any tags not in the above list will not be interpreted as formatting commands
but will be printed verbatim -- for example, \texttt{<what>bar</what>}
will be rendered literally as ``<what>bar</what>'' (unlike HTML where
unknown tags are simply ignored, i.e. HTML would display ``bar'').

If you insert links to external pages (web sites), its useful to add
the \ttt{target="\_blank"} attribute to ensure pages come up in a new
browser window and not just in the current frame which looks awkward.
(Alternatively, you can use the \ttt{target="\_top"} attribute
which replaces all frames in the current browser).

Examples:

\begin{verbatim}
//
// For more info on Ethernet and other LAN standards, see the
// <a href="http://www.ieee802.org/" target="_blank">IEEE 802
// Committee's site</a>.
//
\end{verbatim}

You can also use the \ttt{<a href=..>} tag to create links within the page:

\begin{verbatim}
//
// See the <a href="#resources">resources</a> in this page.
// ...
// <a name="resources"><b>Resources</b></a>
// ...
//
\end{verbatim}

You can use the \texttt{<pre>..</pre>} HTML tag to insert souce code examples
into the documentation. Line breaks and indentation will be preserved,
but HTML tags continue to be interpreted (or you can turn them off
with \texttt{<nohtml>}, see later).

Example:

\begin{verbatim}
// <pre>
// // my preferred way of indentation in C/C++ is this:
// <b>for</b> (<b>int</b> i=0; i<10; i++)
// {
//     printf(<i>"%d\n"</i>, i);
// }
// </pre>
\end{verbatim}

will be rendered as

\begin{Verbatim}[commandchars=\\\{\}]
// my preferred way of indentation in C/C++ is this:
\textbf{for} (\textbf{int} i=0; i<10; i++)
\{
    printf(\textit{"%d{\textbackslash}n"}, i);
\}
\end{Verbatim}

HTML is also the way to create tables. The example below

\begin{verbatim}
//
// <table border="1">
//   <tr>  <th>#</th> <th>number</th> </tr>
//   <tr>  <td>1</td> <td>one</td>    </tr>
//   <tr>  <td>2</td> <td>two</td>    </tr>
//   <tr>  <td>3</td> <td>three</td>  </tr>
// </table>
//
\end{verbatim}

will be rendered approximately as:

\begin{longtable}{|l|l|}
\hline
\tabheadcol
\tbf{\#} & \tbf{number} \\\hline
1 & one \\\hline
2 & two \\\hline
3 & three \\\hline
\end{longtable}


\subsection{Escaping HTML tags}

Sometimes may need to off interpreting HTML tags (\ttt{<i>}, \ttt{<b>}, etc.)
as formatting instructions, and rather you want them to appear as literal
\ttt{<i>}, \ttt{<b>} texts in the documentation. You can achieve this via
surrounding the text with the \ttt{<nohtml>}...\ttt{</nohtml>} tag.
For example,

\begin{verbatim}
// Use the <nohtml><i></nohtml> tag (like <tt><nohtml><i>this</i></nohtml><tt>)
// to write in <i>italic</i>.
\end{verbatim}

will be rendered as ``Use the <i> tag (like \texttt{<i>this</i>}) to write
in \textit{italic}.''

\ttt{<nohtml>}...\ttt{</nohtml>} will also prevent \fprog{opp\_neddoc}
from hyperlinking words that are accidentally the same as an existing
module or message name. Prefixing the word with a backslash will achieve
the same. That is, either of the following will do:

\begin{verbatim}
// In <nohtml>IP</nohtml> networks, routing is...
\end{verbatim}

\begin{verbatim}
// In \IP networks, routing is...
\end{verbatim}

Both will prevent hyperlinking the word \textit{IP} if you happen to have
an \ttt{IP} module in the NED files.



\subsection{Where to put comments}

You have to put the comments where nedtool will find them.
This is a) above the documented item, or b) after the
documented item, on the same line.

If you put it above, make sure there's no blank line left
between the comment and the documented item. Blank lines
detach the comment from the documented item.

Example:
\begin{verbatim}
// This is wrong! Because of the blank line, this comment is not
// associated with the following simple module!

simple Gen
    parameters:
    ...
endsimple
\end{verbatim}

Do not try to comment groups of parameters together. The result
will be awkward.


\subsection{Customizing the title page}

The title page is the one that appears in the main frame after
opening the documentation in the browser. By default it contains
a boilerplate text with the generic title \textit{``{\opp} Model Documentation''}.
You probably want to customize that, and at least change the title
to the name of the documented simulation model.

You can supply your own version of the title page adding a \ttt{@titlepage}
directive to a file-level comment (a comment that appears at the top of
a NED file, but is separated from the first \ttt{import}, \ttt{channel},
\ttt{module}, etc. definition by at least one blank line).
In theory you can place your title page definition into
any NED or MSG file, but it is probably a good idea to create
a separate \ttt{index.ned} file for it.

The lines you write after the \ttt{@titlepage} line up to the next
\ttt{@page} line (see later) or the end of the comment will be used
as the title page.
You probably want to begin with a title because the documentation
tool doesn't add one (it lets you have full control over the
page contents). You can use the \ttt{<h1>..</h1>} HTML tag
to define a title.

Example:

\begin{verbatim}
//
// @titlepage
// <h1>Ethernet Model Documentation</h1>
//
// This documents the Ethernet model created by David Wu and refined by Andras
// Varga at CTIE, Monash University, Melbourne, Australia.
//
\end{verbatim}


\subsection{Adding extra pages}

You can add new pages to the documentation in a similar way as customizing
the title page. The directive to be used is \ttt{@page}, and it can
appear in any file-level comment (see above).

The syntax of the \ttt{@page} directive is the following:

\begin{verbatim}
// @page filename.html, Title of the Page
\end{verbatim}

Please choose a file name that doesn't collide with the files generated
by the documentation tool (such as \ttt{index.html}).
The page title you supply will appear on the top of the page as well as
in the page index.

The lines after the \ttt{@page} line up to the next \ttt{@page} line
or the end of the comment will be used as the page body.
You don't need to add a title because the documentation tool
automatically adds one.

Example:
\begin{verbatim}
//
// @page structure.html, Directory Structure
//
// The model core model files and the examples have been placed
// into different directories. The <tt>examples/</tt> directory...
//
//
// @page examples.html, Examples
// ...
//
\end{verbatim}

You can create links to the generated pages using standard HTML,
using the \ttt{<a href="...">...</a>} tag. All HTML files are
placed in a single directory, so you don't have to worry about
specifying directories.

Example:
\begin{verbatim}
//
// @titlepage
// ...
// The structure of the model is described <a href="structure.html">here</a>.
//
\end{verbatim}


\subsection{Incorporating externally created pages}

You may want to create pages outside the documentation tool
(e.g. using a HTML editor) and include them in the documentation.
This is possible, all you have to do is declare such pages with
the \ttt{@externalpage} directive in any of the NED files, and
they will be added to the page index. The pages can then be linked to
from other pages using the HTML \ttt{<a href="...">...</a>} tag.

The \ttt{@externalpage} directive is similar in syntax
\ttt{@page}:

\begin{verbatim}
// @externalpage filename.html, Title of the Page
\end{verbatim}

The documentation tool does not check if the page exists
or not. It is your responsibility to copy them manually into
the directory of the generated documentation and then to make
sure the hyperlinks work.



\section{Invoking opp\_neddoc}

The \ttt{opp\_neddoc} tool accepts the following command-line options:

% FIXME explain...

\begin{verbatim}
opp_neddoc - NED and MSG documentation tool, part of OMNeT++
(c) 2003-2004 Andras Varga

Generates HTML model documentation from .ned and .msg files.

Usage: opp_neddoc options files-or-directories ...
 -a, --all    process all *.ned and *.msg files recursively
              ('opp_neddoc -a' is equivalent to 'opp_neddoc .')
 -o <dir>     output directory, defaults to ./html
 -t <filename>, --doxytagfile <filename>
              turn on generating hyperlinks to Doxygen documentation;
              <filename> specifies name of XML tag file generated by Doxygen
 -d <dir>, --doxyhtmldir <dir>
              directory of Doxygen-generated HTML files, relative to the
              opp_neddoc output directory (-o option). -t option must also be
              present to turn on linking to Doxygen. Default: ../api-doc/html
 -n, --no-figures
              do not generate diagrams
 -p, --no-unassigned-pars
              do not document unassigned parameters
 -x, --no-diagrams
              do not generate usage and inheritance diagrams
 -z, --no-source
              do not generate source code listing
 -s, --silent suppress informational messages
 -g, --debug  print invocations of external programs and other info
 -h, --help   displays this help text

Files specified as arguments are parsed and documented. For directories as
arguments, all .ned and .msg files under them (in that directory subtree) are
documented. Wildcards are accepted and they are NOT recursive, e.g.
foo/*.ned does NOT process files in foo/bar/ or any other subdirectory.

Bugs:  (1) handles only files with .ned and .msg extensions, other files are
silently ignored; (2) does not filter out duplicate files (they will show up
multiple times in the documentation); (3) on Windows, file names are handled
case sensitively.
\end{verbatim}


\subsection{Multiple projects}

The generated \ttt{tags.xml} can be used to generate other documentation
that refers to pages in this documentation via HTML links.


\section{How does opp\_neddoc work?}

\ttt{*.ned} and \ttt{*.msg} files are collected (e.g. via the \ttt{find}
command if you used the \ttt{-a} option on Unix) and processed
with \ttt{nedtool}. \ttt{nedtool} parses them and outputs the resulting syntax
tree in XML -- a single large XML file which contains all files.

The \ttt{*.ned} files are processed with the \ttt{-c} (export-diagrams-and-exit)
option of \ttt{gned}. This causes \ttt{gned} to export diagrams for the
compound modules in Postscript. Postscript files are then converted
to GIFs using \ttt{convert} (part of the ImageMagick package).
\ttt{gned} also exports an \ttt{images.xml} file which describes which
image was generated from which compound module, and also contains
additional info (coordinates of submodule rectangles and icons in the image)
for creating clickable image maps.

The XML file containing parsed NED and message files is then processed
with an XSLT stylesheet to generate HTML. XSLT is a very powerful way
of converting an XML document into another XML (or HTML, or text) document.
Additionally, the stylesheet reads \ttt{images.xml} and uses its contents
to make the compound module images clickable.
The stylesheet also outputs a \ttt{tags.xml} file which describes what is documented
in which .html file, so that external documentation can link to this one.

As a final step, the comments in the generated HTML file are processed
with a perl script. The perl script also performs syntax hightlighting
of the source listings in the HTML, and puts hyperlinks on module,
channel, message, etc. names. (It uses the info in the \ttt{tags.xml} file
for the latter task.) This last step, comment formatting and source code
coloring whould have been very difficult to achieve from XSLT, which
(at least in its 1.0 version of the standard) completely lacks powerful
string manipulation functions. (Not even simple find/replace
is supported in strings, let alone regular expressions. Perhaps the
2.0 version of XSLT will improve on this.)

The whole process is controlled by the \ttt{opp\_neddoc} script.





