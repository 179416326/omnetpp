\appendixchapter{Result File Formats}
\label{cha:result-file-formats}

The file format described here applies to \textit{both output vector and
output scalar files}. Their formats are consistent, only the types of
entries occurring into them are different. This unified format also
means that they can be read with a common routine.

Result files are \textit{line oriented}. A line consists of one or more
tokens, separated by whitespace. Tokens either don't
contain whitespace, or whitespace is escaped using a backslash, or
are quoted using double quotes. Escaping within quotes using
backslashes is also permitted.

The first token of a line usually identifies the type of the entry. A
notable exception is an output vector data line, which begins with a
numeric identifier of the given output vector.

A line starting with \# as the first non{}-whitespace character denotes
a comment, and is to be ignored during processing.

Result files are written from simulation runs. A simulation run
generates physically contiguous sets of lines into one or more result
files. (That is, lines from different runs do not arbitrarily mix in
the files.)

A run is identified by a unique textual \textit{runId}, which appears in
all result files written during that run. The runId may appear on the
user interface, so it should be somewhat meaningful to the user.
Nothing should be assumed about the particular format of runId, but it
will be some string concatenated from the simulated network's name, the
time/date, the hostname, and other pieces of data to make it unique.

A simulation run will typically write into two result files (.vec and
.sca). However, when using parallel distributed simulation, the user
will end up with several .vec and .sca files, because different
partitions (a separate process each) will write into different files.
However, all these files will contain the same runId, so it is possible
to relate data that belong together.

Entry types are:

\begin{itemize}
    \item \textbf{version}: result file version
    \item \textbf{run}: simulation run identifier
    \item \textbf{attr}: run, vector, scalar or statistics object attribute
    \item \textbf{param}: module parameter
    \item \textbf{scalar}: scalar data
    \item \textbf{vector}: vector declaration
    \item \textit{vector-id}: vector data
    \item \textbf{statistic}: statistics object
    \item \textbf{field}: field of a statistics object
    \item \textbf{bin}: histogram bin
\end{itemize}



\section{Version}

Specifies the format of the result file. It is written at the beginning of the file.

Syntax:

\hspace{20mm} \textbf{version} \textit{versionNumber}

The version described in this document is 2. Version 1 files are produced
by OMNeT++ 3.3 or earlier.



\section{Run Declaration}

Marks the beginning of a new run in the file. Entries after this line
belong to this run.

Syntax:

\hspace{20mm} \textbf{run} \textit{runId}

Example:

\begin{verbatim}
run TokenRing1-0-20080514-18:19:44-3248
\end{verbatim}

Typically there will be one run per file, but this is not mandatory.
In cases when there are more than one runs in a file and it is not feasible
to keep the entire file in memory during analysis, the offsets of the \textit{run}
lines may be indexed for more efficient random access.

The \textit{run} line may be immediately followed by \textit{attribute} lines.
Attributes may store generic data like the network name, date/time of running
the simulation, configuration options that took effect for the simulation, etc.

Run attribute names used by {\opp} include the following:

Generic attribute names:

\begin{itemize}
    \item \textbf{network}: name of the network simulated
    \item \textbf{datetime}: date/time associated with the run
    \item \textbf{processid}: the PID of the simulation process
    \item \textbf{inifile}: the main configuration file
    \item \textbf{configname}: name of the inifile configuration
    \item \textbf{seedset}: index of the seed-set use for the simulation
\end{itemize}

Attributes associated with parameter studies (iterated runs):

\begin{itemize}
    \item \textbf{runnumber}: the run number within the parameter study
    \item \textbf{experiment}: experiment label
    \item \textbf{measurement}: measurement label
    \item \textbf{replication}: replication label
    \item \textbf{repetition}: the loop counter for repetitions with different seeds
    \item \textbf{iterationvars}: string containing the values of the iteration variables
    \item \textbf{iterationvars2}: string containing the values of the iteration variables
\end{itemize}


An example run header:

\begin{verbatim}
run TokenRing1-0-20080514-18:19:44-3248
attr configname TokenRing1
attr datetime 20080514-18:19:44
attr experiment TokenRing1
attr inifile omnetpp.ini
attr iterationvars ""
attr iterationvars2 $repetition=0
attr measurement ""
attr network TokenRing
attr processid 3248
attr repetition 0
attr replication #0
attr resultdir results
attr runnumber 0
attr seedset 0
\end{verbatim}



\section{Attributes}

Contains an attribute for the preceding run, vector, scalar or
statistics object. Attributes can be used for saving arbitrary
extra information for objects; processors should ignore unrecognized
attributes.

Syntax:

\hspace{20mm} \textbf{attr} \textit{name} \textit{value}

Example:

\begin{verbatim}
attr network "largeNet"
\end{verbatim}



\section{Module Parameters}

Contains a module parameter value for the given run. This is needed so
that module parameters may be included in the analysis (e.g. to
identify the load for a ``thruput vs load'' plot).

It may not be practical to simply store all parameters of all modules in the
result file, because there may be too many. We assume that NED files are
invariant and don't store parameters defined in them. However, we store
parameter assignments that come from omnetpp.ini, in their original
wildcard form (i.e. not expanded) to conserve space. Parameter values
entered interactively by the user are also stored.

When the original NED files are present, it should thus be possible to
reconstruct all parameters for the given simulation.

Syntax:

\hspace{20mm} \textbf{param} \textit{parameterNamePattern} \textit{value}

Example:

\begin{verbatim}
param **.gen.sendIaTime  exponential(0.01)
param **.gen.msgLength   10
param **.fifo.bitsPerSec 1000
\end{verbatim}


\section{Scalar Data}

Contains an output scalar value.

Syntax:

\hspace{20mm} \textbf{scalar} \textit{moduleName} \textit{scalarName} \textit{value}

Examples:

\begin{verbatim}
scalar "net.switchA.relay" "processed frames" 100
\end{verbatim}

Scalar lines may be immediately followed by \textit{attribute} lines.
\opp uses the following vector attributes:

\begin{itemize}
    \item \textbf{unit}: physical unit, e.g. \ttt{s} for seconds
\end{itemize}



\section{Vector Declaration}

Defines an output vector.

Syntax:

\hspace{20mm} \textbf{vector} \textit{vectorId} \ \textit{moduleName} \textit{vectorName}

\hspace{20mm} \textbf{vector }\textit{vectorId} \textit{moduleName vectorName columnSpec}

Where \textit{columnSpec} is a string, encoding the meaning and ordering
the columns of data lines. Characters of the string mean:

\begin{itemize}
  \item \textbf{E} event number
  \item \textbf{T} simulation time
  \item \textbf{V} vector value
\end{itemize}

Common values are \ttt{TV} and \ttt{ETV}. The default value is \ttt{TV}.

Vector lines may be immediately followed by \textit{attribute} lines.
\opp uses the following vector attributes:

\begin{itemize}
    \item \textbf{unit}: physical unit, e.g. \ttt{s} for seconds
    \item \textbf{enum}: symbolic names for values of the vector;
          syntax is \ttt{"IDLE=0, BUSY=1, OFF=2"}
    \item \textbf{type}: data type, one of \ttt{int}, \ttt{double} and \ttt{enum}
    \item \textbf{interpolationmode}: hint for interpolation mode on the
          chart: \ttt{none} (=do not connect the dots), \ttt{sample-hold},
          \ttt{backward-sample-hold}, \ttt{linear}
    \item \textbf{min}: minimum value
    \item \textbf{max}: maximum value
\end{itemize}




\section{Vector Data}

Adds a value to an output vector. This is the same as in older output
vector files.

Syntax:

\hspace{20mm} {\itshape vectorId column1 column2 ...}

Simulation times and event numbers \textit{within an output vector} are
required to be in increasing order.

Performance note: Data lines belonging to the same output vector may be
written out in clusters (of sizes roughly multiple of the disk's
physical block size). Then, since an output vector file is typically
not kept in memory during analysis, indexing the start offsets of these
clusters allows one to read the file and seek in it more efficiently.
This does not require any change or extension to the file format.



\section{Statistics Object}

Represents a statistics object.

Syntax:

\hspace{20mm} \textbf{statistic} \textit{moduleName} \ \textit{statisticName}

Example:

\begin{verbatim}
statistic Aloha.server 	"collision multiplicity"
\end{verbatim}

A \textit{statistic} line may be followed by \textit{field} and \textit{attribute} lines,
and a series of \textit{bin} lines that represent histogram data.

\opp uses the following attributes:

\begin{itemize}
    \item \textbf{unit}: physical unit, e.g. \ttt{s} for seconds
    \item \textbf{isDiscrete}: boolean (\ttt{0} or \ttt{1}), meaning whether
          observations are integers
\end{itemize}

A full example with fields, attributes and histogram bins:

\begin{verbatim}
statistic Aloha.server 	"collision multiplicity"
field count 13908
field mean 6.8510209951107
field stddev 5.2385484477843
field sum 95284
field sqrsum 1034434
field min 2
field max 65
attr isDiscrete  1
attr unit  packets
bin	-INF	0
bin	-0.5	0
bin	0.5	0
bin	1.5	2254
bin	2.5	2047
bin	3.5	1586
bin	4.5	1428
bin	5.5	1101
bin	6.5	952
bin	7.5	785
...
bin	51.5	2
\end{verbatim}


\section{Field}

Represents a field in a statistics object.

Syntax:

\hspace{20mm} \textbf{field} \textit{fieldName} \ \textit{value}

Example:

\begin{verbatim}
field sum 95284
\end{verbatim}

Fields:

\begin{itemize}
    \item \tbf{count}: observation count
    \item \tbf{mean}: mean of the observations
    \item \tbf{stddev}: standard deviation
    \item \tbf{sum}: sum of the observations
    \item \tbf{sqrsum}: sum of the squared observations
    \item \tbf{min}: minimum of the observations
    \item \tbf{max}: maximum of the observations
\end{itemize}

For weighted statistics, additionally the following fields may be recorded:

\begin{itemize}
    \item \tbf{weights}: sum of the weights
    \item \tbf{weightedSum}: the weighted sum of the observations
    \item \tbf{sqrSumWeights}:  sum of the squared weights
    \item \tbf{weightedSqrSum}: weighted sum of the squared observations
\end{itemize}



\section{Histogram Bin}

Represents a bin in a histogram object.

Syntax:

\hspace{20mm} \textbf{bin} \textit{binLowerBound} \textit{value}

Histogram name and module is defined on the \textbf{statistic} line,
which is followed by several \textbf{bin} lines to contain data. Any
non{}-\textbf{bin} line marks the end of the histogram data.

The \textit{binLowerBound }column of \textbf{bin} lines represent the
lower bound of the given histogram cell. \textbf{Bin} lines are in
increasing \textit{binLowerBound} order.

The \textit{value} column of \textbf{bin} lines represent observation
count in the given cell: \textit{value k} is the number of observations
greater or equal than \textit{binLowerBound k}, but smaller than
\textit{binLowerBound k+1}. \textit{Value} is not necessarily an
integer, because the cKSplit and cPSquare algorithms produce
non{}-integer estimates. The first \textbf{bin} line is the underflow
cell, and the last \textbf{bin} line is the overflow cell.


Example:

\begin{verbatim}
bin -INF  0
bin 0 4
bin 2 6
bin 4 2
bin 6 1
\end{verbatim}


%%% Local Variables:
%%% mode: latex
%%% TeX-master: "usman"
%%% End:
