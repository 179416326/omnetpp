\appendixchapter{Result File Formats}
\label{cha:result-file-formats}

The file format described here applies to \textit{both output vector and
output scalar files}. Their formats are consistent, only the types of
entries occurring into them are different. This unified format also
means that they can be read with a common routine.

Result files are \textit{line oriented}. A line consists of one or more
tokens, separated by whitespace. Tokens either don't
contain whitespace, or or whitespace is escaped using a backslash, or
are quoted using double quotes. Escaping within quotes using
backslashes is also permitted.

The first token of a line usually identifies the type of the entry. A
notable exception is an output vector data line, which begins with a
numeric identifier of the given output vector.

A line starting with \# as the first non{}-whitespace character denotes
a comment, and is to be ignored during processing.

Result files are written from simulation runs. A simulation run
generates physically contiguous sets of lines into one or more result
files. (That is, lines from different runs do not arbitrarily mix in
the files.)


A run is identified by a unique textual\textit{ runId}, which appears in
all result files written during that run. The runId may appear on the
user interface, so it should be somewhat meaningful to the user.
Nothing should be assumed about the particular format of runId, but it
will be some string concatenated from the simulated network's name, the
time/date, the hostname, and other pieces of data to make it unique.


A simulation run will typically write into two result files (.vec and
.sca). However, when using parallel distributed simulation, the user
will end up with several .vec and .sca files, because different
partitions (a separate process each) will write into different files.
However, all these files will contain the same runId, so it is possible
to relate data that belong together.

\bigskip

Entry types:

{\bfseries
``Run''}

Marks the beginning of a new run in the file. Entries after this line
belong to this run.


\bigskip

Format:

\textbf{run} \textit{runId}


\bigskip

The following forms are obsolete, but should be understood nevertheless:

\textbf{run} \textit{runNumber}

\textbf{run} \textit{runNumber} \textit{networkName}

\textbf{run} \textit{runNumber} \textit{networkName} \textit{dateTime}


\bigskip

These forms can be distinguished from the first one during processing,
because runId is not numeric while runNumber is. During processing, the
program can construct an artificial runId from the file name, run
number, and the line number at which this line appears in the file.
(The runNumber comes from omnetpp.ini, and it is not guaranteed to be
unique).

In the new format, runNumber, networkName and dateTime will appear on
separate lines as \textit{run} \textit{attributes}.


\bigskip

Old output vector files don't contain a ``run'' line. During processing,
if any line type is encountered before a ``run'' line, then an implicit
\ ``run'' has to be created and subsequenet entries be assumed to be
part of that.


\bigskip

Performance note: if not the whole file is kept in memory during
analysis, then the runs in the file (i.e. offsets of the ``run'' lines)
may be indexed for more efficient random access.


\bigskip

Example:

{\ttfamily
run
{\textquotedbl}largeNet{}-20050710{}-14:34:11{}-localhost{}-12374{\textquotedbl}}


\bigskip

{\bfseries
``Run Attributes''}

Contains an attribute for the current run. These attributes include the
network name, the time/date of execution, the
experiment/measurement/replication labels, the random number seeds,
configuration options that took effect such as the scheduler class,
etc.


\bigskip

Format:

\textbf{attr} \textit{name} \textit{value}


\bigskip

\textbf{TODO} define the list of recognized attribute names


\bigskip

Example

{\ttfamily
attr run 1}

{\ttfamily
attr network {\textquotedbl}largeNet{\textquotedbl} }

{\ttfamily
attr date {\textquotedbl}2005{}-07{}-10 14:34:11{\textquotedbl}}

{\ttfamily
attr host {\textquotedbl}localhost{\textquotedbl} }

{\ttfamily
attr inifile {\textquotedbl}xxx.ini{\textquotedbl}}

{\ttfamily
attr experiment {\textquotedbl}blabla{\textquotedbl} }

{\ttfamily
attr measurement {\textquotedbl}rtete{\textquotedbl} }

{\ttfamily
attr replication {\textquotedbl}12th{\textquotedbl}}

{\ttfamily
attr numseeds 2 }

{\ttfamily
attr seed{}-0{}-mt 573367}

{\ttfamily
attr seed{}-1{}-mt 124643}


\bigskip

{\bfseries
``Param''}

Contains a module parameter value for the given run. This is needed so
that module parameters may be included in the analysis (e.g. to
identify the load for a ``thruput vs load'' plot).


\bigskip

It is not feasible to simply store all parameters of all modules in the
result file (it's just too much). We assume that NED files are
invariant and don't store parameters defined in them. However, we store
parameter assignments that come from omnetpp.ini, in their original
wildcard form (i.e. not expanded) to conserve space. Parameter values
entered interactively by the user are also stored.


\bigskip

When the original NED files are present, it should thus be possible to
reconstruct all parameters for the given simulation.



Format:

\textbf{param} \textit{parameterNamePattern} \textit{value}


\bigskip

Example:

{\ttfamily
param **.gen.sendIaTime \ \ exponential(0.01)}

{\ttfamily
param **.gen.msgLength \ \ \ 10}

{\ttfamily
param **.fifo.bitsPerSec \ 1000}


\bigskip

{\bfseries
``Scalar''}

Contains an output scalar value. This is the same as in older output
scalar files.

\

Format:

\textbf{scalar} \textit{moduleName} \textit{scalarName} \textit{value}


\bigskip

\textbf{TODO} room to include the unit (seconds, bits, megabit/second,
etc), and possible extra data (for tagging, commenting, etc?)


\bigskip

Examples:

{\ttfamily
scalar {\textquotedbl}net.switchA.relay{\textquotedbl}
{\textquotedbl}processed frames{\textquotedbl} 100}


\bigskip

{\bfseries
``Vector''}

Defines an output vector. This is the same as in older output vector
files.

Format:

\textbf{vector} \textit{vectorId} \ \textit{moduleName}
\ \textit{vectorName}

\textbf{vector} \textit{vectorId} \ \textit{moduleName}
\ \textit{vectorName} \ 1

\textbf{vector }\textit{vectorId} \textit{moduleName vectorName columns}


\textit{ }Where columns is a string encoding the meaning and ordering
the columns of data lines. Characters of the string mean:

  \textbf{E} event number
  \textbf{T} simulation time
  \textbf{V} vector value

The default value of columns is 'TV'
for compatibility with old vector files.


\bigskip

{\bfseries
``Vector Attributes''}

Vector attributes may follow the definition of vectors. These attibutes
include vector unit, enum definitions, interpolation mode, etc.


\bigskip

Format:

\textbf{attr} name value

\


\bigskip

{\bfseries
``Vector Data''}

Adds a value to an output vector. This is the same as in older output
vector files.

Format:

{\itshape vectorId column1 column2 ...}

Simulation times and event numbers \textit{within an output vector} are
required to be in increasing order.

Performance note: Data lines belonging to the same output vector may be
written out in clusters (of sizes roughly multiple of the disk's
physical block size). Then, since an output vector file is typically
not kept in memory during analysis, indexing the start offsets of these
clusters allows one to read the file and seek in it more efficiently.
This does not require any change or extension to the file format.

\bigskip

{\bfseries ``Histogram''}

Contains histogram data.

Format:

\textbf{histogram} \textit{moduleName} \ \textit{histogramName}

\textbf{bin} {--}INF \textit{value0}

\textbf{bin} \textit{binLowerBound1} \textit{value1}

\textbf{bin} \textit{binLowerBound2} \textit{value2}

{\bfseries {\dots}}

Histogram name and module is defined on the \textbf{histogram} line,
which is followed by several \textbf{bin} lines to contain data. Any
non{}-\textbf{bin} line marks the end of the histogram data.

The \textit{binLowerBound }column of \textbf{bin} lines represent the
lower bound of the given histogram cell. \textbf{Bin} lines are in
increasing \textit{binLowerBound} order.

The \textit{value} column of \textbf{bin} lines represent observation
count in the given cell: \textit{value k} is the number of observations
greater or equal than \textit{binLowerBound k}, but smaller than
\textit{binLowerBound k+1}. \textit{Value} is not necessarily an
integer, because the cKSplit and cPSquare algorithms produce
non{}-integer estimates. The first \textbf{bin} line is the underflow
cell, and the last \textbf{bin} line is the overflow cell.



%%% Local Variables:
%%% mode: latex
%%% TeX-master: "usman"
%%% End:
