\appendixchapter{Configuration Options}
\label{cha:config-options}

\section{Configuration Options}

%% To bring this chapter up to date, run opp_run -h latexconfig, and paste
%% the result. When committing, make sure no option gets lost (e.g parsim
%% or MPI-related options; opp_run can only print options from code that
%% enabled at compile time!)

This section lists all configuration options that are available in ini files.
A similar list can be obtained from any simulation executable by running it
with the \ttt{-h configdetails} option.

\begin{description}
\item[check-signals] = \textit{<bool>}, default: \ttt{true}; per-run setting \\
    Controls whether the simulation kernel will validate signals emitted by
    modules and channels against signal declarations (@signal properties) in
    NED files. The default setting depends on the build type: 'true' in DEBUG,
    and 'false' in RELEASE mode.
\item[cmdenv-autoflush] = \textit{<bool>}, default: \ttt{false}; per-run setting \\
    Call fflush(stdout) after each event banner or status update; affects both
    express and normal mode. Turning on autoflush may have a performance
    penalty, but it can be useful with printf-style debugging for tracking down
    program crashes.
\item[cmdenv-config-name] = \textit{<string>}; global setting \\
    Specifies the name of the configuration to be run (for a value `Foo',
    section [Config Foo] will be used from the ini file). See also
    cmdenv-runs-to-execute=. The -c command line option overrides this setting.
\item[<object-full-path>.cmdenv-ev-output] = \textit{<bool>}, default: \ttt{true}; per-object setting \\
    When cmdenv-express-mode=false: whether Cmdenv should print log messages
    (EV<<, EV\_INFO, etc.) from the selected modules.
\item[cmdenv-event-banner-details] = \textit{<bool>}, default: \ttt{false}; per-run setting \\
    When cmdenv-express-mode=false: print extra information after event
    banners.
\item[cmdenv-event-banners] = \textit{<bool>}, default: \ttt{true}; per-run setting \\
    When cmdenv-express-mode=false: turns printing event banners on/off.
\item[cmdenv-express-mode] = \textit{<bool>}, default: \ttt{true}; per-run setting \\
    Selects ``normal'' (debug/trace) or ``express'' mode.
\item[cmdenv-extra-stack] = \textit{<double>}, unit=\ttt{B}, default: \ttt{8KiB}; global setting \\
    Specifies the extra amount of stack that is reserved for each activity()
    simple module when the simulation is run under Cmdenv.
\item[cmdenv-interactive] = \textit{<bool>}, default: \ttt{false}; global setting \\
    Defines what Cmdenv should do when the model contains unassigned
    parameters. In interactive mode, it asks the user. In non-interactive mode
    (which is more suitable for batch execution), Cmdenv stops with an error.
\item[cmdenv-log-format] = \textit{<string>}, default: \ttt{[\%l]	}; per-run setting \\
    Specifies the format string that determines the prefix of each log line.
    The format string may contain format directives in the syntax '\%x' (a '\%'
    followed by a single format character).  For example '\%l' stands for log
    level, and '\%J' for source component. See the manual for the list of
    available format characters.
\item[cmdenv-log-level] = \textit{<string>}, default: \ttt{DEBUG}; per-run setting \\
    Specifies the level of detail recorded by log statements, output below the
    specified level is omitted. This setting is with AND relationship with
    per-component log level settings. Available values are (case insensitive):
    fatal, error, warn, info, detail, debug or trace. Note that the level of
    detail is also controlled by the specified per component log levels and the
    GLOBAL\_COMPILETIME\_LOGLEVEL macro that is used to completely remove log
    statements from the executable.
\item[cmdenv-message-trace] = \textit{<bool>}, default: \ttt{false}; per-run setting \\
    When cmdenv-express-mode=false: print a line per message sending (by
    send(),scheduleAt(), etc) and delivery on the standard output.
\item[cmdenv-module-messages] = \textit{<bool>}, default: \ttt{true}; per-run setting \\
    When cmdenv-express-mode=false: turns printing module EV<< output on/off.
\item[cmdenv-output-file] = \textit{<filename>}; global setting \\
    When a filename is specified, Cmdenv redirects standard output into the
    given file. This is especially useful with parallel simulation. See the
    `fname-append-host' option as well.
\item[cmdenv-performance-display] = \textit{<bool>}, default: \ttt{true}; per-run setting \\
    When cmdenv-express-mode=true: print detailed performance information.
    Turning it on results in a 3-line entry printed on each update, containing
    ev/sec, simsec/sec, ev/simsec, number of messages created/still
    present/currently scheduled in FES.
\item[cmdenv-runs-to-execute] = \textit{<string>}; global setting \\
    Specifies which runs to execute from the selected configuration (see
    cmdenv-config-name=). It accepts a comma-separated list of run numbers or
    run number ranges, e.g. 1,3..4,7..9. If the value is missing, Cmdenv
    executes all runs in the selected configuration. The -r command line option
    overrides this setting.
\item[cmdenv-status-frequency] = \textit{<double>}, unit=\ttt{s}, default: \ttt{2s}; per-run setting \\
    When cmdenv-express-mode=true: print status update every n seconds.
\item[configuration-class] = \textit{<string>}; global setting \\
    Part of the Envir plugin mechanism: selects the class from which all
    configuration information will be obtained. This option lets you replace
    omnetpp.ini with some other implementation, e.g. database input. The
    simulation program still has to bootstrap from an omnetpp.ini (which
    contains the configuration-class setting). The class should implement the
    cConfigurationEx interface.
\item[constraint] = \textit{<string>}; per-run setting \\
    For scenarios. Contains an expression that iteration variables (\$\{\}
    syntax) must satisfy for that simulation to run. Example: \$i < \$j+1.
\item[cpu-time-limit] = \textit{<double>}, unit=\ttt{s}; per-run setting \\
    Stops the simulation when CPU usage has reached the given limit. The
    default is no limit.
\item[debug-on-errors] = \textit{<bool>}, default: \ttt{false}; global setting \\
    When set to true, runtime errors will cause the simulation program to break
    into the C++ debugger (if the simulation is running under one, or
    just-in-time debugging is activated). Once in the debugger, you can view
    the stack trace or examine variables.
\item[debug-statistics-recording] = \textit{<bool>}, default: \ttt{false}; per-run setting \\
    Turns on the printing of debugging information related to statistics
    recording (@statistic properties)
\item[debugger-attach-command] = \textit{<string>}, default: \ttt{nemiver --attach=\%u \&}; global setting \\
    Command line to launch the debugger. It must contain exactly one percent
    sign, as '\%u', which will be replaced by the PID of this process. The
    command must not block (i.e. it should end in '\&' on Unix-like systems).
\item[debugger-attach-on-error] = \textit{<bool>}, default: \ttt{false}; global setting \\
    When set to true, runtime errors and crashes will trigger an external
    debugger to be launched, allowing you to perform just-in-time debugging on
    the simulation process. The debugger command is configurable. Note that
    debugging (i.e. attaching to) a non-child process needs to be explicitly
    enabled on some systems, e.g. Ubuntu.
\item[debugger-attach-on-startup] = \textit{<bool>}, default: \ttt{false}; global setting \\
    When set to true, the simulation program will launch an external debugger
    attached to it, allowing you to set breakpoints before proceeding. The
    debugger command is configurable.  Note that debugging (i.e. attaching to)
    a non-child process needs to be explicitly enabled on some systems, e.g.
    Ubuntu.
\item[debugger-attach-wait-time] = \textit{<double>}, unit=\ttt{s}, default: \ttt{20s}; global setting \\
    An interval to wait after launching the external debugger, to give the
    debugger time to start up and attach to the simulation process.
\item[description] = \textit{<string>}; per-run setting \\
    Descriptive name for the given simulation configuration. Descriptions get
    displayed in the run selection dialog.
\item[eventlog-file] = \textit{<filename>}, default: \ttt{\$\{resultdir\}/\$\{configname\}-\$\{runnumber\}.elog}; per-run setting \\
    Name of the eventlog file to generate.
\item[eventlog-message-detail-pattern] = \textit{<custom>}; per-run setting \\
    A list of patterns separated by '|' character which will be used to write
    message detail information into the eventlog for each message sent during
    the simulation. The message detail will be presented in the sequence chart
    tool. Each pattern starts with an object pattern optionally followed by ':'
    character and a comma separated list of field patterns. In both patterns
    and/or/not/* and various field match expressions can be used. The object
    pattern matches to class name, the field pattern matches to field name by
    default.
      EVENTLOG-MESSAGE-DETAIL-PATTERN := ( DETAIL-PATTERN '|' )*
    DETAIL\_PATTERN
      DETAIL-PATTERN := OBJECT-PATTERN [ ':' FIELD-PATTERNS ]
      OBJECT-PATTERN := MATCH-EXPRESSION
      FIELD-PATTERNS := ( FIELD-PATTERN ',' )* FIELD\_PATTERN
      FIELD-PATTERN := MATCH-EXPRESSION
    Examples (enter them without quotes):
      "*": captures all fields of all messages
      "*Frame:*Address,*Id": captures all fields named somethingAddress and
    somethingId from messages of any class named somethingFrame
      "MyMessage:declaredOn(MyMessage)": captures instances of MyMessage
    recording the fields declared on the MyMessage class
      "*:(not declaredOn(cMessage) and not declaredOn(cNamedObject) and not
    declaredOn(cObject))": records user-defined fields from all messages
\item[eventlog-recording-intervals] = \textit{<custom>}; per-run setting \\
    Simulation time interval(s) when events should be recorded. Syntax:
    [<from>]..[<to>],... That is, both start and end of an interval are
    optional, and intervals are separated by comma. Example: ..10.2, 22.2..100,
    233.3..
\item[experiment-label] = \textit{<string>}, default: \ttt{\$\{configname\}}; per-run setting \\
    Identifies the simulation experiment (which consists of several,
    potentially repeated measurements). This string gets recorded into result
    files, and may be referred to during result analysis.
\item[extends] = \textit{<string>}; per-run setting \\
    Name of the configuration this section is based on. Entries from that
    section will be inherited and can be overridden. In other words,
    configuration lookups will fall back to the base section.
\item[fingerprint] = \textit{<string>}; per-run setting \\
    The expected fingerprints of the simulation. If you need multiple
    fingerprints, separate them with commas. When provided, the fingerprints
    will be calculated from the specified properties of simulation events,
    messages, and statistics during execution, and checked against the provided
    values. Fingerprints are suitable for crude regression tests. As
    fingerprints occasionally differ across platforms, more than one value can
    be specified for a single fingerprint, separated by spaces, and a match
    with any of them will be accepted. To obtain a fingerprint, enter a dummy
    value (such as "0000"), and run the simulation.
\item[fingerprint-class] = \textit{<string>}, default: \ttt{omnetpp::cSingleFingerprint}; global setting \\
    Part of the Envir plugin mechanism: selects the fingerprint class to be
    used to calculate the simulation fingerprint. The class has to implement
    the cFingerprint interface.
\item[fingerprint-events] = \textit{<string>}, default: \ttt{*}; per-run setting \\
    Configures the fingerprint calculator to consider only certain events. The
    value is a pattern that will be matched against the event name by default.
    It may also be an expression containing pattern matching characters, field
    access, and logical operators. The default setting is '*' which includes
    all events in the calculated fingerprint. If you configured multiple
    fingerprints, separate filters with commas.
\item[fingerprint-ingredients] = \textit{<string>}, default: \ttt{tplx}; per-run setting \\
    Specifies the list of ingredients to be taken into account for fingerprint
    computation. Each character corresponds to one ingredient: 'e' event
    number, 't' simulation time, 'n' message (event) full name, 'c' message
    (event) class name, 'k' message kind, 'l' message bit length, 'o' message
    control info class name, 'd' message data, 'i' module id, 'm' module full
    name, 'p' module full path, 'a' module class name, 'r' random numbers
    drawn, 's' scalar results, 'z' statistic results, 'v' vector results, 'x'
    extra data provided by modules. Note: ingredients specified in an expected
    fingerprint (characters after the '/' in the fingerprint value) take
    precedence over this setting. If you configured multiple fingerprints,
    separate ingredients with commas.
\item[fingerprint-modules] = \textit{<string>}, default: \ttt{*}; per-run setting \\
    Configures the fingerprint calculator to consider only certain modules. The
    value is a pattern that will be matched against the module full path by
    default. It may also be an expression containing pattern matching
    characters, field access, and logical operators. The default setting is '*'
    which includes all events in all modules in the calculated fingerprint. If
    you configured multiple fingerprints, separate filters with commas.
\item[fingerprint-results] = \textit{<string>}, default: \ttt{*}; per-run setting \\
    Configures the fingerprint calculator to consider only certain results. The
    value is a pattern that will be matched against the result full path by
    default. It may also be an expression containing pattern matching
    characters, field access, and logical operators. The default setting is '*'
    which includes all results in all modules in the calculated fingerprint. If
    you configured multiple fingerprints, separate filters with commas.
\item[fname-append-host] = \textit{<bool>}; global setting \\
    Turning it on will cause the host name and process Id to be appended to the
    names of output files (e.g. omnetpp.vec, omnetpp.sca). This is especially
    useful with distributed simulation. The default value is true if parallel
    simulation is enabled, false otherwise.
\item[futureeventset-class] = \textit{<string>}, default: \ttt{omnetpp::cEventHeap}; global setting \\
    Part of the Envir plugin mechanism: selects the class for storing the
    future events in the simulation. The class has to implement the
    cFutureEventSet interface.
\item[image-path] = \textit{<path>}; global setting \\
    A semicolon-separated list of directories that contain module icons and
    other resources. This list with be concatenated with OMNETPP\_IMAGE\_PATH.
\item[load-libs] = \textit{<filenames>}; global setting \\
    A space-separated list of dynamic libraries to be loaded on startup. The
    libraries should be given without the `.dll' or `.so' suffix -- that will
    be automatically appended.
\item[<object-full-path>.log-level] = \textit{<string>}, default: \ttt{TRACE}; per-object setting \\
    Specifies the per-component level of detail recorded by log statements,
    output below the specified level is omitted. Available values are (case
    insensitive): fatal, error, warn, info, detail, debug or trace. Note that
    the level of detail is also controlled by the globally specified runtime
    log level and the GLOBAL\_COMPILETIME\_LOGLEVEL macro that is used to
    completely remove log statements from the executable.
\item[max-module-nesting] = \textit{<int>}, default: \ttt{50}; per-run setting \\
    The maximum allowed depth of submodule nesting. This is used to catch
    accidental infinite recursions in NED.
\item[measurement-label] = \textit{<string>}, default: \ttt{\$\{iterationvars\}}; per-run setting \\
    Identifies the measurement within the experiment. This string gets recorded
    into result files, and may be referred to during result analysis.
\item[<object-full-path>.module-eventlog-recording] = \textit{<bool>}, default: \ttt{true}; per-object setting \\
    Enables recording events on a per module basis. This is meaningful for
    simple modules only.
    Example:
     **.router[10..20].**.module-eventlog-recording = true
     **.module-eventlog-recording = false
\item[ned-path] = \textit{<path>}; global setting \\
    A semicolon-separated list of directories. The directories will be regarded
    as roots of the NED package hierarchy, and all NED files will be loaded
    from their subdirectory trees. This option is normally left empty, as the
    OMNeT++ IDE sets the NED path automatically, and for simulations started
    outside the IDE it is more convenient to specify it via a command-line
    option or the NEDPATH environment variable.
\item[network] = \textit{<string>}; per-run setting \\
    The name of the network to be simulated.  The package name can be omitted
    if the ini file is in the same directory as the NED file that contains the
    network.
\item[num-rngs] = \textit{<int>}, default: \ttt{1}; per-run setting \\
    The number of random number generators.
\item[output-scalar-file] = \textit{<filename>}, default: \ttt{\$\{resultdir\}/\$\{configname\}-\$\{runnumber\}.sca}; per-run setting \\
    Name for the output scalar file.
\item[output-scalar-file-append] = \textit{<bool>}, default: \ttt{false}; per-run setting \\
    What to do when the output scalar file already exists: append to it
    (OMNeT++ 3.x behavior), or delete it and begin a new file (default).
\item[output-scalar-precision] = \textit{<int>}, default: \ttt{14}; per-run setting \\
    The number of significant digits for recording data into the output scalar
    file. The maximum value is ~15 (IEEE double precision).
\item[output-vector-file] = \textit{<filename>}, default: \ttt{\$\{resultdir\}/\$\{configname\}-\$\{runnumber\}.vec}; per-run setting \\
    Name for the output vector file.
\item[output-vector-precision] = \textit{<int>}, default: \ttt{14}; per-run setting \\
    The number of significant digits for recording data into the output vector
    file. The maximum value is ~15 (IEEE double precision). This setting has no
    effect on the "time" column of output vectors, which are represented as
    fixed-point numbers and always get recorded precisely.
\item[output-vectors-memory-limit] = \textit{<double>}, unit=\ttt{B}, default: \ttt{16MiB}; per-run setting \\
    Total memory that can be used for buffering output vectors. Larger values
    produce less fragmented vector files (i.e. cause vector data to be grouped
    into larger chunks), and therefore allow more efficient processing later.
\item[outputscalarmanager-class] = \textit{<string>}, default: \ttt{omnetpp::envir::cFileOutputScalarManager}; global setting \\
    Part of the Envir plugin mechanism: selects the output scalar manager class
    to be used to record data passed to recordScalar(). The class has to
    implement the cIOutputScalarManager interface.
\item[outputvectormanager-class] = \textit{<string>}, default: \ttt{omnetpp::envir::cIndexedFileOutputVectorManager}; global setting \\
    Part of the Envir plugin mechanism: selects the output vector manager class
    to be used to record data from output vectors. The class has to implement
    the cIOutputVectorManager interface.
\item[parallel-simulation] = \textit{<bool>}, default: \ttt{false}; global setting \\
    Enables parallel distributed simulation.
\item[<object-full-path>.param-record-as-scalar] = \textit{<bool>}, default: \ttt{false}; per-object setting \\
    Applicable to module parameters: specifies whether the module parameter
    should be recorded into the output scalar file. Set it for parameters whose
    value you'll need for result analysis.
\item[parsim-communications-class] = \textit{<string>}, default: \ttt{omnetpp::cFileCommunications}; global setting \\
    If parallel-simulation=true, it selects the class that implements
    communication between partitions. The class must implement the
    cParsimCommunications interface.
\item[parsim-debug] = \textit{<bool>}, default: \ttt{true}; global setting \\
    With parallel-simulation=true: turns on printing of log messages from the
    parallel simulation code.
\item[parsim-filecommunications-prefix] = \textit{<string>}, default: \ttt{comm/}; global setting \\
    When cFileCommunications is selected as parsim communications class:
    specifies the prefix (directory+potential filename prefix) for creating the
    files for cross-partition messages.
\item[parsim-filecommunications-preserve-read] = \textit{<bool>}, default: \ttt{false}; global setting \\
    When cFileCommunications is selected as parsim communications class:
    specifies that consumed files should be moved into another directory
    instead of being deleted.
\item[parsim-filecommunications-read-prefix] = \textit{<string>}, default: \ttt{comm/read/}; global setting \\
    When cFileCommunications is selected as parsim communications class:
    specifies the prefix (directory) where files will be moved after having
    been consumed.
\item[parsim-idealsimulationprotocol-tablesize] = \textit{<int>}, default: \ttt{100000}; global setting \\
    When cIdealSimulationProtocol is selected as parsim synchronization class:
    specifies the memory buffer size for reading the ISP event trace file.
\item[parsim-mpicommunications-mpibuffer] = \textit{<int>}; global setting \\
    When cMPICommunications is selected as parsim communications class:
    specifies the size of the MPI communications buffer. The default is to
    calculate a buffer size based on the number of partitions.
\item[parsim-namedpipecommunications-prefix] = \textit{<string>}, default: \ttt{comm/}; global setting \\
    When cNamedPipeCommunications is selected as parsim communications class:
    selects the prefix (directory+potential filename prefix) where name pipes
    are created in the file system.
\item[parsim-nullmessageprotocol-laziness] = \textit{<double>}, default: \ttt{0.5}; global setting \\
    When cNullMessageProtocol is selected as parsim synchronization class:
    specifies the laziness of sending null messages. Values in the range [0,1)
    are accepted. Laziness=0 causes null messages to be sent out immediately as
    a new EOT is learned, which may result in excessive null message traffic.
\item[parsim-nullmessageprotocol-lookahead-class] = \textit{<string>}, default: \ttt{cLinkDelayLookahead}; global setting \\
    When cNullMessageProtocol is selected as parsim synchronization class:
    specifies the C++ class that calculates lookahead. The class should
    subclass from cNMPLookahead.
\item[parsim-synchronization-class] = \textit{<string>}, default: \ttt{omnetpp::cNullMessageProtocol}; global setting \\
    If parallel-simulation=true, it selects the parallel simulation algorithm.
    The class must implement the cParsimSynchronizer interface.
\item[<object-full-path>.partition-id] = \textit{<string>}; per-object setting \\
    With parallel simulation: in which partition the module should be
    instantiated. Specify numeric partition ID, or a comma-separated list of
    partition IDs for compound modules that span across multiple partitions.
    Ranges ("5..9") and "*" (=all) are accepted too.
\item[print-undisposed] = \textit{<bool>}, default: \ttt{true}; global setting \\
    Whether to report objects left (that is, not deallocated by simple module
    destructors) after network cleanup.
\item[qtenv-default-config] = \textit{<string>}; global setting \\
    Specifies which config Qtenv should set up automatically on startup. The
    default is to ask the user.
\item[qtenv-default-run] = \textit{<int>}, default: \ttt{0}; global setting \\
    Specifies which run (of the default config, see qtenv-default-config) Qtenv
    should set up automatically on startup. The default is to ask the user.
\item[qtenv-extra-stack] = \textit{<double>}, unit=\ttt{B}, default: \ttt{48KiB}; global setting \\
    Specifies the extra amount of stack that is reserved for each activity()
    simple module when the simulation is run under Qtenv.
\item[realtimescheduler-scaling] = \textit{<double>}; global setting \\
    When cRealTimeScheduler is selected as scheduler class: ratio of simulation
    time to real time. For example, scaling=2 will cause simulation time to
    progress twice as fast as runtime.
\item[record-eventlog] = \textit{<bool>}, default: \ttt{false}; per-run setting \\
    Enables recording an eventlog file, which can be later visualized on a
    sequence chart. See eventlog-file= option too.
\item[repeat] = \textit{<int>}, default: \ttt{1}; per-run setting \\
    For scenarios. Specifies how many replications should be done with the same
    parameters (iteration variables). This is typically used to perform
    multiple runs with different random number seeds. The loop variable is
    available as \$\{repetition\}. See also: seed-set= key.
\item[replication-label] = \textit{<string>}, default: \ttt{\#\$\{repetition\}}; per-run setting \\
    Identifies one replication of a measurement (see repeat= and
    measurement-label= as well). This string gets recorded into result files,
    and may be referred to during result analysis.
\item[result-dir] = \textit{<string>}, default: \ttt{results}; per-run setting \\
    Value for the \$\{resultdir\} variable, which is used as the default
    directory for result files (output vector file, output scalar file,
    eventlog file, etc.)
\item[<object-full-path>.result-recording-modes] = \textit{<string>}, default: \ttt{default}; per-object setting \\
    Defines how to calculate results from the @statistic property matched by
    the wildcard. Special values: default, all: they select the modes listed in
    the record= key of @statistic; all selects all of them, default selects the
    non-optional ones (i.e. excludes the ones that end in a question mark).
    Example values: vector, count, last, sum, mean, min, max, timeavg, stats,
    histogram. More than one values are accepted, separated by commas.
    Expressions are allowed. Items prefixed with '-' get removed from the list.
    Example: **.queueLength.result-recording-modes=default,-vector,+timeavg
\item[<object-full-path>.rng-\%] = \textit{<int>}; per-object setting \\
    Maps a module-local RNG to one of the global RNGs. Example: **.gen.rng-1=3
    maps the local RNG 1 of modules matching `**.gen' to the global RNG 3. The
    value may be an expression, with the index and ancestorIndex() operators
    being potentially very useful. The default is one-to-one mapping, i.e. RNG
    k of all modules refer to the global RNG k (for k=0..num-rngs-1).
\item[rng-class] = \textit{<string>}, default: \ttt{omnetpp::cMersenneTwister}; per-run setting \\
    The random number generator class to be used. It can be `cMersenneTwister',
    `cLCG32', `cAkaroaRNG', or you can use your own RNG class (it must be
    subclassed from cRNG).
\item[runnumber-width] = \textit{<int>}, default: \ttt{0}; per-run setting \\
    Setting a nonzero value will cause the \$runnumber variable to get padded
    with leading zeroes to the given length.
\item[<object-full-path>.scalar-recording] = \textit{<bool>}, default: \ttt{true}; per-object setting \\
    Whether the matching output scalars should be recorded. Syntax:
    <module-full-path>.<scalar-name>.scalar-recording=true/false. Example:
    **.queue.packetsDropped.scalar-recording=true
\item[scheduler-class] = \textit{<string>}, default: \ttt{omnetpp::cSequentialScheduler}; global setting \\
    Part of the Envir plugin mechanism: selects the scheduler class. This
    plugin interface allows for implementing real-time, hardware-in-the-loop,
    distributed and distributed parallel simulation. The class has to implement
    the cScheduler interface.
\item[sectionbasedconfig-configreader-class] = \textit{<string>}; global setting \\
    When configuration-class=SectionBasedConfiguration: selects the
    configuration reader C++ class, which must subclass from
    cConfigurationReader.
\item[seed-\%-lcg32] = \textit{<int>}; per-run setting \\
    When cLCG32 is selected as random number generator: seed for the kth RNG.
    (Substitute k for '\%' in the key.)
\item[seed-\%-mt] = \textit{<int>}; per-run setting \\
    When Mersenne Twister is selected as random number generator (default):
    seed for RNG number k. (Substitute k for '\%' in the key.)
\item[seed-\%-mt-p\%] = \textit{<int>}; per-run setting \\
    With parallel simulation: When Mersenne Twister is selected as random
    number generator (default): seed for RNG number k in partition number p.
    (Substitute k for the first '\%' in the key, and p for the second.)
\item[seed-set] = \textit{<int>}, default: \ttt{\$\{runnumber\}}; per-run setting \\
    Selects the kth set of automatic random number seeds for the simulation.
    Meaningful values include \$\{repetition\} which is the repeat loop counter
    (see repeat= key), and \$\{runnumber\}.
\item[sim-time-limit] = \textit{<double>}, unit=\ttt{s}; per-run setting \\
    Stops the simulation when simulation time reaches the given limit. The
    default is no limit.
\item[simtime-precision] = \textit{<custom>}, default: \ttt{ps}; global setting \\
    Sets the resolution for the 64-bit fixed-point simulation time
    representation. Accepted values are: second-or-smaller time units (s, ms,
    us, ns, ps, fs or as), power-of-ten multiples of such units (e.g. 100ms),
    and base-10 scale exponents in the -18..0 range. The maximum representable
    simulation time depends on the resolution. The default is picosecond
    resolution, which offers a range of ~110 days.
\item[simtime-scale] = \textit{<int>}, default: \ttt{-12}; global setting \\
    DEPRECATED in favor of simtime-precision. Sets the scale exponent, and thus
    the resolution of time for the 64-bit fixed-point simulation time
    representation. Accepted values are -18..0; for example, -6 selects
    microsecond resolution. -12 means picosecond resolution, with a maximum
    simtime of ~110 days.
\item[snapshot-file] = \textit{<filename>}, default: \ttt{\$\{resultdir\}/\$\{configname\}-\$\{runnumber\}.sna}; per-run setting \\
    Name of the snapshot file.
\item[snapshotmanager-class] = \textit{<string>}, default: \ttt{omnetpp::envir::cFileSnapshotManager}; global setting \\
    Part of the Envir plugin mechanism: selects the class to handle streams to
    which snapshot() writes its output. The class has to implement the
    cISnapshotManager interface.
\item[tkenv-default-config] = \textit{<string>}; global setting \\
    Specifies which config Tkenv should set up automatically on startup. The
    default is to ask the user.
\item[tkenv-default-run] = \textit{<int>}, default: \ttt{0}; global setting \\
    Specifies which run (of the default config, see tkenv-default-config) Tkenv
    should set up automatically on startup. The default is to ask the user.
\item[tkenv-extra-stack] = \textit{<double>}, unit=\ttt{B}, default: \ttt{48KiB}; global setting \\
    Specifies the extra amount of stack that is reserved for each activity()
    simple module when the simulation is run under Tkenv.
\item[tkenv-plugin-path] = \textit{<path>}; global setting \\
    Specifies the search path for Tkenv plugins. Tkenv plugins are .tcl files
    that get evaluated on startup.
\item[total-stack] = \textit{<double>}, unit=\ttt{B}; global setting \\
    Specifies the maximum memory for activity() simple module stacks. You need
    to increase this value if you get a ``Cannot allocate coroutine stack''
    error.
\item[<object-full-path>.typename] = \textit{<string>}; per-object setting \\
    Specifies type for submodules and channels declared with 'like <>'.
\item[user-interface] = \textit{<string>}; global setting \\
    Selects the user interface to be started. Possible values are Cmdenv and
    Tkenv. This option is normally left empty, as it is more convenient to
    specify the user interface via a command-line option or the IDE's Run and
    Debug dialogs. New user interfaces can be defined by subclassing
    cRunnableEnvir.
\item[<object-full-path>.vector-max-buffered-values] = \textit{<int>}; per-object setting \\
    For output vectors: the maximum number of values to buffer per vector,
    before writing out a block into the output vector file. The default is no
    per-vector limit (i.e. only the total memory limit is in effect)
\item[<object-full-path>.vector-record-eventnumbers] = \textit{<bool>}, default: \ttt{true}; per-object setting \\
    Whether to record event numbers for an output vector. Simulation time and
    value are always recorded. Event numbers are needed by the Sequence Chart
    Tool, for example.
\item[<object-full-path>.vector-recording] = \textit{<bool>}, default: \ttt{true}; per-object setting \\
    Whether data written into an output vector should be recorded.
\item[<object-full-path>.vector-recording-intervals] = \textit{<custom>}; per-object setting \\
    Recording interval(s) for an output vector. Syntax: [<from>]..[<to>],...
    That is, both start and end of an interval are optional, and intervals are
    separated by comma. Example: ..100, 200..400, 900..
\item[warmup-period] = \textit{<double>}, unit=\ttt{s}; per-run setting \\
    Length of the initial warm-up period. When set, results belonging to the
    first x seconds of the simulation will not be recorded into output vectors,
    and will not be counted into output scalars (see option
    **.result-recording-modes). This option is useful for steady-state
    simulations. The default is 0s (no warmup period). Note that models that
    compute and record scalar results manually (via recordScalar()) will not
    automatically obey this setting.
\item[warnings] = \textit{<bool>}, default: \ttt{true}; per-run setting \\
    Enables warnings.
\end{description}

Predefined variables that can be used in config values:

\begin{description}
\item[\$\{runid\}] : \\
    A reasonably globally unique identifier for the run, produced by
    concatenating the configuration name, run number, date/time, etc.
\item[\$\{inifile\}] : \\
    Name of the (primary) inifile
\item[\$\{configname\}] : \\
    Name of the active configuration
\item[\$\{runnumber\}] : \\
    Sequence number of the current run within all runs in the active
    configuration
\item[\$\{network\}] : \\
    Value of the "network" configuration option
\item[\$\{experiment\}] : \\
    Value of the "experiment-label" configuration option
\item[\$\{measurement\}] : \\
    Value of the "measurement-label" configuration option
\item[\$\{replication\}] : \\
    Value of the "replication-label" configuration option
\item[\$\{processid\}] : \\
    PID of the simulation process
\item[\$\{datetime\}] : \\
    Date and time the simulation run was started
\item[\$\{resultdir\}] : \\
    Value of the "result-dir" configuration option
\item[\$\{repetition\}] : \\
    The iteration number in 0..N-1, where N is the value of the "repeat"
    configuration option
\item[\$\{seedset\}] : \\
    Value of the "seed-set" configuration option
\item[\$\{iterationvars\}] : \\
    Concatenation of all user-defined iteration variables in name=value form
\item[\$\{iterationvars2\}] : \\
    Concatenation of all user-defined iteration variables in name=value form,
    plus \$\{repetition\}
\end{description}

%%% Local Variables:
%%% mode: latex
%%% TeX-master: "usman"
%%% End:
