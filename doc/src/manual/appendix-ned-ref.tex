\chapter{NED Reference}
\label{cha:ned-ref}

\section{Syntax}

\subsection{NED file extension}

NED files have the \texttt{.ned} file name suffix. This is mandatory, and
cannot be overridden.

\subsection{NED file encoding}

NED files are read as 8-bit-clean ASCII. This permits NED files saved
in UTF-8, or any character set compatible with ASCII (8859-1, etc).

\begin{note}
    There is no official way to determine the encoding of a NED file. It is up
    to the user to configure the correct encoding in text editors and other
    tools that are used to edit or process NED files.
\end{note}

Keywords and other syntactic elements in NED are ASCII, and identifiers must be
ASCII as well. Comments and string literals may contain characters above
127. String literals (e.g. in parameter values) will be passed to the C++ code
as \ttt{const char *} without any conversion. 

Line ending may be either CR or CRLF, regardless of platform.


\subsection{Reserved words}

Authors have to take care that no reserved words are used for names.
The reserved words of the NED language are:

\ttt{allowunconnected bool channel channelinterface connections const
default double extends false for gates if import index inout input int like
module moduleinterface network output package parameters property simple
sizeof string submodules this true types volatile xml xmldoc}


\subsection{Identifiers}

Identifiers must be composed of letters of the English alphabet (a-z, A-Z),
numbers (0-9) and the underscore ``\_''. Identifiers may only begin with a
letter or underscore.

The recommended way to compose identifiers from multiple words is to
capitalize the beginning of each word (\textit{camel-case}).

\subsection{Case sensitivity}

Keywords and identifiers in the NED language are case sensitive. For example,
\ttt{TCP} and \ttt{Tcp} are two different names. 

\subsection{Comments}

Comments can be placed anywhere in the NED file, with the usual C++
syntax: comments begin with a double slash `//', and last until
the end of the line.

\subsection{Grammar}

The grammar of the NED language can be found in Appendix FIXME.


\section{Packages}

NED supports hierarchical namespaces called \ttt{packages}. The solution is
roughly modelled after Java's packages, with minor changes.


\section{Packages, imports}

A NED file may contain a package declaration:

  package inet.protocols.transport.tcp;

If there is no package declaration, the file is said to be in the
"default package".

Names from other NED files can be referred to either by fully qualified
name ("inet.protocols.network.ip.RoutingTable"), or by short name
("RoutingTable") if the name is visible.

Visible names are:
  - anything from the same package;
  - imported names.

Import directives also have a similar syntax to Java, but they
are more flexible with wildcards. All of the following are legal:

  import  inet.protocols.network.ip.RoutingTable;
  import  inet.protocols.network.ip.*;
  import  inet.protocols.network.ip.Ro*Ta*;
  import  inet.protocols.*.ip.*;
  import  inet.**.RoutingTable;

One asterisk "*" stands for "any character sequence not containing
period"; two asterisks mean "any character sequence which may
contain period". No other wildcards are recognized.

An import not containing wildcard MUST match an existing NED type.
However, it is legal for an import that does contain wildcards
not to match any NED type (although that might generate a warning.)

Inner types may not be referred to outside their enclosing types.


\section{Directory Structure, package.ned}

Like in Java, the directory of a NED file MUST match the package declaration.
However, it is possible to omit directories at the top which don't
contain any NED files (like the "/org/projectname/" directories in Java).

There is a notion of "NED source folders". If a NED Source Folder is
named "src", then a NED file containing the package declaration
"inet.protocols.transport.tcp" must be in the following folder:

   src/inet/protocols/transport/tcp/

If the "inet" and "protocols" directories don't contain any NED files,
they can be omitted:

   src/transport/tcp

and the "src" directory must contain a "package.ned" file which declares
what package it corresponds to:

  src/package.ned:
  package inet.protocols;

All NED files under the "src" directory tree must have package
declarations consistent with that.

"package.ned" files are allowed in other folders as well (as long as
their package declarations are consistent with the NED source
directory's "package.ned"), and any file-level comment in them
will be treated as the package's documentation (similar to Java's
package.html).


\section{Name lookups}

\subsection{Base types and submodules}

Fully qualified names and simple names are accepted. Simple names
are looked up among the inner types of the enclosing type (compound
module), then using imports, then in the same package.


\subsection{Parametric module types ("like" submodules)}

Lookup of the actual module type for "like" submodules differs for normal
lookups. This lookup ignores the imports in the file altogether.
Instead, it collects all modules that support the given interface
and match the given type name string (i.e. end in the same simple name,
or have the same fully qualified name). The result must be exactly
one module type.

The algorithm for parametric channel types works in the same way.


\subsection{Network name in the ini file}

Simple (unqualified) names are tried with the same package as the
ini file is in (provided it's in a NED directory).


\section{The NEDPATH environment variable}

A simulation may need NED files from several "NED source folders".
When a simulation is launched as a separate application, it will expect
to receive the list of NED source folders to scan in the NEDPATH
environment variable (similar to Java's CLASSPATH). NEDPATH should
contain the source folder names separated by a semicolon (";").
On Unix-like systems, colon (":") will also be accepted as separator.

In the future, zip files may become supported in NEDPATH as well;
then the root folder of the zip file will be treated as a NED source
folder.


\section{Inheritance rules}

\begin{itemize}
  \item A simple module may only extend a simple module.
  \item A compound module may only extend a compound module.
  \item A channel may only extend a channel.
  \item A module interface may only extend a module interface (or several module
        interfaces).
  \item A channel interface may only extend a channel interface (or several
        channel interfaces).
\end{itemize}

A \ttt{network} is a shorthand for a compound module with the \ttt{@isNetwork}
property set, so the same rules apply to it as to compound modules.

Inheritance may:
\begin{itemize}
\item add new properties, parameters, gates, inner types, submodules,
connections
\item modify inherited properties, and properties of inherited parameters and gates
\item it may NOT modify inherited submodules, connections and inner types
\end{itemize}

See other rules for specifics.


\subsection{Rules for interfaces}

An interface (module interface or channel interface) may NOT contain
parameter assignment or parameter default value assignment. An interface
may NOT contain inner types (rationale: what use would they be?).

A module type that implements ("like") a module interface is required
to have at least the parameters and the gates defined in the interface,
with the same types.

Regarding properties, parameter properties and gate properties defined
in the interface: the module is required to have those properties,
with at least the same values. It may have additional properties,
and properties may add more keys and values (see property inheritance.)

Similar rules apply to channel types.

\subsection{Rules for parameter assignments}

Default values for parameters may be overridden in subclasses.
Parameter (non-default) assignments may NOT be overridden in subclasses.


\subsection{Rules for gate vector sizes}

Gate vector sizes may NOT be overridden in subclasses.


\subsection{Rules for property inheritance}

Generally, properties may be modified via inheritance. Inheritance may:

\begin{itemize}
    \item add new keys
    \item add/overwrite values for existing keys
    \item remove a value from an existing key (by using the special value '-')
\end{itemize}

\section{Special properties}

\subsection{Rules for @isNetwork}

The @isNetwork property is only accepted on simple modules and
compound modules. The value may be empty, true or false:

\begin{verbatim}
  @isNetwork;
  @isNetwork();
  @isNetwork(true);
  @isNetwork(false);
\end{verbatim}

The empty value corresponds to \ttt{@isNetwork(true)}.

The @isNetwork property does NOT get inherited, that is, a subclass
of a module with @isNetwork set does NOT automatically become a network.
The @isNetwork property needs to be explicitly added to the subclass
to make it a network.

Rationale: subclassing may add gates, whereas networks cannot have gates.

\subsection{Rules for @unit()}

A parameter may have a @unit property to associate it with a physical unit.
The @unit property should contain one string value for the default key.
Examples:
  @unit("s")
  @unit(s)
  @unit("second")
  @unit(second)

When present, values assigned to the parameter must be in the
same or in a compatible (that is, convertible) unit.
Examples:
  double a @unit(s) = 5s;  // OK
  double a @unit(s) = 5;   // error: should be 5s
  double a @unit(s) = 5kg; // error: incompatible unit

@unit behavior for non-numeric parameters (string, XML) is unspecified
(may be ignored or may be an error).

The @unit property of a parameter may NOT be modified via inheritance.

Example:
\begin{verbatim} 
simple A {
    double p @unit(s);
}
simple B extends A {
    p @unit(mW);  // illegal: cannot override @unit
}
\end{verbatim}


\subsection{Rules for @class and @namespace}

@namespace property may be added to the package.ned file

@class

- @unit and expression evaluation wrt units
- rules for resolving an identifier in an expression



%%% Local Variables:
%%% mode: latex
%%% TeX-master: "usman"
%%% End:
