\chapter{Network Graphics And Animation}
\label{cha:graphics}

\section{Display strings}
\label{sec:ch-graphics:display-strings}

\subsection{Display string syntax}

Display strings\index{display strings} specify the arrangement and
appearance of modules in graphical user interfaces (currently only
Tkenv): they control how the objects (compound modules, their
submodules and connections) are displayed. Display strings occur in
NED description's \fpar[ned!keywords!display]{display:}
phrases.

The display string format is a semicolon-separated list of tags.
Each tag consists of a key (usually one letter), an equal sign
and a comma-separated list of parameters, like:

\begin{verbatim}
  "p=100,100;b=60,10,rect;o=blue,black,2"
\end{verbatim}

Parameters may be omitted also at the end and also inside the
parameter list, like:

\begin{verbatim}
  "p=100,100;b=,,rect;o=blue,black"
\end{verbatim}

Module/submodule parameters can be included with the \ttt{\$name} notation:

\begin{verbatim}
  "p=$xpos,$ypos;b=rect,60,10;o=$fillcolor,black,2"
\end{verbatim}

Objects that may have display strings are:
\begin{itemize}
  \item \textit{submodules} -- display string may contain position, arrangement
        (for module vectors), icon, icon color, auxiliary icon, status text,
        communication range (as circle or filled circle), etc.
  \item \textit{connections} -- display string can specify positioning, arrow color,
        arrow thickness
  \item \textit{compound modules} -- display string can specify background color,
        border color, border thickness
  \item \textit{messages} -- display string can specify icon, icon color, etc.
\end{itemize}

The following NED sample shows where to place display strings in the code.

\begin{Verbatim}[commandchars=\\\{\}]
\tbf{module} ClientServer
    \tbf{submodules}:
        pc: Host;
            \tbf{display}: "p=66,55;i=comp"; // position and icon
        server: Server;
            \tbf{display}: "p=135,73;i=server1";
    \tbf{connections}:
        pc.out --> server.in
            \tbf{display} "m=m,61,40,41,28"; // note missing ":"
        server.out --> pc.in
            \tbf{display} "m=m,15,57,35,69";
    \tbf{display}: "o=#ffffff"; // affects background
\tbf{endmodule}
\end{Verbatim}


\subsection{Submodule display strings}


The following table lists the tags used in submodule display strings:

\index{display strings!tags}

\begin{longtable}{|p{6cm}|p{8cm}|}
\hline
% ROW 1
\tabheadcol
\tbf{Tag} & \tbf{Meaning} \\\hline
% ROW 2
\tbf{p=}\textit{xpos},\textit{ypos}
&
{\raggedright Place submodule at (\textit{xpos},\textit{ypos}) pixel position,
with the origin being the top-left corner of the enclosing module.

Defaults: an appropriate automatic layout is where submodules do not overlap.

If applied to a submodule vector, \textit{ring} or \textit{row} layout is
selected automatically.}\\\hline
% ROW 3
\tbf{p=}\textit{xpos},\textit{ypos},\tbf{row},\textit{deltax} &
{\raggedright Used for module vectors. Arranges submodules in a row starting
at (\textit{xpos},\textit{ypos}), keeping \textit{deltax} distances.

Defaults: \textit{deltax} is chosen so that submodules do not overlap.

\tbf{row} may be abbreviated as \tbf{r}.}\\\hline
% ROW 4
\tbf{p=}\textit{xpos},\textit{ypos},\tbf{column},\textit{deltay}
&
{\raggedright Used for module vectors. Arranges submodules in a column starting
at (\textit{xpos},\textit{ypos}), keeping \textit{deltay} distances.

Defaults: \textit{deltay} is chosen so that submodules do not overlap.

\tbf{column} may be abbreviated as \tbf{col} or \tbf{c}.}\\\hline
% ROW 5
\tbf{p=}\textit{xpos},\textit{ypos},\tbf{matrix},
\textit{itemsperrow},\textit{deltax},\textit{deltay}
&
{\raggedright Used for module vectors. Arranges submodules in a matrix starting
at (\textit{xpos},\textit{ypos}), at most \textit{itemsperrow} submodules in
a row, keeping \textit{deltax} and \textit{deltay} distances.

Defaults: \textit{itemsperrow}=5, \textit{deltax,deltay} are chosen so that
submodules do not overlap.

\tbf{matrix} may be abbreviated as \tbf{m}.}\\\hline
% ROW 6
\tbf{p=}\textit{xpos},\textit{ypos},\tbf{ring},\textit{width,height}
&
{\raggedright Used for module vectors. Arranges submodules in an ellipse,
with the top-left corner of the ellipse's bounding box at (\textit{xpos},\textit{ypos}),
with the \textit{width} and \textit{height}.

Defaults: \textit{width,height} are chosen so that submodules do not overlap.

\tbf{ring} may be abbreviated as \tbf{ri}.}\\\hline
% ROW 7
\tbf{p=}\textit{xpos},\textit{ypos},\tbf{exact},\textit{deltax},\textit{deltay}
&
{\raggedright Used for module vectors. Each submodule is placed at
\textit{(xpos+deltax}, \textit{ypos+deltay)}.
This is useful if \textit{deltax} and \textit{deltay} are parameters
 (e.g.:\textit{''p=100,100,exact,\$x,\$y''})
which take different values for each module in the vector.

Defaults: \textit{none}

\tbf{exact} may be abbreviated as \tbf{e} or \tbf{x}.}\\\hline
% ROW 8
\tbf{b=}\textit{width},\textit{height},\tbf{rect}
&
{\raggedright Rectangle with the given \textit{height} and \textit{width}.

Defaults: \textit{width}=40, \textit{height}=24}\\\hline
% ROW 9
\tbf{b=}\textit{width},\textit{height},\tbf{oval}
&
{\raggedright Ellipse with the given \textit{height} and \textit{width}.

Defaults: \textit{width}=40, \textit{height}=24}\\\hline
% ROW 10
\tbf{o=}\textit{fillcolor},\textit{outlinecolor},\textit{borderwidth}
&
{\raggedright Specifies options for the rectangle or oval.
For color notation, see section \ref{sec:ch-graphics:colors}.

Defaults: \textit{fillcolor}=\#8080ff (a lightblue), \textit{outlinecolor}=black,
\textit{borderwidth}=2}\\\hline
% ROW 11
\tbf{i=}\textit{iconname},\textit{color},\textit{percentage}
&
{\raggedright Use the named icon. It can be colorized, and percentage
specifies the amount of colorization.

Defaults: \textit{iconname}: no default -- if no icon name is present, \textit{box} is used;
\textit{color}: no coloring; \textit{percentage}: 30\%}\\\hline
%%
\tbf{is=}\textit{size}
&
{\raggedright Specifies the size of the icon. \textit{size} can be one of
\ttt{l}, \ttt{vl}, \ttt{s} and \ttt{vs} (for large, very large, small, very small).
If this option is present, size cannot be included in the icon name (\ttt{"i="} tag)
with the \ttt{"i=\textit{<iconname>}\_\textit{<size>}"} notation.}\\\hline
%%
\tbf{i2=}\textit{iconname},\textit{color},\textit{percentage}
&
{\raggedright Displays a small "modifier" icon at the top right corner of the primary icon.
Suggested icons are \ttt{status/busy}, \ttt{status/down}, \ttt{status/up},
\ttt{status/asleep}, etc.

The arguments are analoguous with those of \ttt{"i="}.}\\\hline
%%
\tbf{r=}\textit{radius},\textit{fillcolor},\textit{color},\textit{width}
&
{\raggedright Draws a circle (or a filled circle) around the submodule with
the given radius. It can be used to visualize transmission range of
wireless nodes.

Defaults: \textit{radius}=100, \textit{fillcolor}=none, \textit{color}=black,
\textit{width}=1 (unfilled black circle)}\\\hline
%%
\tbf{q=}\textit{queue-object-name}
&
{\raggedright Displays the queue length next to submodule icon.
It expects a \cclass{cQueue} object's name (as set by the \fname{setName()}
method, see section \ref{sec:sim-lib:name}). Tkenv will do a depth-first
search to find the object, and it will find the queue object within submodules
as well.}\\\hline
%%
\tbf{t=}\textit{text},\textit{pos},\textit{color}
&
{\raggedright Displays a short text above or next to the icon.
The text is meant to convey status information
(\textit{"up"}, \textit{"down"}, \textit{"5Kb in buffer"}) or statistics
(\textit{"4 pks received"}). \textit{pos} can be \ttt{"l"}, \ttt{"r"} or \ttt{"t"}
for left, right and top.

%FIXME todo insert: The string may contain \t, \n, \r, \f, \b, \x0A with same meaning as in C/C++

Defaults: \textit{pos}="t", \textit{color}=blue}\\\hline
%%
\tbf{tt=}\textit{tooltip-text}
&
{\raggedright Displays the given text in a tooltip when the user moves
the mouse over the icon. This complements the \ttt{t=} tag, and
lets you display more information that otherwise would not fit on the
screen. }\\\hline

\end{longtable}

Examples:

\begin{verbatim}
  "p=100,60;i=workstation"
  "p=100,60;b=30,30,rect;o=4"
\end{verbatim}



\subsection{Background display strings}

Compound module display strings specify the background. They can contain
the following tags:


\begin{longtable}{|p{6cm}|p{8cm}|}
\hline
%%
\tabheadcol
\tbf{Tag} & \tbf{Meaning}\\
\hline
%%
\tbf{p=}\textit{xpos},\textit{ypos} & Place enclosing module at
(\textit{xpos},\textit{ypos}) pixel position, with (0,0) being
the top-left corner of the window.\\\hline
%%
\tbf{b=}\textit{width},\textit{height},\tbf{rect}
&
{\raggedright Display enclosing module as a rectangle with the given \textit{height}
and \textit{width}.

Defaults: \textit{width,} \textit{height} are chosen automatically}\\\hline
%%
\tbf{b=}\textit{width},\textit{height},\tbf{oval}
&
{\raggedright Display enclosing module as an ellipse with the given \textit{height}
and \textit{width}.

Defaults: \textit{width,} \textit{height} are chosen automatically}\\\hline
%%
\tbf{o=}\textit{fillcolor},\textit{outlinecolor},\textit{borderwidth}
&
{\raggedright Specifies options for the rectangle or oval.
For color notation, see section \ref{sec:ch-graphics:colors}.

Defaults: \textit{fillcolor}=\#8080ff (a lightblue), \textit{outlinecolor}=black,
\textit{borderwidth}=2}\\\hline
%%
\tbf{tt=}\textit{tooltip-text}
&
{\raggedright Displays the given text in a tooltip when the user moves
the mouse over the module name in the top-left corner. }\\\hline
\end{longtable}


\subsection{Connection display strings}

Tags that can be used in connection display strings:

\begin{longtable}{|p{6cm}|p{8cm}|}
\hline
% ROW 1
\tabheadcol
\tbf{Tag} & \tbf{Meaning}\\\hline
% ROW 2
\tbf{m=auto} \linebreak
\tbf{m=north} \linebreak
\tbf{m=west} \linebreak
\tbf{m=east} \linebreak
\tbf{m=south}
&
Drawing mode. Specifies the exact placement of the connection
arrow. The arguments can be abbreviated as a,e,w,n,s.\\\hline
% ROW 3
{\raggedright \tbf{m=manual},\textit{srcpx},\textit{srcpy},\textit{destpx},\textit{destpy}}
&
{\raggedright The manual mode takes four parameters that explicitly specify
anchoring of the ends of the arrow: \textit{srcpx}, \textit{srcpy},
\textit{destpx}, \textit{destpy}.
Each value is a percentage of the width/height of the source/destination
module's enclosing rectangle, with the upper-left corner being
the origin. Thus,
\begin{verbatim}
m=m,50,50,50,50
\end{verbatim}
would connect the centers of the two module rectangles.}\\\hline
% ROW 4
\tbf{o=}\textit{color},\textit{width} &
Specifies the appearance of the arrow.
For color notation, see section \ref{sec:ch-graphics:colors}.

Defaults: \textit{color}=black, \textit{width}=2\\\hline
%%
\tbf{t=}\textit{text},\textit{color}
&
{\raggedright Displays a short text near the connection arrow.
The text may convey status information or connection properties
(\textit{"down"}, \textit{"100Mb"}) or statistics.

%FIXME todo insert: The string may contain \t, \n, \r, \f, \b, \x0A with same meaning as in C/C++

Defaults: \textit{color}=\#005030}\\\hline
%%
\tbf{tt=}\textit{tooltip-text}
&
{\raggedright Displays the given text in a tooltip when the user moves
the mouse over the connection arrow. This complements the \ttt{t=} tag, and
lets you display more information that otherwise would not fit on the
screen. }\\\hline
\end{longtable}


Examples:
\begin{verbatim}
  "m=a;o=blue,3"
\end{verbatim}


\subsection{Message display strings}

Message objects do not store a display string by default, but you can redefine
the \cclass{cMessage}'s \fname{displayString()} method and make it return
one.

\begin{verbatim}
const char *CustomPacket::displayString() const
{
    return "i=msg/packet_vs";
}
\end{verbatim}

This display string affects how messages are shown during animation.
By default, they are displayed as a small filled circle, in one of
8 basic colors (the color is determined as \textit{message kind modulo 8}),
and with the message class and/or name displayed under it
The latter is configurable in the Tkenv Options dialog, and message kind
dependent coloring can also be turned off there.

The following tags can be used in message display strings:

\begin{longtable}{|p{6cm}|p{8cm}|}
\hline
% ROW 1
\tabheadcol
\tbf{Tag} & \tbf{Meaning} \\\hline
% ROW 2
\tbf{b=}\textit{width},\textit{height},\tbf{oval}
&
{\raggedright Ellipse with the given \textit{height} and \textit{width}.

Defaults: \textit{width}=10, \textit{height}=10}\\\hline
% ROW 3
\tbf{b=}\textit{width},\textit{height},\tbf{rect}
&
{\raggedright Rectangle with the given \textit{height} and \textit{width}.

Defaults: \textit{width}=10, \textit{height}=10}\\\hline
% ROW 4
\tbf{o=}\textit{fillcolor},\textit{outlinecolor},\textit{borderwidth}
&
{\raggedright Specifies options for the rectangle or oval.
For color notation, see section \ref{sec:ch-graphics:colors}.

Defaults: \textit{fillcolor}=red, \textit{outlinecolor}=black,
\textit{borderwidth}=1}\\\hline
% ROW 5
\tbf{i=}\textit{iconname},\textit{color},\textit{percentage}
&
{\raggedright Use the named icon. It can be colorized, and percentage
specifies the amount of colorization. If color name is \ttt{"kind"},
a message kind dependent colors is used (like default behaviour).

Defaults: \textit{iconname}: no default -- if no icon name is present, a small
red solid circle will be used;
\textit{color}: no coloring; \textit{percentage}: 30\%}\\\hline
%%
\tbf{tt=}\textit{tooltip-text}
&
{\raggedright Displays the given text in a tooltip when the user moves
the mouse over the message icon.}\\\hline

\end{longtable}

Examples:

\begin{verbatim}
   "i=penguin"
\end{verbatim}

\begin{verbatim}
   "b=15,15,rect;o=white,kind,5"
\end{verbatim}

%FIXME more examples, WITH EXPLANATIONS


\section{Colors}
\label{sec:ch-graphics:colors}

\subsection{Color names}

Any valid Tk color specification is accepted: English color names
(blue, lightgray, wheat) or \textit{\#rgb}, \textit{\#rrggbb} format
(where \textit{r},\textit{g},\textit{b} are hex digits).

It is also possible to specify colors in HSB (hue-saturation-brightness) as
\textit{@hhssbb} (with \textit{h}, \textit{s}, \textit{b} being hex digits).
HSB makes it easier to scale colors e.g. from white to bright red.

You can produce a transparent background by specifying a hyphen (\textit{"-"})
as color.


\subsection{Icon colorization}

The \ttt{"i="} display string tag allows for colorization of icons.
It accepts a target color and a percentage as the degree of colorization.
Percentage has no effect if the target color is missing.
Brightness of icon is also affected -- to keep the original brightness,
specify a color with about 50% brightness (e.g. \#808080 mid-grey,
\#008000 mid-green).

Examples:

\begin{itemize}
  \item \ttt{"i=device/server,gold"} creates a gold server icon
  \item \ttt{"i=misc/globe,\#808080,100"} makes the icon grayscale
  \item \ttt{"i=block/queue,white,100"} yields a "burnt-in" black-and-white icon
\end{itemize}

Colorization works with both submodule and message icons.


\section{The icons}
\label{sec:ch-graphics:icon-library}

\subsection{The image path}

In the current {\opp} version, module icons are PNG or GIF files. The icons shipped
with {\opp} are in the \ttt{images/} subdirectory. Both the graphical NED editor
and Tkenv need the exact location of this directory to load the icons.

Icons are loaded from all directories in the \textit{image path},
a semicolon-separated list of directories.
The default image path is compiled into Tkenv with the value
\ttt{"\textit{omnetpp-dir}/images;./images;./bitmaps"} -- which will work fine
as long as you don't move the directory, and you'll also be able to
load more icons from the \ttt{images/} subdirectory of the current
directory. As people usually run simulation models from the model's
directory, this practically means that custom icons placed in the
\ttt{images/} subdirectory of the model's directory are automatically
loaded.

The compiled-in image path can be overridden with the \ttt{OMNETPP\_IMAGE\_PATH}
environment variable. The way of setting environment variables is system
specific: in Unix, if you're using the bash shell, adding a line

\begin{verbatim}
export OMNETPP_IMAGE_PATH="/home/you/images;./images"
\end{verbatim}

to \ttt{~/.bashrc} or \ttt{~/.bash\_profile} will do; on Windows, environment variables
can be set via the \textit{My Computer --> Properties} dialog.

You can also add to the image path from \ttt{omnetpp.ini}, with
the \ttt{image-path} setting:

\begin{verbatim}
[Tkenv]
image-path = "/home/you/model-framework/images;/home/you/extra-images"
\end{verbatim}

The value should be quoted, otherwise the first semicolon separator will be
interpreted as comment sign, which will result in the rest of the
directories being ignored.


\subsection{Categorized icons}

Since {\opp} 3.0, icons are organized into several categories, represented
by folders. These categories include:

\begin{itemize}
  \item block/ - icons for subcomponents (queues, protocols, etc).
  \item device/ - network devices: servers, hosts, routers, etc.
  \item abstract/ - symbolic icons for various devices
  \item misc/ - node, subnet, cloud, building, town, city, etc.
  \item msg/ - icons that can be used for messages
\end{itemize}

Old (pre-3.0) icons are in the \ttt{old/} folder.

Tkenv and GNED now load icons from subdirectories of all directories
of the image path, and these icons can be referenced from display strings
by naming the subdirectory (subdirectories) as well:
\ttt{"subdir/icon"}, \ttt{"subdir/subdir2/icon"}, etc.

For compatibility, if the display string contains a icon without
a category (i.e. subdirectory) name, {\opp} tries it as "old/icon" as well.

%FIXME If you create new or custom icons...

\subsection{Icon size}

Icons come in various sizes: normal, large, small, very small. Sizes are
encoded into the icon name's suffix: \ttt{\_l}, \ttt{\_s}, \ttt{\_vs}.
In display strings, one can either use the suffix (\ttt{"i=device/router\_l"}),
or the \ttt{"is}" (\textit{icon size}) display string tag ("i=device/router;is=l").


\section{Layouting}
\label{sec:ch-graphics:layouting}

{\opp} implements an automatic layouting feature, using
a variation of the SpringEmbedder algorithm. Modules which have
not been assigned explicit positions via the \ttt{"p="} tag will be
automatically placed by the algorithm.

SpringEmbedder is a graph layouting algorithm based on a physical model.
Graph nodes (modules) repent each other like electric charges
of the same sign, and connections are sort of springs which try
to contract and pull the nodes they're attached to. There is also friction
built in, in order to prevent oscillation of the nodes. The layouting algorithm
simulates this physical system until it reaches equilibrium
(or times out). The physical rules above have been slightly tweaked
to get better results.

The algorithm doesn't move any module which has fixed coordinates.
Predefined row, matrix, ring or other arrangements (defined
via the 3rd and further args of the \ttt{"p="} tag) will be preserved --
you can think about them as if those modules were attached
to a wooden framework so that they can only move as one unit.

Caveats:

\begin{itemize}
  \item If the full graph is too big after layouting, it is scaled
    back so that it fits on the screen, \textit{unless it contains
    any fixed-position module}. (For obvious reasons: if there's a module
    with manually specified position, we don't want to move that one).
    To prevent rescaling, you can specify a sufficiently large bounding
    box in the background display string, e.g. \ttt{"b=2000,3000"}.
  \item Size is ignored by the present layouter, so longish modules
    (such as an Ethernet segment) may produce funny results.
  \item The algorithm is prone to produce erratic results, especially
    when the number of submodules is small, or when using predefined
    (matrix, row, ring, etc) layouts. The "Re-layout" toobar button
    can then be very useful. Larger networks usually produce
    satisfactory results.
\end{itemize}

Parameters to the layouter algoritm (repulsive/attractive forces,
number of iterations,random number seed) can be specified via the
\ttt{"l="} background display string tag. Its current arguments are
(with default values):
\ttt{"l=\textit{<repulsion>}=10,\textit{<attraction>}=0.3,
\textit{<edgelen>}=40,\textit{<maxiter>}=500,\textit{<rng-seed>}"}.
The \ttt{"l="} tag is somewhat experimental and its arguments
may change in further releases.


\section{Enhancing animation}

\subsection{Changing display strings at runtime}

Often it is useful to manipulate the display string at runtime.
Changing colors, icon, or text may convey status change, and
changing a module's position is useful when simulating mobile
networks.

Display strings are stored in \cclass{cDisplayString} objects inside
modules and gates. \cclass{cDisplayString} also lets you manipulate the string.

To get a pointer to the \cclass{cDisplayString} object, you can call
the module's \fname{displayString()} method:

\begin{Verbatim}
cDisplayString *dispStr = displayString();
\end{Verbatim}

%FIXME finish!

\begin{Verbatim}
cDisplayString *bgDispStr = parentModule()->backgroundDisplayString();
\end{Verbatim}

\begin{Verbatim}
cDisplayString *gateDispStr = gate("out")->displayString();
\end{Verbatim}

%FIXME The \cclass{cDisplayString} utility class lets..

As far as \cclass{cDisplayString} is concerned, a display string
(e.g. \ttt{"p=100,125;i=cloud"}) is a string that consist of several
\textit{tags} separated by semicolons, and each tag has a \textit{name}
and after an equal sign, zero or more \textit{arguments} separated by commas.

The class facilitates tasks such as finding out what tags a display string
has, adding new tags, adding arguments to existing tags,
removing tags or replacing arguments. The internal storage method allows
very fast operation; it will generally be faster than direct string manipulation.
The class doesn't try to interpret the display string in any way, nor does
it know the meaning of the different tags; it merely parses the string
as data elements separated by semicolons, equal signs and commas.

An example:

\begin{verbatim}
dispStr->parse("a=1,2;p=alpha,,3");
dispStr->insertTag("x");
dispStr->setTagArg("x",0,"joe");
dispStr->setTagArg("x",2,"jim");
dispStr->setTagArg("p",0,"beta");
ev << dispStr->getString();  // result: "x=joe,,jim;a=1,2;p=beta,,3"
\end{verbatim}


\subsection{Bubbles}

Modules can let the user know about important events (such as a node
going down or coming up) by displaying a bubble with a short message
("Going down", "Coming up", etc.) This is done by the \fname{bubble()} method
of \cclass{cModule}. The method takes the string to be displayed
as a \ttt{const char *} pointer.

An example:

\begin{verbatim}
bubble("Going down!");
\end{verbatim}

If the module contains a lot of code that modifies the display string or
displays bubbles, it is recommended to make these calls conditional
on \ttt{ev.isGUI()}. The \ttt{ev.isGUI()} call returns \textit{false}
when the simulation is run under Cmdenv, so one can make the code skip
potentially expensive display string manipulation.

\begin{verbatim}
if (ev.isGUI())
    bubble("Going down!");
\end{verbatim}



%%% Local Variables:
%%% mode: latex
%%% TeX-master: "usman"
%%% End:
