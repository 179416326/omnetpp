\\appendixchapter{Python API for chart scripts}
\\label{cha:chart-api}

This chapter describes the API of the Python modules available
for chart scripts, provided by the Analysis Tool in the IDE.

Some conventional import aliases appear in code fragments throughout this
chapter, such as `np` for NumPy and `pd` for Pandas.

% --------
% The rest of this file is the output of the tools/extract_chart_api.py
% script (tools/chartapi.txt), pasted here manually, do not edit here!
% --------

%
% generated with extract_chart_api.py
%

\section{results module}
\label{cha:chart-api:results}

\par This module lets the scripts that power the charts in the IDE query
any simulation results and metadata referenced by the .anf file they
are in, returned as as Pandas DataFrames in various formats.\par The \ttt{filter\_expressions} parameter in all functions has the same syntax.
It is always evaluated independently on every loaded result item or metadata entry, and its value
determines whether the given item or piece of metadata is included in the returned \ttt{DataFrame}.
\subsection{get\_results()}
\label{cha:chart-api:results:get-results}

\begin{flushleft}
\ttt{get\_results(filter\_expression="", row\_types=["runattr", "itervar", "config", "scalar", "vector", "statistic", "histogram", "param", "attr"], omit\_unused\_columns=True, start\_time=-inf, end\_time=inf)}
\end{flushleft}


\par Returns a filtered set of results and metadata in CSV-like format.
The items can be any type, even mixed together in a single \ttt{DataFrame}.
They are selected from the complete set of data referenced by the analysis file (\ttt{.anf}),
including only those for which the given \ttt{filter\_expression} evaluates to \ttt{True}.
\subsubsection{Parameters}
\label{sssec:num1}

\begin{itemize}
  \item \textbf{filter\_expression} \textit{(string)}: The filter expression to select the desired items. Example: \ttt{module ={\textasciitilde} "*host*" AND name ={\textasciitilde} "numPacket*"}
  \item \textbf{row\_types}: Optional. When given, filters the returned rows by type. Should be a unique list, containing any number of these strings:
\ttt{"runattr"}, \ttt{"itervar"}, \ttt{"config"}, \ttt{"scalar"}, \ttt{"vector"}, \ttt{"statistic"}, \ttt{"histogram"}, \ttt{"param"}, \ttt{"attr"}
  \item \textbf{omit\_unused\_columns} \textit{(bool)}: Optional. If \ttt{True}, all columns that would only contain \ttt{None} are removed from the returned DataFrame
  \item \textbf{start\_time}, \textbf{end\_time} \textit{(double)}: Optional time limits to trim the data of vector type results.
The unit is seconds, both the \ttt{vectime} and \ttt{vecvalue} arrays will be affected, the interval is left-closed, right-open.

\end{itemize}

\subsubsection{Columns of the returned DataFrame}
\label{sssec:num2}

\begin{itemize}
  \item \textbf{runID} \textit{(string)}:  Identifies the simulation run
  \item \textbf{type} \textit{(string)}: Row type, one of the following: scalar, vector, statistics, histogram, runattr, itervar, param, attr
  \item \textbf{module} \textit{(string)}: Hierarchical name (a.k.a. full path) of the module that recorded the result item
  \item \textbf{name} \textit{(string)}: Name of the result item (scalar, statistic, histogram or vector)
  \item \textbf{attrname} \textit{(string)}: Name of the run attribute or result item attribute (in the latter case, the module and name columns identify the result item the attribute belongs to)
  \item \textbf{attrvalue} \textit{(string)}: Value of run and result item attributes, iteration variables, saved ini param settings (runattr, attr, itervar, param)
  \item \textbf{value} \textit{(double or string)}:  Output scalar or attribute value
  \item \textbf{count}, \textbf{sumweights}, \textbf{mean}, \textbf{min}, \textbf{max}, \textbf{stddev} \textit{(double)}: Fields of the statistics or histogram
  \item \textbf{binedges}, \textbf{binvalues} \textit{(np.array)}: Histogram bin edges and bin values, as space-separated lists. \ttt{len(binedges)==len(binvalues)+1}
  \item \textbf{underflows}, \textbf{overflows} \textit{(double)}: Sum of weights (or counts) of underflown and overflown samples of histograms
  \item \textbf{vectime}, \textbf{vecvalue} \textit{(np.array)}: Output vector time and value arrays, as space-separated lists

\end{itemize}

\subsection{get\_runs()}
\label{cha:chart-api:results:get-runs}

\begin{flushleft}
\ttt{get\_runs(filter\_expression="", include\_runattrs=False, include\_itervars=False, include\_param\_assignments=False, include\_config\_entries=False)}
\end{flushleft}


\par Returns a filtered list of runs, identified by their run ID.
\subsubsection{Parameters}
\label{sssec:num3}

\begin{itemize}
  \item \textbf{filter\_expression}: The filter expression to select the desired runs.
Example: \ttt{runattr:network ={\textasciitilde} "Aloha" AND config:Aloha.slotTime ={\textasciitilde} 0}
  \item \textbf{include\_runattrs}, \textbf{include\_itervars}, \textbf{include\_param\_assignments}, \textbf{include\_config\_entries} \textit{(bool)}:
Optional. When set to \ttt{True}, additional pieces of metadata about the run is appended to the result, pivoted into columns.

\end{itemize}

\subsubsection{Columns of the returned DataFrame}
\label{sssec:num4}

\begin{itemize}
  \item \textbf{runID} (string): Identifies the simulation run
  \item Additional metadata items (run attributes, iteration variables, etc.), as requested

\end{itemize}

\subsection{get\_runattrs()}
\label{cha:chart-api:results:get-runattrs}

\begin{flushleft}
\ttt{get\_runattrs(filter\_expression="", include\_runattrs=False, include\_itervars=False, include\_param\_assignments=False, include\_config\_entries=False)}
\end{flushleft}


\par Returns a filtered list of run attributes.\par The set of run attributes is fixed: \ttt{configname}, \ttt{datetime}, \ttt{experiment},
\ttt{inifile}, \ttt{iterationvars}, \ttt{iterationvarsf}, \ttt{measurement}, \ttt{network},
\ttt{processid}, \ttt{repetition}, \ttt{replication}, \ttt{resultdir}, \ttt{runnumber}, \ttt{seedset}.
\subsubsection{Parameters}
\label{sssec:num5}

\begin{itemize}
  \item \textbf{filter\_expression}: The filter expression to select the desired run attributes.
Example: \ttt{name ={\textasciitilde} *date* AND config:Aloha.slotTime ={\textasciitilde} 0}
  \item \textbf{include\_runattrs}, \textbf{include\_itervars}, \textbf{include\_param\_assignments}, \textbf{include\_config\_entries} \textit{(bool)}:
Optional. When set to \ttt{True}, additional pieces of metadata about the run is appended to the result, pivoted into columns.

\end{itemize}

\subsubsection{Columns of the returned DataFrame}
\label{sssec:num6}

\begin{itemize}
  \item \textbf{runID} \textit{(string)}: Identifies the simulation run
  \item \textbf{name} \textit{(string)}: The name of the run attribute
  \item \textbf{value} \textit{(string)}: The value of the run attribue
  \item Additional metadata items (run attributes, iteration variables, etc.), as requested

\end{itemize}

\subsection{get\_itervars()}
\label{cha:chart-api:results:get-itervars}

\begin{flushleft}
\ttt{get\_itervars(filter\_expression="", include\_runattrs=False, include\_itervars=False, include\_param\_assignments=False, include\_config\_entries=False, as\_numeric=False)}
\end{flushleft}


\par Returns a filtered list of iteration variables.
\subsubsection{Parameters}
\label{sssec:num7}

\begin{itemize}
  \item \textbf{filter\_expression} \textit{(string)}: The filter expression to select the desired iteration variables.
Example: \ttt{name ={\textasciitilde} iaMean AND config:Aloha.slotTime ={\textasciitilde} 0}
  \item \textbf{include\_runattrs}, \textbf{include\_itervars}, \textbf{include\_param\_assignments}, \textbf{include\_config\_entries} \textit{(bool)}:
Optional. When set to \ttt{True}, additional pieces of metadata about the run is appended to the result, pivoted into columns.
  \item \textbf{as\_numeric} \textit{(bool)}: Optional. When set to \ttt{True}, the returned values will be converted to \ttt{double}.
Non-numeric values will become \ttt{NaN}.

\end{itemize}

\subsubsection{Columns of the returned DataFrame}
\label{sssec:num8}

\begin{itemize}
  \item \textbf{runID} \textit{(string)}: Identifies the simulation run
  \item \textbf{name} \textit{(string)}: The name of the iteration variable
  \item \textbf{value} \textit{(string or double)}: The value of the iteration variable. Its type depends on the \ttt{as\_numeric} parameter.
  \item Additional metadata items (run attributes, iteration variables, etc.), as requested

\end{itemize}

\subsection{get\_scalars()}
\label{cha:chart-api:results:get-scalars}

\begin{flushleft}
\ttt{get\_scalars(filter\_expression="", include\_attrs=False, include\_runattrs=False, include\_itervars=False, include\_param\_assignments=False, include\_config\_entries=False, merge\_module\_and\_name=False)}
\end{flushleft}


\par Returns a filtered list of scalar results.
\subsubsection{Parameters}
\label{sssec:num9}

\begin{itemize}
  \item \textbf{filter\_expression} \textit{(string)}: The filter expression to select the desired scalars.
Example: \ttt{name ={\textasciitilde} "channelUtilization*" AND runattr:replication ={\textasciitilde} "\#0"}
  \item \textbf{include\_attrs} \textit{(bool)}: Optional. When set to \ttt{True}, result attributes (like \ttt{unit}
or \ttt{source} for example) are appended to the DataFrame, pivoted into columns.
  \item \textbf{include\_runattrs}, \textbf{include\_itervars}, \textbf{include\_param\_assignments}, \textbf{include\_config\_entries} \textit{(bool)}:
Optional. When set to \ttt{True}, additional pieces of metadata about the run is appended to the DataFrame, pivoted into columns.
  \item \textbf{merge\_module\_and\_name} \textit{(bool)}: Optional. When set to \ttt{True}, the value in the \ttt{module} column
is prepended to the value in the \ttt{name} column, joined by a period, in every row.

\end{itemize}

\subsubsection{Columns of the returned DataFrame}
\label{sssec:num10}

\begin{itemize}
  \item \textbf{runID} \textit{(string)}: Identifies the simulation run
  \item \textbf{name} \textit{(string)}: The name of the scalar
  \item \textbf{value} \textit{(double)}: The value of the scalar
  \item Additional metadata items (result attributes, run attributes, iteration variables, etc.), as requested

\end{itemize}

\subsection{get\_parameters()}
\label{cha:chart-api:results:get-parameters}

\begin{flushleft}
\ttt{get\_parameters(filter\_expression="", include\_attrs=False, include\_runattrs=False, include\_itervars=False, include\_param\_assignments=False, include\_config\_entries=False, merge\_module\_and\_name=False, as\_numeric=False)}
\end{flushleft}


\par Returns a filtered list of parameters - actually computed values of individual \ttt{cPar} instances in the fully built network.\par Parameters are considered "pseudo-results", similar to scalars - except their values are strings. Even though they act
mostly as input to the actual simulation run, the actually assigned value of individual \ttt{cPar} instances is valuable information,
as it is the result of the network setup process. For example, even if a parameter is set up as an expression like \ttt{normal(3, 0.4)}
from \ttt{omnetpp.ini}, the returned DataFrame will contain the single concrete value picked for every instance of the parameter.
\subsubsection{Parameters}
\label{sssec:num11}

\begin{itemize}
  \item \textbf{filter\_expression} \textit{(string)}: The filter expression to select the desired parameters.
Example: \ttt{name ={\textasciitilde} "x" AND module ={\textasciitilde} Aloha.server}
  \item \textbf{include\_attrs} \textit{(bool)}: Optional. When set to \ttt{True}, result attributes (like \ttt{unit} for
example) are appended to the DataFrame, pivoted into columns.
  \item \textbf{include\_runattrs}, \textbf{include\_itervars}, \textbf{include\_param\_assignments}, \textbf{include\_config\_entries} \textit{(bool)}:
Optional. When set to \ttt{True}, additional pieces of metadata about the run is appended to the DataFrame, pivoted into columns.
  \item \textbf{merge\_module\_and\_name} \textit{(bool)}: Optional. When set to \ttt{True}, the value in the \ttt{module} column
is prepended to the value in the \ttt{name} column, joined by a period, in every row.
  \item \textbf{as\_numeric} \textit{(bool)}: Optional. When set to \ttt{True}, the returned values will be converted to \ttt{double}.
Non-numeric values will become \ttt{NaN}.

\end{itemize}

\subsubsection{Columns of the returned DataFrame}
\label{sssec:num12}

\begin{itemize}
  \item \textbf{runID} \textit{(string)}: Identifies the simulation run
  \item \textbf{name} \textit{(string)}: The name of the parameter
  \item \textbf{value} \textit{(string or double)}: The value of the parameter. Its type depends on the \ttt{as\_numeric} parameter.
  \item Additional metadata items (result attributes, run attributes, iteration variables, etc.), as requested

\end{itemize}

\subsection{get\_vectors()}
\label{cha:chart-api:results:get-vectors}

\begin{flushleft}
\ttt{get\_vectors(filter\_expression="", include\_attrs=False, include\_runattrs=False, include\_itervars=False, include\_param\_assignments=False, include\_config\_entries=False, merge\_module\_and\_name=False, start\_time=-inf, end\_time=inf)}
\end{flushleft}


\par Returns a filtered list of vector results.
\subsubsection{Parameters}
\label{sssec:num13}

\begin{itemize}
  \item \textbf{filter\_expression} \textit{(string)}: The filter expression to select the desired vectors.
Example: \ttt{name ={\textasciitilde} "radioState*" AND runattr:replication ={\textasciitilde} "\#0"}
  \item \textbf{include\_attrs} \textit{(bool)}: Optional. When set to \ttt{True}, result attributes (like \ttt{unit}
or \ttt{source} for example) are appended to the DataFrame, pivoted into columns.
  \item \textbf{include\_runattrs}, \textbf{include\_itervars}, \textbf{include\_param\_assignments}, \textbf{include\_config\_entries} \textit{(bool)}:
Optional. When set to \ttt{True}, additional pieces of metadata about the run is appended to the DataFrame, pivoted into columns.
  \item \textbf{merge\_module\_and\_name} \textit{(bool)}: Optional. When set to \ttt{True}, the value in the \ttt{module} column
is prepended to the value in the \ttt{name} column, joined by a period, in every row.
  \item \textbf{start\_time}, \textbf{end\_time} \textit{(double)}: Optional time limits to trim the data of vector type results.
The unit is seconds, both the \ttt{vectime} and \ttt{vecvalue} arrays will be affected, the interval is left-closed, right-open.

\end{itemize}

\subsubsection{Columns of the returned DataFrame}
\label{sssec:num14}

\begin{itemize}
  \item \textbf{runID} \textit{(string)}: Identifies the simulation run
  \item \textbf{name} \textit{(string)}: The name of the vector
  \item \textbf{vectime}, \textbf{vecvalue} \textit{(np.array)}: The simulation times and the corresponding values in the vector
  \item Additional metadata items (result attributes, run attributes, iteration variables, etc.), as requested

\end{itemize}

\subsection{get\_statistics()}
\label{cha:chart-api:results:get-statistics}

\begin{flushleft}
\ttt{get\_statistics(filter\_expression="", include\_attrs=False, include\_runattrs=False, include\_itervars=False, include\_param\_assignments=False, include\_config\_entries=False, merge\_module\_and\_name=False)}
\end{flushleft}


\par Returns a filtered list of statistics results.
\subsubsection{Parameters}
\label{sssec:num15}

\begin{itemize}
  \item \textbf{filter\_expression} \textit{(string)}: The filter expression to select the desired statistics.
Example: \ttt{name ={\textasciitilde} "collisionLength:stat" AND itervar:iaMean ={\textasciitilde} "5"}
  \item \textbf{include\_attrs} \textit{(bool)}: Optional. When set to \ttt{True}, result attributes (like \ttt{unit}
or \ttt{source} for example) are appended to the DataFrame, pivoted into columns.
  \item \textbf{include\_runattrs}, \textbf{include\_itervars}, \textbf{include\_param\_assignments}, \textbf{include\_config\_entries} \textit{(bool)}:
Optional. When set to \ttt{True}, additional pieces of metadata about the run is appended to the DataFrame, pivoted into columns.
  \item \textbf{merge\_module\_and\_name} \textit{(bool)}: Optional. When set to \ttt{True}, the value in the \ttt{module} column
is prepended to the value in the \ttt{name} column, joined by a period, in every row.

\end{itemize}

\subsubsection{Columns of the returned DataFrame}
\label{sssec:num16}

\begin{itemize}
  \item \textbf{runID} \textit{(string)}: Identifies the simulation run
  \item \textbf{name} \textit{(string)}: The name of the vector
  \item \textbf{count}, \textbf{sumweights}, \textbf{mean}, \textbf{stddev}, \textbf{min}, \textbf{max} \textit{(double)}: The characteristic mathematical properties of the statistics result
  \item Additional metadata items (result attributes, run attributes, iteration variables, etc.), as requested

\end{itemize}

\subsection{get\_histograms()}
\label{cha:chart-api:results:get-histograms}

\begin{flushleft}
\ttt{get\_histograms(filter\_expression="", include\_attrs=False, include\_runattrs=False, include\_itervars=False, include\_param\_assignments=False, include\_config\_entries=False, merge\_module\_and\_name=False, include\_statistics\_fields=False)}
\end{flushleft}


\par Returns a filtered list of histogram results.
\subsubsection{Parameters}
\label{sssec:num17}

\begin{itemize}
  \item \textbf{filter\_expression} \textit{(string)}: The filter expression to select the desired histogram.
Example: \ttt{name ={\textasciitilde} "collisionMultiplicity:histogram" AND itervar:iaMean ={\textasciitilde} "2"}
  \item \textbf{include\_attrs} \textit{(bool)}: Optional. When set to \ttt{True}, result attributes (like \ttt{unit}
or \ttt{source} for example) are appended to the DataFrame, pivoted into columns.
  \item \textbf{include\_runattrs}, \textbf{include\_itervars}, \textbf{include\_param\_assignments}, \textbf{include\_config\_entries} \textit{(bool)}:
Optional. When set to \ttt{True}, additional pieces of metadata about the run is appended to the DataFrame, pivoted into columns.
  \item \textbf{merge\_module\_and\_name} \textit{(bool)}: Optional. When set to \ttt{True}, the value in the \ttt{module} column
is prepended to the value in the \ttt{name} column, joined by a period, in every row.

\end{itemize}

\subsubsection{Columns of the returned DataFrame}
\label{sssec:num18}

\begin{itemize}
  \item \textbf{runID} \textit{(string)}: Identifies the simulation run
  \item \textbf{name} \textit{(string)}: The name of the vector
  \item \textbf{count}, \textbf{sumweights}, \textbf{mean}, \textbf{stddev}, \textbf{min}, \textbf{max} \textit{(double)}: The characteristic mathematical properties of the histogram
  \item \textbf{binedges}, \textbf{binvalues} \textit{(np.array)}: The histogram edge locations and the weighted sum of the collected samples in each bin. \ttt{len(binedges) == len(binvalues) + 1}
  \item \textbf{underflows}, \textbf{overflows} \textit{(double)}: The weighted sum of the samples that fell outside of the histogram bin range in the two directions
  \item Additional metadata items (result attributes, run attributes, iteration variables, etc.), as requested

\end{itemize}

\subsection{get\_config\_entries()}
\label{cha:chart-api:results:get-config-entries}

\begin{flushleft}
\ttt{get\_config\_entries(filter\_expression, include\_runattrs=False, include\_itervars=False, include\_param\_assignments=False, include\_config\_entries=False)}
\end{flushleft}


\par Returns a filtered list of config entries. That is: parameter assignment patterns; and global and per-object config options.
\subsubsection{Parameters}
\label{sssec:num19}

\begin{itemize}
  \item \textbf{filter\_expression} \textit{(string)}: The filter expression to select the desired config entries.
Example: \ttt{name ={\textasciitilde} sim-time-limit AND itervar:numHosts ={\textasciitilde} 10}
  \item \textbf{include\_runattrs}, \textbf{include\_itervars}, \textbf{include\_param\_assignments}, \textbf{include\_config\_entries} \textit{(bool)}:
Optional. When set to \ttt{True}, additional pieces of metadata about the run is appended to the result, pivoted into columns.

\end{itemize}

\subsubsection{Columns of the returned DataFrame}
\label{sssec:num20}

\begin{itemize}
  \item \textbf{runID} \textit{(string)}: Identifies the simulation run
  \item \textbf{name} \textit{(string)}: The name of the config entry
  \item \textbf{value} \textit{(string or double)}: The value of the config entry
  \item Additional metadata items (run attributes, iteration variables, etc.), as requested

\end{itemize}

\section{chart module}
\label{cha:chart-api:chart}

\par This module is the interface for displaying data using the built-in plot widgets (i.e. without matplotlib).
It also provides functions to access the properties of the chart object, regardless of the used plotting method.
\subsection{extract\_label\_columns()}
\label{cha:chart-api:chart:extract-label-columns}

\begin{flushleft}
\ttt{extract\_label\_columns(df)}
\end{flushleft}


\par Utility function to make a reasonable guess as to which column of
the given DataFrame is most suitable to act as a chart title and
which ones can be used as legend labels.\par Ideally a "title column" should be one in which all lines have the same
value, and can be reasonably used as a title. Some often used candidates
are: \ttt{title}, \ttt{name}, and \ttt{module}.\par Label columns should be a minimal set of columns whose corresponding
value tuples uniquely identify every line in the DataFrame. These will
primarily be iteration variables and run attributes.
\subsubsection{Returns:}
\label{sssec:num21}

\par A pair of a string and a list; the first value is the name of the
"title" column, and the second one is a list of pairs, each
containing the index and the name of a "label" column.\par Example: \ttt{('title', [(8, 'numHosts'), (7, 'iaMean')])}
\subsection{make\_legend\_label()}
\label{cha:chart-api:chart:make-legend-label}

\begin{flushleft}
\ttt{make\_legend\_label(legend\_cols, row)}
\end{flushleft}


\par Produces a reasonably good label text (to be used in a chart legend) for a result row from
a DataFrame, given a list of selected columns as returned by \ttt{extract\_label\_columns()}.
\subsection{make\_chart\_title()}
\label{cha:chart-api:chart:make-chart-title}

\begin{flushleft}
\ttt{make\_chart\_title(df, title\_col, legend\_cols)}
\end{flushleft}


\par Produces a reasonably good chart title text from a result DataFrame, given a selected "title"
column, and a list of selected "legend" columns as returned by \ttt{extract\_label\_columns()}.
\subsection{plot\_scalars()}
\label{cha:chart-api:chart:plot-scalars}

\begin{flushleft}
\ttt{plot\_scalars(df\_or\_values, labels=None, row\_label=None)}
\end{flushleft}


\par Takes a set of scalar type results and plots them as a "native" bar chart - using the
IDE's built-in drawing widget, and not Matplotlib.\par \ttt{df\_or\_values} can be in either of three different formats:\begin{itemize}
  \item A \ttt{pd.DataFrame} with (at least) \ttt{value}, \ttt{module} and \ttt{name} columns,
  for example as returned by \ttt{results.get\_scalars()}
  \item A \ttt{pd.DataFrame} containing only the numeric values. This is what you
  might get for example from \ttt{pd.pivot\_table()}.
  Each line in the dataframe will be in it's own bar group, similar to
  what \ttt{df.plot.bar()} would draw by default.
  The names in both the horizontal and vertical indices are used as
  labels in the legend.
  \item A simple \ttt{list} or \ttt{np.array} of numbers. They will all be in a single group.

\end{itemize}
\par The optional \ttt{labels} and \ttt{row\_label} parameters are only used in the third
case, as markers for the legend; and ignored in the first two cases.\par Returns \ttt{None}.
\subsection{plot\_vector()}
\label{cha:chart-api:chart:plot-vector}

\begin{flushleft}
\ttt{plot\_vector(label, xs, ys)}
\end{flushleft}


\par Plots a single vector as a "native" line chart - using the
IDE's built-in drawing widget, and not Matplotlib.
\subsubsection{Parameters}
\label{sssec:num22}

\begin{itemize}
  \item \textbf{label}: A string that identifies this line on the plot. It will appear on the legend, so should be unique across invocations.
  \item \textbf{xs}, \textbf{ys}: Lists or \ttt{np.array}s of numbers. Containing the X and Y coordinates (like time and value) of each point on the line,
respectively. There is no requirement on either of these to hold positive, unique or monotonous values, so this method can be used to
draw arbitrary scatter plots as well.

\end{itemize}
\par Can be called repeatedly, each invocation adds a new line to the existing ones on the plot.\par Returns \ttt{None}.
\subsection{plot\_vectors()}
\label{cha:chart-api:chart:plot-vectors}

\begin{flushleft}
\ttt{plot\_vectors(df\_or\_list)}
\end{flushleft}


\par Takes a set of vector type results and plots them as a "native" line chart - using the
IDE's built-in drawing widget, and not Matplotlib.\par \ttt{df\_or\_values} can be in either of three different formats:\begin{itemize}
  \item A \ttt{pd.DataFrame} with (at least) \ttt{vectime} and \ttt{vecvalue} columns,
  for example as returned by \ttt{results.get\_vectors()}
  \item A \ttt{pd.DataFrame} containing only the numeric values. This is what you
  might get for example from \ttt{pd.pivot\_table()}.
  Each line in the dataframe will be in it's own bar group, similar to
  what \ttt{df.plot.bar()} would draw by default.
  The names in both the horizontal and vertical indices are used as
  labels in the legend.
  \item A simple \ttt{list} or \ttt{np.array} of numbers. They will all be in a single group.

\end{itemize}
\par The optional \ttt{labels} and \ttt{row\_label} parameters are only used in the third
case, as markers for the legend; and ignored in the first two cases.\par There is no requirement on the inputs to hold positive, unique or monotonous values,
so this method can be used to draw arbitrary scatter plots as well.\par Returns \ttt{None}.
\subsection{plot\_scatter()}
\label{cha:chart-api:chart:plot-scatter}

\begin{flushleft}
\ttt{plot\_scatter(df, xdata, iso\_column=None)}
\end{flushleft}


\par Pivots a DataFrame of numeric results (like scalars, itervars or statistics fields)
and plots them on a line type chart as a scatter plot. The data points can be optionally
separated into multiple isolines.
\subsubsection{Parameters}
\label{sssec:num23}

\begin{itemize}
  \item \textbf{df}: A DataFrame that holds the data to be pivoted
  \item \textbf{xdata} \textit{(string)}: The name of a column in \ttt{df} which will provide the X coordinate for the data points
  \item \textbf{iso\_column} \textit{(string)}: Optional. The name of a column in \ttt{df} whose the values will be used to separate
the data points into multiple isolines.

\end{itemize}
\par Returns \ttt{None}.
\subsection{plot\_histogram()}
\label{cha:chart-api:chart:plot-histogram}

\begin{flushleft}
\ttt{plot\_histogram(label, edges, values, sumweights=-1.0, lowest=nan, highest=nan)}
\end{flushleft}


\par Plots a single histogram as a "native" chart - using the
IDE's built-in drawing widget, and not Matplotlib.
\subsubsection{Parameters}
\label{sssec:num24}

\begin{itemize}
  \item \textbf{label}: A string that identifies this line on the plot. It will appear on the legend, so should be unique across invocations.
  \item \textbf{edges}, \textbf{values}: Lists or \ttt{np.array}s of numbers. Containing the bin edges and the values of the histogram,
respectively. \ttt{edges} should be one element longer than `values.
  \item \textbf{sumweights}: Optional. In case of a weighted histogram, the total sum of weights of the collected histograms, including
underflows and overflows.

\end{itemize}
\par Can be called repeatedly, each invocation adds a new histogram to the existing ones on the plot.\par Returns \ttt{None}.
\subsection{plot\_histograms()}
\label{cha:chart-api:chart:plot-histograms}

\begin{flushleft}
\ttt{plot\_histograms(df)}
\end{flushleft}


\par Takes a set of histogram type results and plots them as a "native" chart - using the
IDE's built-in drawing widget, and not Matplotlib.\par \ttt{df} is a DataFrame, containing (at least) \ttt{binedges} and \ttt{binvalues} columns,
for example as returned by \ttt{results.get\_histograms()}\par Returns \ttt{None}.
\subsection{set\_property()}
\label{cha:chart-api:chart:set-property}

\begin{flushleft}
\ttt{set\_property(key, value)}
\end{flushleft}


\par Sets one property of the chart to the given value. If there was no property with the given
name (key) yet, it is added.\par Please note that this function does not change or affect the actual chart
object at all (as that is treated strictly as a read-only input); instead, it only
changes some visual property on its view. It is not persistent and is not reflected
in later calls to \ttt{get\_property()} or \ttt{get\_properties()}.
\subsection{set\_properties()}
\label{cha:chart-api:chart:set-properties}

\begin{flushleft}
\ttt{set\_properties(*vargs, **kwargs)}
\end{flushleft}


\par Updates or adds any number of properties of the chart with the given values. Any
existing property values will be overwritten, and any new keys will be inserted.\par This function accepts any number of dictionaries (holding the desired new property
keys and values) as parameters.\par Please note that this function does not change or affect the actual chart
object at all (as that is treated strictly as a read-only input); instead, it only
changes some visual property on its view. This change is not persistent and is not
reflected in later calls to \ttt{get\_property()} or \ttt{get\_properties()}.
\subsection{get\_properties()}
\label{cha:chart-api:chart:get-properties}

\begin{flushleft}
\ttt{get\_properties()}
\end{flushleft}


\par Returns the currently set properties of the chart as a \ttt{dict}
whose keys and values are both all strings.
\subsection{get\_property()}
\label{cha:chart-api:chart:get-property}

\begin{flushleft}
\ttt{get\_property(key)}
\end{flushleft}


\par Returns the value of a single property of the chart, or \ttt{None} if there is
no property with the given name (key) set on the chart.
\subsection{get\_default\_properties()}
\label{cha:chart-api:chart:get-default-properties}

\begin{flushleft}
\ttt{get\_default\_properties()}
\end{flushleft}


\par Returns a string-to-string \ttt{dict} that holds the default property keys and values
for the chart, corresponding to its type.\par This is useful for discovering the basic set of accepted visual property keys
of charts using built-in plot widgets (i.e. not matplotlib).
\subsection{get\_name()}
\label{cha:chart-api:chart:get-name}

\begin{flushleft}
\ttt{get\_name()}
\end{flushleft}


\par Returns the name of the chart as a string.
