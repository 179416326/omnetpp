\appendixchapter{Eventlog File Format}
\label{cha:eventlog-file-format}

This appendix documents the format of the eventlog file. Eventlog
files are written by the simulation (when enabled). Everything
that happens during the simulation gets recorded into the file,
  \footnote{With certain granularity of course, and subject to
  filters that were active during simulation}
so the file can later be used to reproduce the history of the
simulation on a sequence chart, or in some other form.

The file is line-oriented text file. Blank lines and lines beginning
with "\#" (comments) will get ignored. Other lines begin with an
\textit{entry identifier} like \ttt{E} for \textit{Event} or
\ttt{BS} for \textit{BeginSend}, followed by \textit{attribute-identifier}
and \textit{value} pairs. One exception is debug output
(recorded from \ttt{ev<<...} statements), which are represented
by lines that begin with a hypen, and continue with the actual text.

\begin{verbatim}
<file> ::= <line>*
<line> ::= <empty-line> | <user-log-message> | <event-log-entry>
<empty-line> ::= CR LF
<user-log-message> ::= - SPACE <text> CR LF
<event-log-entry> ::= <event-log-entry-type> SPACE <parameters> CR LF
<event-log-entry-type> ::= SB | SE | BU | MB | ME | MC | MD | MR | GC | GD |
                           CC | CD | CS | MS | CE | BS | ES | SD | SH | DM | E
<parameters> ::= (<parameter>)*
<parameter> ::= <name> SPACE <value>
<name> ::= <text>
<value> ::= <boolean> | <integer> | <text> | <quoted-text>
\end{verbatim}

Here is a fragment of an existing eventlog file as an example:

\begin{verbatim}
E # 14 t 1.018454036455 m 8 ce 9 msg 6
BS id 6 tid 6 c cMessage n send/endTx pe 14
ES t 4.840247053855
MS id 8 d t=TRANSMIT,,#808000;i=device/pc_s
MS id 8 d t=,,#808000;i=device/pc_s

E # 15 t 1.025727827674 m 2 ce 13 msg 25
- another frame arrived while receiving -- collision!
CE id 0 pe 12
BS id 0 tid 0 c cMessage n end-reception pe 15
ES t 1.12489449434
BU id 2 txt "Collision! (3 frames)"
DM id 25 pe 15
\end{verbatim}

A correct eventlog also fulfills the following requirements:
\begin{itemize}
   \item simulation events are in increasing event number and simulation time order
   \item FIXME anything more?
\end{itemize}

The various entry types and their supported attributes are as follows:

\begin{verbatim}
SB SimulationBeginEntry // recorded at the first event
{
   v int version       // OMNeT++ version, e.g. 0x401 (=1025) is release 4.1
   rid string runId    // identifies the simulation run
}

SE SimulationEndEntry // optional last line of an event log file
{
}

BU BubbleEntry // display a bubble message
{
   id int moduleId // id of the module which printed the bubble message
   txt string text // displayed message text
}

MB ModuleMethodBeginEntry // beginning of a call to another module
{
   sm int fromModuleId // id of the caller module
   tm int toModuleId   // id of the module being called
   m string method     // C++ method name
}

ME ModuleMethodEndEntry // end of a call to another module
{
}

MC ModuleCreatedEntry // creating a module
{
   id int moduleId              // id of the new module
   c string moduleClassName     // C++ class name of the module
   t string nedTypeName         // fully qualified NED type name
   pid int parentModuleId  -1   // id of the parent module
   n string fullName            // full dotted hierarchic module name
   cm bool compoundModule false // simple or compound module
}

MD ModuleDeletedEntry // deleting a module
{
   id int moduleId // id of the module being deleted
}

MR ModuleReparentedEntry // reparenting a module
{
   id int moduleId         // id of the module being reparented
   p int newParentModuleId // id of the new parent module
}

GC GateCreatedEntry     // gate created
{
   m int moduleId       // module in which the gate was create
   g int gateId         // id of the new gate
   n string name        // gate name
   i int index -1       // gate index if vector, -1 otherwise
   o bool isOutput      // input or output gate
}

GD GateDeletedEntry     // gate deleted
{
   m int moduleId       // module in which the gate was created
   g int gateId         // id of the deleted gate
}

CC ConnectionCreatedEntry // creating a connection
{
   sm int sourceModuleId  // id of the source module identifying the connection
   sg int sourceGateId    // id of the gate at the source module identifying the connection
   dm int destModuleId    // id of the destination module
   dg int destGateId      // id of the gate at the destination module
}

CD ConnectionDeletedEntry // deleting a connection
{
   sm int sourceModuleId // id of the source module identifying the connection
   sg int sourceGateId   // id of the gate at the source module identifying the connection
}

CS ConnectionDisplayStringChangedEntry // a connection display string change
{
   sm int sourceModuleId  // id of the source module identifying the connection
   sg int sourceGateId    // id of the gate at the source module identifying the connection
   d string displayString // the new display string
}

MS ModuleDisplayStringChangedEntry // a module display string change
{
   id int moduleId        // id of the module
   d string displayString // the new display string
}

E EventEntry // an event that is processing of a message
{
   # long eventNumber          // unique event number
   t simtime_t simulationTime  // simulation time when the event occurred
   m int moduleId              // id of the processing module
   ce long causeEventNumber -1 // event number from which the message being processed was sent or -1 if the message was sent from initialize
   msg long messageId          // life time unique id of the message being processed
}

CE CancelEventEntry // canceling an event caused by self message
{
   id long messageId              // id of the message being removed from the FES
   pe long previousEventNumber -1 // event number from which the message being cancelled was sent or -1 if the message was sent from initialize
}

BS BeginSendEntry // beginning to send a message
{
   id long messageId                       // life time unique id of the message being sent
   //XXX what about writing dup, encapsulate/decapsulate as separate lines?
   tid long messageTreeId                  // id of the message inherited by dup
   eid long messageEncapsulationId   -1    // id of the message inherited by encapsulation
   etid long messageEncapsulationTreeId -1 // id of the message inherited by both dup and encapsulation
   c string messageClassName               // C++ class name of the message
   n string messageFullName                // message name
   pe long previousEventNumber -1          // event number from which the message being sent was processed or -1 if the message has not yet been processed before
   k short messageKind     0               // message kind
   p short messagePriority 0               // message priority
   l int64 messageLength   0               // message length in bits
   er bool hasBitError   false             // true indicates the message has bit errors
   d string detail       NULL              // detailed information of message content when recording message data is turned on
   //XXX and many other fields, including message contents
}

ES EndSendEntry // prediction of the arrival of a message
{
   t simtime_t arrivalTime // when the message will arrive to its destination module
}

SD SendDirectEntry // sending a message directly to a destination gate
{
   sm int senderModuleId            // id of the source module from which the message is being sent
   dm int destModuleId              // id of the destination module to which the message is being sent
   dg int destGateId                // id of the gate at the destination module to which the message is being sent
   pd simtime_t propagationDelay  0 // propagation delay that is while the message is propagated through the connection
   td simtime_t transmissionDelay 0 // transmission delay that is while the whole message is sent from the source gate
}

SH SendHopEntry // sending a message through a connection identified by its source module and gate id
{
   sm int senderModuleId            // id of the source module from which the message is being sent
   sg int senderGateId              // id of the gate at the source module from which the message is being sent
   pd simtime_t propagationDelay  0 // propagation delay that is while the message is propagated through the connection
   td simtime_t transmissionDelay 0 // transmission delay that is while the whole message is sent from the source gate
}

DM DeleteMessageEntry // deleting a message
{
   id int messageId               // id of the message being deleted
   pe long previousEventNumber -1 // event number from which the message being deleted was sent or -1 if the message was sent from initialize
}
\end{verbatim}

%%% Local Variables:
%%% mode: latex
%%% TeX-master: "usman"
%%% End:
