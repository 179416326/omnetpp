\appendixchapter{Eventlog File Format}
\label{cha:eventlog-file-format}

This appendix documents the format of the eventlog file. Eventlog
files are written by the simulation (when enabled). Everything
that happens during the simulation is recorded into the file,
  \footnote{With certain granularity of course, and subject to
  filters that were active during simulation}
so the file can later be used to reproduce the history of the
simulation on a sequence chart, or in some other form.

The file is a line-oriented text file. Blank lines and lines beginning
with "\#" (comments) will be ignored. Other lines begin with an
\textit{entry identifier} like \ttt{E} for \textit{Event} or
\ttt{BS} for \textit{BeginSend}, followed by \textit{attribute-identifier}
and \textit{value} pairs. One exception is debug output
(recorded from \ttt{EV<<...} statements), which are represented
by lines that begin with a hyphen, and continue with the actual text.

The grammar of the eventlog file is the following:

\begin{verbatim}
<file> ::= <line>*
<line> ::= <empty-line> | <user-log-message> | <event-log-entry>
<empty-line> ::= CR LF
<user-log-message> ::= - SPACE <text> CR LF
<event-log-entry> ::= <event-log-entry-type> SPACE <parameters> CR LF
<event-log-entry-type> ::= SB | SE | BU | MB | ME | MC | MD | MR | GC | GD |
                           CC | CD | CS | MS | CE | BS | ES | SD | SH | DM | E
<parameters> ::= (<parameter>)*
<parameter> ::= <name> SPACE <value>
<name> ::= <text>
<value> ::= <boolean> | <integer> | <text> | <quoted-text>
\end{verbatim}

The eventlog file must also fulfill the following requirements:
\begin{itemize}
   \item simulation events are in increasing event number and simulation time order
%%   \item FIXME anything more?
\end{itemize}

Here is a fragment of an existing eventlog file as an example:

\begin{filelisting}
E # 14 t 1.018454036455 m 8 ce 9 msg 6
BS id 6 tid 6 c cMessage n send/endTx pe 14
ES t 4.840247053855
MS id 8 d t=TRANSMIT,,#808000;i=device/pc_s
MS id 8 d t=,,#808000;i=device/pc_s

E # 15 t 1.025727827674 m 2 ce 13 msg 25
- another frame arrived while receiving -- collision!
CE id 0 pe 12
BS id 0 tid 0 c cMessage n end-reception pe 15
ES t 1.12489449434
BU id 2 txt "Collision! (3 frames)"
DM id 25 pe 15
\end{filelisting}

\section{Supported Entry Types and Their Attributes}

The following entries and attributes are supported in the eventlog file:

%%
%% generated with tools/stripeventlog.pl from src/eventlog/eventlogentries.txt
%%

\tbf{SB} \textit{(SimulationBegin)}: mandatory first line of an eventlog file

\begin{itemize}
  \item \tbf{v} (\textit{version}, int): OMNeT++ version, e.g. 0x401 (=1025) is release 4.1
  \item \tbf{rid} (\textit{runId}, string): identifies the simulation run
  \item \tbf{b} (\textit{keyframeBlockSize}, int): the distance between keyframes in event numbers
\end{itemize}

\tbf{SE} \textit{(SimulationEnd)}: optional last line of an eventlog file

\begin{itemize}
  \item \tbf{e} (\textit{isError}, bool): specifies if the simulation terminated due to an error
  \item \tbf{c} (\textit{resultCode}, int): the error code in case of an error, otherwise the normal result code
  \item \tbf{m} (\textit{message}, string): human readable description
\end{itemize}

\tbf{BU} \textit{(Bubble)}: display a bubble message

\begin{itemize}
  \item \tbf{id} (\textit{moduleId}, int): id of the module which printed the bubble message
  \item \tbf{txt} (\textit{text}, string): displayed message text
\end{itemize}

\tbf{MB} \textit{(ModuleMethodBegin)}: beginning of a call to another module

\begin{itemize}
  \item \tbf{sm} (\textit{fromModuleId}, int): id of the caller module
  \item \tbf{tm} (\textit{toModuleId}, int): id of the module being called
  \item \tbf{m} (\textit{method}, string): C++ method name
\end{itemize}

\tbf{ME} \textit{(ModuleMethodEnd)}: end of a call to another module

\begin{itemize}
  \item no parameters
\end{itemize}

\tbf{MC} \textit{(ModuleCreated)}: creating a module

\begin{itemize}
  \item \tbf{id} (\textit{moduleId}, int): id of the new module
  \item \tbf{c} (\textit{moduleClassName}, string): C++ class name of the module
  \item \tbf{t} (\textit{nedTypeName}, string): fully qualified NED type name
  \item \tbf{pid} (\textit{parentModuleId}, int): id of the parent module
  \item \tbf{n} (\textit{fullName}, string): full dotted hierarchical module name
  \item \tbf{cm} (\textit{compoundModule}, bool): whether module is a simple or compound module
\end{itemize}

\tbf{MD} \textit{(ModuleDeleted)}: deleting a module

\begin{itemize}
  \item \tbf{id} (\textit{moduleId}, int): id of the module being deleted
\end{itemize}

\tbf{MR} \textit{(ModuleReparented)}: reparenting a module

\begin{itemize}
  \item \tbf{id} (\textit{moduleId}, int): id of the module being reparented
  \item \tbf{p} (\textit{newParentModuleId}, int): id of the new parent module
\end{itemize}

\tbf{GC} \textit{(GateCreated)}: gate created

\begin{itemize}
  \item \tbf{m} (\textit{moduleId}, int): module in which the gate was created
  \item \tbf{g} (\textit{gateId}, int): id of the new gate
  \item \tbf{n} (\textit{name}, string): gate name
  \item \tbf{i} (\textit{index}, int): gate index if vector, -1 otherwise
  \item \tbf{o} (\textit{isOutput}, bool): whether the gate is input or output
\end{itemize}

\tbf{GD} \textit{(GateDeleted)}: gate deleted

\begin{itemize}
  \item \tbf{m} (\textit{moduleId}, int): module in which the gate was created
  \item \tbf{g} (\textit{gateId}, int): id of the deleted gate
\end{itemize}

\tbf{CC} \textit{(ConnectionCreated)}: creating a connection

\begin{itemize}
  \item \tbf{sm} (\textit{sourceModuleId}, int): id of the source module identifying the connection
  \item \tbf{sg} (\textit{sourceGateId}, int): id of the gate at the source module identifying the connection
  \item \tbf{dm} (\textit{destModuleId}, int): id of the destination module
  \item \tbf{dg} (\textit{destGateId}, int): id of the gate at the destination module
\end{itemize}

\tbf{CD} \textit{(ConnectionDeleted)}: deleting a connection

\begin{itemize}
  \item \tbf{sm} (\textit{sourceModuleId}, int): id of the source module identifying the connection
  \item \tbf{sg} (\textit{sourceGateId}, int): id of the gate at the source module identifying the connection
\end{itemize}

\tbf{CS} \textit{(ConnectionDisplayStringChanged)}: a connection display string change

\begin{itemize}
  \item \tbf{sm} (\textit{sourceModuleId}, int): id of the source module identifying the connection
  \item \tbf{sg} (\textit{sourceGateId}, int): id of the gate at the source module identifying the connection
  \item \tbf{d} (\textit{displayString}, string): the new display string
\end{itemize}

\tbf{MS} \textit{(ModuleDisplayStringChanged)}: a module display string change

\begin{itemize}
  \item \tbf{id} (\textit{moduleId}, int): id of the module
  \item \tbf{d} (\textit{displayString}, string): the new display string
\end{itemize}

\tbf{E} \textit{(Event)}: an event that is processing a message

\begin{itemize}
  \item \tbf{\#} (\textit{eventNumber}, eventnumber\_t): unique event number
  \item \tbf{t} (\textit{simulationTime}, simtime\_t): simulation time when the event occurred
  \item \tbf{m} (\textit{moduleId}, int): id of the processing module
  \item \tbf{ce} (\textit{causeEventNumber}, eventnumber\_t): event number from which the message being processed was sent, or -1 if the message was sent from initialize
  \item \tbf{msg} (\textit{messageId}, long): lifetime-unique id of the message being processed
\end{itemize}

\tbf{KF} \textit{(Keyframe)}:

\begin{itemize}
  \item \tbf{p} (\textit{previousKeyframeFileOffset}, int64): file offset of the previous keyframe entry
  \item \tbf{c} (\textit{consequenceLookaheadLimits}, string): consequence lookahead data
  \item \tbf{s} (\textit{simulationStateEntries}, string): simulation state data
\end{itemize}

\tbf{abstract} \textit{(Message)}: base class for entries referring to a message

\begin{itemize}
  \item \tbf{id} (\textit{messageId}, long): lifetime-unique id of the message
  \item \tbf{tid} (\textit{messageTreeId}, long): id of the message inherited by dup
  \item \tbf{eid} (\textit{messageEncapsulationId}, long): id of the message inherited by encapsulation
  \item \tbf{etid} (\textit{messageEncapsulationTreeId}, long): id of the message inherited by both dup and encapsulation
  \item \tbf{c} (\textit{messageClassName}, string): C++ class name of the message
  \item \tbf{n} (\textit{messageName}, string): message name
  \item \tbf{k} (\textit{messageKind}, short): message kind
  \item \tbf{p} (\textit{messagePriority}, short): message priority
  \item \tbf{l} (\textit{messageLength}, int64): message length in bits
  \item \tbf{er} (\textit{hasBitError}, bool): true indicates that the message has bit errors
  \item \tbf{d} (\textit{detail}, string): detailed information of message content when recording message data is turned on
  \item \tbf{pe} (\textit{previousEventNumber}, eventnumber\_t): event number from which the message being cloned was sent, or -1 if the message was sent from initialize
\end{itemize}

\tbf{CE} \textit{(CancelEvent)}: canceling an event caused by a self message

\begin{itemize}
  \item no parameters
\end{itemize}

\tbf{BS} \textit{(BeginSend)}: beginning to send a message

\begin{itemize}
  \item no parameters
\end{itemize}

\tbf{ES} \textit{(EndSend)}: prediction of the arrival of a message

\begin{itemize}
  \item \tbf{t} (\textit{arrivalTime}, simtime\_t): when the message will arrive to its destination module
  \item \tbf{is} (\textit{isReceptionStart}, bool): true indicates the message arrives with the first bit
\end{itemize}

\tbf{SD} \textit{(SendDirect)}: sending a message directly to a destination gate

\begin{itemize}
  \item \tbf{sm} (\textit{senderModuleId}, int): id of the source module from which the message is being sent
  \item \tbf{dm} (\textit{destModuleId}, int): id of the destination module to which the message is being sent
  \item \tbf{dg} (\textit{destGateId}, int): id of the gate at the destination module to which the message is being sent
  \item \tbf{pd} (\textit{propagationDelay}, simtime\_t): propagation delay as the message is propagated through the connection
  \item \tbf{td} (\textit{transmissionDelay}, simtime\_t): transmission duration as the whole message is sent from the source gate
\end{itemize}

\tbf{SH} \textit{(SendHop)}: sending a message through a connection identified by its source module and gate id

\begin{itemize}
  \item \tbf{sm} (\textit{senderModuleId}, int): id of the source module from which the message is being sent
  \item \tbf{sg} (\textit{senderGateId}, int): id of the gate at the source module from which the message is being sent
  \item \tbf{pd} (\textit{propagationDelay}, simtime\_t): propagation delay as the message is propagated through the connection
  \item \tbf{td} (\textit{transmissionDelay}, simtime\_t): transmission duration as the whole message is sent from the source gate
\end{itemize}

\tbf{CM} \textit{(CreateMessage)}: creating a message

\begin{itemize}
  \item no parameters
\end{itemize}

\tbf{CL} \textit{(CloneMessage)}: cloning a message either via the copy constructor or dup

\begin{itemize}
  \item \tbf{cid} (\textit{cloneId}, long): lifetime-unique id of the clone
\end{itemize}

\tbf{DM} \textit{(DeleteMessage)}: deleting a message

\begin{itemize}
  \item no parameters
\end{itemize}

%%% Local Variables:
%%% mode: latex
%%% TeX-master: "usman"
%%% End:
