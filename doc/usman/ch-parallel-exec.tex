\chapter{Parallel Execution}
\label{cha:parallel-execution}

\section{{\opp} support for parallel execution}

\subsection{Introduction to Parallel Discrete Event Simulation}

{\opp} supports parallel execution\index{parallel execution} of large
simulations. The following paragraphs provide a very brief (and thus
not very accurate) picture of the problems and methods of parallel
discrete event simulation (PDES\index{PDES}). Interested readers --
and those who are thinking about doing PDES with {\opp} -- are
strongly encouraged to look into the literature.


For parallel execution, the model is to be partitioned to several
segments that will be simulated independently on different hosts or
processors. Each segment will have its own local Future Event Set,
thus they will maintain local simulation times. The main issue with
parallel simulations is keeping segments synchronized in order to
avoid violating causality of events. Without synchronization, a
message sent by one segment could arrive in another segment when the
simulation time in the receiving segment has already passed the
timestamp (arrival time) of the message. This would break
causality\index{event!causality} of events in the receiving segment.

There are mainly three different methods used for synchronizing
segments:
\begin{enumerate}
  \item{\textbf{Conservative synchronization}\index{conservative
      synchronization} exploits knowledge about when segments send
    messages to other segments, and uses 'null' messages to propagate
    this info to other segments. This may speed up simulation, since
    e.g. if a segment knows it won't receive any messages from other
    segments until $t+\Delta t$ simulation time, it may advance until
    $t+\Delta t$ without the need for external synchronization.
    Conservative syncronization requires modifications to existing
    models, i.e., inserting code which sends out the 'null' messages.
    Conservative simulation tends to converge to sequential simulation
    (slowed down by communication between segments) if there's not
    enough parallelism in the model, or parallelism is not exploited
    by sending enough 'null' messages.}

  \item{\textbf{Optimistic synchronization}\index{optimistic
      synchronization} allows incausalities to occur, but detects and
    repairs them. Repairing involves rollbacks to a previous state,
    sending out anti-messages to cancel messages sent out during the
    period that is being rolled back, etc.  Optimistic synchronization
    is extremely difficult to implement, because it requires periodic
    state saving and the ability to restore previous states. In any
    case, implementing optimistic synchronization in {\opp} would
    require -- in addition to a more complicated simulation kernel --
    writing significantly more complex simple\index{module!simple}
    module code from the user.  Optimistic synchronization may be slow
    in cases of excessive rollbacks.}

  \item{\textbf{Statistical synchronization}\index{statistical
      synchronization} is a compromise where segments do not exchange
    individual messages but distributions of the traffic flow
    characteristics. While conservative and optimistic synchronization
    are exact methods (they produce exactly the same results as the
    corresponding sequential simulation would), this is certainly not
    true for statistical synchronization where the results may contain
    error introduced by the statistical nature of the synchronization.
    Statistical synchronization does not require changes to existing
    models, only the insertion of extra modules, called
    \textit{''statistical interfaces''}, therefore it is significantly
    easier to implement than either conservative or optimistic. In
    addition to easier implementation, there is a potential for much
    larger speedup than with conservative or optimistic, because the
    method is much less sensitive to communication delay between
    processors running the segments. Therefore, for parallel
    simulation on a cluster of workstations, statistical
    synchronisation may be the only feasible method.}
\end{enumerate}





\subsection{{\opp} support for parallel simulation}

The simulation kernel makes it possible to send messages from one
segment to another. A message can contain arbitrarily complex data
structures; these are transferred transparently, even between hosts of
different architectures. The simulation kernel provides a simple
synchronization mechanism (\textit{syncpoints}, available through the
\fname{syncpoint()} call) that can ensure that causality is kept when
sending messages between segments. Syncpoints correspond to
\textit{null messages} found in the literature.

Message sending and syncpoints enable one to implement conservative
PDES and also Statistical Synchronization. The simulation class
library contains objects that explicitly support the implementation
of models using Statistical Synchronization.


High level debugging is supported by saving the textual output
from remote segments to a log file and/or relaying them to a
single console.

{\opp} supports flexible partitioning of the
model\index{model!partitioning}. In the NED language, by using
\textit{machine parameters} you can specify \textit{logical
  hosts}\index{logical hosts} for different modules at any level of
the module hierarchy of the network. You map logical hosts to physical
ones in the ini file; if you map several logical hosts into the same
physical machine, they will be merged into a single {\opp} process.


One may choose between using the MPI\index{MPI} (Message Passing
Interface) and the PVM3\index{PVM} (Parallel Virtual Machine Version
3) libraries for communication between hosts. Both libraries are
portable and widely used in university and research environments. MPI
is newer though and considered to be the successor of PVM. You can
find MPI and PVM readings in the Reference.





\subsection{Syncpoints}

\textbf{Overview}


When running a simulation in parallel, different segments of
the model execute as independent UNIX processes, typically on
separate hosts. Since the hosts can be of different speed and
the simulated model segments can be of different complexity,
at a given moment the model times of different segments will
differ: some segments are ahead of the others and some lag behind.
Suppose that a message is sent from segment A to segment B which
is ahead of A in model time. If B processed the message, causality
would break. This should never happen.


The solution built in {\opp} is the following. Segment A must know in
advance when it will send the next message to segment B and announce
it with the \fname{syncpoint()} call. The simulation kernel sends the
syncpoint to segment B. When segment B's model time reaches the
specified time, segment B's simulation kernel blocks execution until
the promised message arrives from A. Then the simulation continues,
typically but not necessarily with the message that has just been
received from A.

In the reverse case when A is ahead of B, A's message arrived at B
before it has reached the syncpoint. In this case, there is no problem
and the syncpoint is just an unnecessary precaution. B just inserts
the message in its future event set, clears the syncpoint and
continues execution.


\textbf{The syncpoint API}


The \fname{syncpoint()} call takes two arguments. The first is the
model time when (or more precisely: when of after when) the simple
module will send a message to another simple module in a different
segment. The second argument is a gate given with its number or its
name. The gate implicitly specifies the destination segment to
synchronize with.

\begin{verbatim}
syncpoint(t, "outgate");
\end{verbatim}


\textbf{Details of the syncpoint implementation}


If the destination module is in the same segment, the call is ignored.
(This makes it possible to run models designed to execute in parallel
as a single process, without any modification.) Each segment keeps a
list of syncpoints sent to it (time + gate), ordered by time.
Simulation executes normally until it comes to an event that has a
time \textit{definitely past} the first syncpoint in the list. That
event is not processed, but the segment goes into a blocked state.
While the segment is blocked, it listens for messages arriving from
other segments. (In the actual implementation, passive wait is used so
a blocked segment doesn't use much CPU time.) Each message that
arrives deletes the first syncpoint in the list that matches its gate.
The segment goes out of the blocked state when -- because of deletions
-- the first syncpoint in the list is no longer past the event in
question. Then the simulation goes on normally, either with the newly
arrived message (or the earliest of them) or the original event. A
message that arrives outside of the blocked state also causes deletion
of the first matching syncpoint in the list; this case corresponds to
the reverse case when the sender segment is ahead of the receiving
segment in model time.


\textbf{Deadlock}


It is possible to cause deadlock\index{parallel simulation!deadlock}
with carelessly placed syncpoints.  Suppose that segment A declares a
syncpoint at 10s with segment B, but it will actually send a message
only at 10.5s. If segment B does the same to segment A, a nice
deadlock is created. {\opp} makes no effort to detect or prevent such
deadlocks; it is entirely the simulation programmer's task to take
care that deadlocks do not occur.




\section{Configuring a simulation for parallel execution}

\subsection{Configuring {\opp}}

\subsubsection{Choosing between MPI and PVM}


You must let {\opp} know if you want to use MPI\index{MPI} or
PVM\index{PVM}. This can be configured in the \texttt{[General]}
section of the ini file, via te
\fpar[parallel-system]{parallel-system=} entry. Its value can be
``PVM'' or ``MPI''; it defaults to ``MPI''.

\begin{Verbatim}[commandchars=\\\{\}]
; file: omnetpp.ini
;...
\textbf{[General]}
parallel-system = MPI
\end{Verbatim}


\subsubsection{Mapping logical machines to physical ones}
\label{sec:ch-parallel-exec:machine-mapping}

The \fpar[ned!keywords!on]{on:} phrases in the NED descriptions
specify the logical machine(s) on which the module is run. The machine
parameters are mapped to physical machines in the \texttt{[Machines]}
section of the configuration file:

\begin{Verbatim}[commandchars=\\\{\}]
; file: omnetpp.ini
;...
\tbf{[Machines]}
node1 = whale.hit.bme.hu
node4 = whale.hit.bme.hu
node2 = puppis.hit.bme.hu
node3 = dolphin.hit.bme.hu
;...
\end{Verbatim}


\subsubsection{Configuration of the slaves}

Slave processes can be configured in the \texttt{[Slaves]} section of
the configuration file:

\begin{Verbatim}[commandchars=\\\{\}]
; file: omnetpp.ini
\textbf{[Slaves]}
write-slavelog=
slavelog-file=
module-messages=
errmsgs-to-console=
infomsgs-to-console=
modmsgs-to-console=
\end{Verbatim}

Screen input/output of the slaves is re-routed to the console.
However, any file I/O is done in the local file system of each host.



\subsection{Setting up PVM}

\subsubsection{The PVM virtual machine}


The \ttt{pvmhosts} file is used by PVM to describe what computers will
participate in the virtual machine, where the executables (in our
case, the {\opp} programs) are located on each computer, what working
directories should be set etc.


It is advisable to have a common, shared directory mounted on all
participating hosts; this eliminates the tedious work of having to
copy files to all hosts again and again.


If using {\opp}, it is a good idea to write separate \texttt{pvmhosts}
files for each simulation program. Since simulation programs are
typically in separate directories, the \texttt{pvmhosts} file in each
directory can name that directory as executables directory and working
directory for each host. This way, there is no need to create soft
links or explicitly name directories in the {\opp} ini files.


Each line in the \texttt{pvmhosts} file describes one host. An example
line (this all should be a \textit{single line}!):

\begin{verbatim}
whale ip=whale.hit.bme.hu lo=andras
   dx=/home/andras/pvm/pvm
   ep=/home/andras/omnetpp/projects/fddi
   wd=/home/andras/omnetpp/projects/fddi
\end{verbatim}

To start PVM with this configuration:

\begin{verbatim}
cd ~/omnetpp/projects/fddi
pvm pvmhosts
\end{verbatim}

\subsubsection{Configuration and running}

The user must have PVM installed on the hosts he is going to
run segments on.


To set up a simulation for distributed execution, the user must:
\begin{enumerate}
\item{set the \fvar{PVM\_ROOT} environment variable}
\item{link the simulation executable with \ttt{sim\_pvm} instead of
  \ttt{sim\_std} (You can do it by setting \fvar{PVM\_SUPPORT} to
  \ttt{yes} in a \fprog{opp\_makemake}-generated makefile.)}
\item{set \ttt{distributed=true} in the \ttt{[General]} section of the
  configuration file.}
\item{specify the logical-hosts-to-physical-machines mapping in the
  \ttt{[Machines]} section}
\item{copy the simulation executable and the configuration file to
  each host if they have physically different disks}
\item{start \ttt{pvm} with an appropriate \ttt{pvmhosts} file}
\item{start the simulation executable on the host which is supposed to
  be the console. That process will start up the program on the other
  hosts too and do the simulation.}
\end{enumerate}

The first machine is called ``console'' or ``master'', the
others are called ``slaves''.


\subsubsection{If there are problems...}


PVM programs in general are more difficult to get running than
ordinary programs. Wrong settings in the PVM configuration files
can cause various problems, for example. Also, parallel programs
are a lot harder to test and debug.


What can you do if your distributed {\opp} simulation won't
work?
\begin{itemize}
  \item{First of all, check the \ttt{pvmhosts} file to see if PVM
    looks for the executables in the right directories on all hosts
    and the working directories are right (typically, the same
    directory as the executable's).}
  \item{In the ini file, enable writing the \ttt{slave.log} files for
    the slave processes and check what is written into them.}
  \item{You can try enabling the \fdef{SINGLE\_HOST} define in the
    \ttt{sim/pvm/pvmmod.cc} source file. This will make {\opp} run all
    segments of the distributed simulation on the local host, making
    things a lot easier to manage. }
  \item{Also, try the defining \fdef{PVM\_DEBUG} at the same place: it
    enables a lot of \fname{ev.printf()}s in the code interfacing with
    PVM, so it is easier to spot where the problems are.}
  \item{PVM itself also has an environment variable which, if set,
    causes the PVM library to print out debugging information.
    However, this is very low-level information, it will rarely be
    useful.}
\end{itemize}




\subsection{Setting up MPI}

TBD





\section{Statistical synchronization}
\begin{sloppypar}
\subsection{The description of the Statistical Synchronization Method (SSM)}
\end{sloppypar}

Similarly to other parallel discrete event simulation methods, the
model to be simulated - which is more or less a precise representation
of a real system - is divided into segments, where the segments
usually describe the behaviour of functional units of the real system.
The communication of the segments can be represented by sending and
receiving various messages. The simulators of the segments are
executed by separate processors.


The communication of these segments is simulated with appropriate
interfaces. The messages generated in a given segment and to be
processed in a different segment are not transmitted there, but the
output interfaces collect the statistical data of them.  If the input
interfaces generate messages for the segments according to the
statistical characteristics of the messages collected by the proper
output interfaces, the segments with their input- and output
interfaces can be simulated separately, giving statistically correct
results. The events in one segment have not the same effect in other
segments as in the original model, so the results collected during the
SSM\index{SSM} are not exact. The precision depends on the
segmentation, on the accuracy of statistics collection and
regeneration, and on the frequency of the statistics exchange among
the processors.


\textbf{Segmentation}


The segments of the simulator\index{simulation!parallel segments} are
executed by separate processors, they have their own, independent
virtual times. Because the interactions among segments are performed
by the statistical parameters of these interactions, the segmentation
should be done so, that the overwhelming majority of the interactions
should happen within the segments and not among them. This speeds up
the so-called inter-segment transients\index{inter-segment transients}
and improves the accuracy as well.


\textbf{Timing of statistics exchange}


Asynchronous statistics exchange\index{statistic!asynchronous
  exchange} means, that whenever a statistical result collection in an
output interface is ready, it is applied - after mapping and
correction - in the proper input interface.  This is clearly more
efficient, than the so-called synchronous statistics
exchange\index{statistic!synchronous exchange}, which means, that we
delay the application of collected values until all the output
interfaces get ready with the result collection. Frequent statistics
exchange makes the inter-segment transient faster, but the lower
sample numbers makes the estimation - and the whole simulation - less
precise.


To learn more about SSM, see [PON92] and [PON93].





\subsection{Using SSM in {\opp}}

{\opp} directly supports the implementation of statistical
interface\index{statistical interface} with the following classes:

\cclass{cLongHistogram}, \cclass{cDoubleHistogram}, \cclass{cPSquare}, \cclass{cPar}.


%%% Local Variables:
%%% mode: latex
%%% TeX-master: "usman"
%%% End:
